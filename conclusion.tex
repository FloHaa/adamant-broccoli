\chapter{Conclusion and outlook}
This thesis gives an overview of the power supply structure in South Africa and there need for flexible energy sources. It is shown that SA has high solar radiations which solar power plants can use to support the South African system demand by a specified generation profile. It was shown that solar energy can generate power during the full year and support thereby the key demand times. Thereby the use of specified energy storage is indispensable. State of art for CSP TES is the direct (CR) or indirect (PTC) use of molten salt as storage medium. It showed that for the extend power supply of PV system a battery storage is most practically and that thereby Li-Ion has the best storage characteristics. It was shown, that the PTC technology has disadvantages in form of maximum system temperature and optical losses compared to CR technology, which particular occur by high load covering. Nevertheless all simulated solar power plant reached a high annual power output. As it was to expect, the stand alone PV system is the most economically solar power plant, but can not reach high load coverings without a EES. From the LCOE  results, it became clear that CSP technology with a TES is much more economically than a PV system with a EES. Also with the expected economically development till 2050 will not change that fact.

\subsubsection{Conclusion of main findings in this thesis}
\begin{itemize}
\item It was shown, that SA has a great potential for solar power due to high solar irradiation. The annual GHI of Upigton is more than \SI{2280}{\kilo\watt\hour\per\square\metre\per\year}, while  DNI is \SI{2621}{\kilo\watt\hour\per\square\metre\per\year}. Other regions in SA has even higher irradiation values. 
\item Currently SA has a total capacity of \SI{1059.05}{\mega\watt} of PV, which can supply a part of the South African system load during the day but come to a standstill when the evening peak begin.
\item In the first half of 2015 SA had a total of \SI{1083}{\mega\watt\hour} unserved energy between 7:00 and 22:00 due to load shedding. This showed South Africas need for flexible power generation during daytime that espacially needs supply in the evening peak. 
\item Therefore a generation profile was designed to support the SA system load at there relevant demand times while generating electricity the full day. To cover \SI{90}{\percent} of this generation profile the solar power plants needs to produce more than \SI{640575}{\mega\watt\hour} per year.
\item For CSP applications is the CR and PTC technology in combination with molten salt TES state of the art for long duration generation.
\item Multi-crystalline PV cells dominated the worldwide PV production in 2014 and the most reasonable EES solution for a PV system is a Li-Ion battery storage.
\item The cost of capital has a decisive influence on the LCOE of solar power plants.
\item A stand alone PV system with \SI{100}{\mega\watt} power\textsubscript{AC} can deliver \SI{233.69}{\giga\watt\hour} electricity in the first year at a LCOE of \SI{77.61}{USD/MWh}. Therewith is PV in SA today more economically then new-build mid-merit fossil fired power plants.
\item With the most effective configuration to cover \SI{90}{\percent} of the generation profile the CR uses a SM of 3.0 and 14~h of TES for a LCOE result of \SI{154.75}{USD/MWh}. Thereby it showed that this is the economically best solution to cover the generation profile.
\item The most effective configuration of PTC is SM of 5.0 and 12~h of TES which results a LCOE of \SI{167.45}{USD/MWh} while covering more than \SI{90}{\percent} of the generation profile .
\item The PTC technology has rising optical efficiency losses during low irradiance angles. Therefore the SM is much higher then that of the CR.
\item The high solar multiples of the CSP technology, which is necessary for power generation in Winter led to high unused solar field capacities in the summer months.
\item To covering about \SI{90}{\percent} of the generation profile the PV system with EES needs a PVM of 2.8 and 7~h of EES which leads to a LCOE result of \SI{417.57}{USD/MWh} and therewith the less economically solution.
\item Costs of Li-Ion battery decreased the last years annually at about \SI{8}{\percent} which led to average battery cost for marked-leading actors of about \SI{300}{\usd/\kilo\watt\hour} in 2014. It can be assumed that the specific costs of Li-Ion battery decreases to \SI{200}{\usd/\kilo\watt\hour} in 2020 and \SI{150}{\usd/\kilo\watt\hour} in 2030.
\item Specific investment costs of CSP could decrease by about \SI{54}{\percent} and the PV system costs by about \SI{60}{\percent} till 2050.
\item Therefrom it was shown that PV systems with EES will not come more economically than CSP with TES in the next decades.
\end{itemize}
\subsubsection{Outlook}
This thesis made it clear that PV is in combination with EES economically not reasonable for long duration generation supply which is actually needed in SA. CSP against it, can support the SA power supply flexible and much more economical useful with long duration generation profiles. 

Nevertheless it is clear that PV can produce energy more favourable during the day time then CSP and can thereby reduce the coal fired energy production in SA. Therefore PV is getting essential for the SA power supply, but it needs generation balancing power plants in combination. CSP is such a flexible technology. Therefore CSP could balancing in the future changeable renewable generation in the SA power supply. 
\clearpage
