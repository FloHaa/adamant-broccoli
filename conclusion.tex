\chapter{Conclusion and outlook}
%This thesis gives an overview of the power supply structure in South Africa and there need for flexible energy sources. It is shown that SA has high solar radiations which solar power plants can use to support the South African system demand by a specified generation profile. It was shown that solar energy can generate power during the full year and support thereby the key demand times. Thereby the use of specified energy storage is indispensable. State of art for CSP TES is the direct (CR) or indirect (PTC) use of molten salt as storage medium. It showed that for the extend power supply of PV system a battery storage is most practically and that thereby Li-Ion has the best storage characteristics. It was shown, that the PTC technology has disadvantages in form of maximum system temperature and optical losses compared to CR technology, which particular occur by high load covering. Nevertheless all simulated solar power plant reached a high annual power output. As it was to expect, the stand alone PV system is the most economically solar power plant, but can not reach high load coverings without a EES. From the LCOE  results, it became clear that CSP technology with a TES is much more economically than a PV system with a EES. Also with the expected economically development till 2050 will not change that fact.

This thesis gives an overview of the energy infrastructure in South Africa and demonstrates the country's need for flexible energy sources. It has been shown that South Africa has high solar radiation, which solar power plants can use to support the South African system demand given a specific load profile. It was shown that solar energy can generate electricity throughout the year while providing support during times of peak demand. To achieve this, energy storage is indispensable. The state of art for CSP TES is the direct (CR) or indirect (PTC) use of molten salt as the storage medium. It has been shown that, for PV systems, a battery storage is most feasible and that Li-ion has the best storage characteristics in such an application. It was also shown that PTC technology is limited by its maximum system temperature and optical losses when compared to CR technology, which are particularly apparent with high load coverage. Nevertheless, all simulated solar power plants reached a high annual power output. As expected, the stand-alone PV system is the most economical solar power plant, but it cannot reach high load coverage without an EES. Based on the LCOE, it could be shown that CSP technology with TES is much more economical than a PV system with EES. The expected cost developments to 2050 will not change this.


\subsubsection{Main findings}
%\begin{itemize}
%\item It was shown, that SA has a great potential for solar power due to high solar irradiation. The annual GHI of Upigton is more than \SI{2280}{\kilo\watt\hour\per\square\metre\per\year}, while  DNI is \SI{2621}{\kilo\watt\hour\per\square\metre\per\year}. Other regions in SA has even higher irradiation values. 
%\item Currently SA has a total capacity of \SI{1059.05}{\mega\watt} of PV, which can supply a part of the South African system load during the day but come to a standstill when the evening peak begin.
%\item In the first half of 2015 SA had a total of \SI{1083}{\mega\watt\hour} unserved energy between 7:00 and 22:00 due to load shedding. This showed South Africas need for flexible power generation during daytime that espacially needs supply in the evening peak. 
%\item Therefore a generation profile was designed to support the SA system load at there relevant demand times while generating electricity the full day. To cover \SI{90}{\percent} of this generation profile the solar power plants needs to produce more than \SI{640575}{\mega\watt\hour} per year.
%\item For CSP applications is the CR and PTC technology in combination with molten salt TES state of the art for long duration generation.
%\item Multi-crystalline PV cells dominated the worldwide PV production in 2014 and the most reasonable EES solution for a PV system is a Li-Ion battery storage.
%\item The cost of capital has a decisive influence on the LCOE of solar power plants.
%\item A stand alone PV system with \SI{100}{\mega\watt} power\textsubscript{AC} can deliver \SI{233.69}{\giga\watt\hour} electricity in the first year at a LCOE of \SI{77.61}{\usd/\mega\watt\hour}. Therewith is PV in SA today more economically then new-build mid-merit fossil fired power plants.
%\item With the most effective configuration to cover \SI{90}{\percent} of the generation profile the CR uses a SM of 3.0 and 14~h of TES for a LCOE result of \SI{154.75}{\usd/\mega\watt\hour}. Thereby it showed that this is the economically best solution to cover the generation profile.
%\item The most effective configuration of PTC is SM of 5.0 and 12~h of TES which results a LCOE of \SI{167.45}{\usd/\mega\watt\hour} while covering more than \SI{90}{\percent} of the generation profile .
%\item The PTC technology has rising optical efficiency losses during low irradiance angles. Therefore the SM is much higher then that of the CR.
%\item The high solar multiples of the CSP technology, which is necessary for power generation in Winter led to high unused solar field capacities in the summer months.
%\item To covering about \SI{90}{\percent} of the generation profile the PV system with EES needs a PVM of 2.8 and 7~h of EES which leads to a LCOE result of \SI{417.57}{\usd/\mega\watt\hour} and therewith the less economically solution.
%\item Costs of Li-Ion battery decreased the last years annually at about \SI{8}{\percent} which led to average battery cost for marked-leading actors of about \SI{300}{\usd/\kilo\watt\hour} in 2014. It can be assumed that the specific costs of Li-Ion battery decreases to \SI{200}{\usd/\kilo\watt\hour} in 2020 and \SI{150}{\usd/\kilo\watt\hour} in 2030.
%\item Specific investment costs of CSP could decrease by about \SI{54}{\percent} and the PV system costs by about \SI{60}{\percent} till 2050.
%\item Therefrom it was shown that PV systems with EES will not come more economically than CSP with TES in the next decades.
%\end{itemize}

\begin{itemize}
\item It was shown that South Africa has great potential for solar power development due to high solar irradiation. The annual GHI of Upington is more than \SI{2280}{\kilo\watt\hour\per\square\metre\per\year}, while the DNI is \SI{2621}{\kilo\watt\hour\per\square\metre\per\year}. Other regions of South Africa have even higher irradiation values.
\item Currently South Africa has a total capacity of \SI{1059.05}{\mega\watt} of PV, which can supply a part of the South African system load during the day, but not during the evening demand peak.

%\item In the first half of 2015 SA had a total of \SI{1083}{\mega\watt\hour} unserved energy between 7:00 and 22:00 due to load shedding. This showed South Africas need for flexible power generation during daytime that espacially needs supply in the evening peak. 
\item In the first half of 2015, South Africa had a total of \SI{1083}{\mega\watt\hour} unserved demand between 7:00 and 22:00 as evidenced by load shedding. This showed South Africa's need for flexible power generation that can serve the evening peak demand. 

%\item Therefore a generation profile was designed to support the SA system load at there relevant demand times while generating electricity the full day. To cover \SI{90}{\percent} of this generation profile the solar power plants needs to produce more than \SI{640575}{\mega\watt\hour} per year.

\item To this end, a load profile was designed to correlate to the system load at the relevant high-demand times. To cover \SI{90}{\percent} of this load profile, the solar power plants need to produce more than \SI{640575}{\mega\watt\hour} per year.

%\item For CSP applications is the CR and PTC technology in combination with molten salt TES state of the art for long duration generation.

\item For CSP applications, CR and PTC technologies, in combination with molten salt TES, are the state of the art for long duration generation.

%\item Multi-crystalline PV cells dominated the worldwide PV production in 2014 and the most reasonable EES solution for a PV system is a Li-Ion battery storage.

\item Multi-crystalline PV cells dominated worldwide PV production in 2014 and the most reasonable EES solution for a PV system is  Li-ion battery storage.

%\item The cost of capital has a decisive influence on the LCOE of solar power plants.
\item The cost of capital has a decisive influence on the LCOE of solar power plants.

%\item A stand alone PV system with \SI{100}{\mega\watt} power\textsubscript{AC} can deliver \SI{233.69}{\giga\watt\hour} electricity in the first year at a LCOE of \SI{77.61}{\usd/\mega\watt\hour}. Therewith is PV in SA today more economically then new-build mid-merit fossil fired power plants.

\item A stand-alone PV system with \SI{100}{\mega\wattsac} output can deliver \SI{233.69}{\giga\watt\hour} electricity in the first year at a LCOE of \SI{77.61}{\usd/\mega\watt\hour}. This means that PV in South Africa is already more economical than new-build mid-merit fossil-fired power plants.

%\item With the most effective configuration to cover \SI{90}{\percent} of the generation profile the CR uses a SM of 3.0 and 14~h of TES for a LCOE result of \SI{154.75}{\usd/\mega\watt\hour}. Thereby it showed that this is the economically best solution to cover the generation profile.

\item The most effective CR configuration to cover \SI{90}{\percent} of the prescribed load profile has a SM of \num{3.0} and \SI{14}{\hour} of TES. The resulting LCOE of \SI{154.75}{\usd/\mega\watt\hour} makes this the most economical solution.

%\item The most effective configuration of PTC is SM of 5.0 and 12~h of TES which results a LCOE of \SI{167.45}{\usd/\mega\watt\hour} while covering more than \SI{90}{\percent} of the generation profile .

\item The most effective configuration of PTC has a SM of \num{5.0} and \SI{12}{\hour} of TES, which results in a LCOE of \SI{167.45}{\usd/\mega\watt\hour} while covering more than \SI{90}{\percent} of the generation profile .

%\item The PTC technology has rising optical efficiency losses during low irradiance angles. Therefore the SM is much higher then that of the CR.
\item Parabolic trough technology has rising optical efficiency losses during periods with low irradiance angles, leading to much higher SM than with CR of equivalent output.

%\item The high solar multiples of the CSP technology, which is necessary for power generation in Winter led to high unused solar field capacities in the summer months.

\item The high solar multiples of the CSP technology, which are necessary for power generation in winter, led to high unused solar field capacities in the summer months.


%\item To covering about \SI{90}{\percent} of the generation profile the PV system with EES needs a PVM of 2.8 and 7~h of EES which leads to a LCOE result of \SI{417.57}{\usd/\mega\watt\hour} and therewith the less economically solution.

\item To cover \SI{90}{\percent} of the prescribed load profile, the PV system with EES needs a PVM of \num{2.8} and \SI{7}{\hour} of EES which leads to a LCOE result of \SI{417.57}{\usd/\mega\watt\hour} and therewith the less economically solution.


%\item Costs of Li-Ion battery decreased the last years annually at about \SI{8}{\percent} which led to average battery cost for marked-leading actors of about \SI{300}{\usd/\kilo\watt\hour} in 2014. It can be assumed that the specific costs of Li-Ion battery decreases to \SI{200}{\usd/\kilo\watt\hour} in 2020 and \SI{150}{\usd/\kilo\watt\hour} in 2030.

\item Costs of Li-ion battery storage have decreased about \SI{8}{\percent} annually, resulting in average battery costs for market-leading actors of about \SI{300}{\usd/\kilo\watt\hour} in 2014. It is reasonable to expect specific costs of Li-ion battery storage to decrease further to \SI{200}{\usd/\kilo\watt\hour} by 2020 and \SI{150}{\usd/\kilo\watt\hour} by 2030.

%\item Specific investment costs of CSP could decrease by about \SI{54}{\percent} and the PV system costs by about \SI{60}{\percent} till 2050.

\item Specific investment costs of CSP could decrease by about \SI{54}{\percent} and PV system costs by about \SI{60}{\percent} by 2050.

%\item Therefrom it was shown that PV systems with EES will not come more economically than CSP with TES in the next decades.

\item Based on these observations, PV systems with EES will not become more economical than CSP with TES in the coming decades.
\end{itemize}

\subsubsection{Outlook}
%This thesis made it clear that PV is in combination with EES economically not reasonable for long duration generation supply which is actually needed in SA. CSP against it, can support the SA power supply flexible and much more economical useful with long duration generation profiles. 

In this thesis, it has been shown that PV in combination with EES is not economically feasible for the long-duration generation that is currently needed in South Africa. Concentrating solar power can, by contrast, flexibly support the South African electricity system and do so much more economically in long-duration generation applications. 

%Nevertheless it is clear that PV can produce energy more favourable during the day time then CSP and can thereby reduce the coal fired energy production in SA. Therefore PV is getting essential for the SA power supply, but it needs generation balancing power plants in combination. CSP is such a flexible technology. Therefore CSP could balancing in the future changeable renewable generation in the SA power supply. 

Nevertheless, it is clear that PV can generate electricity more effectively than CSP during daylight, and can thereby help reduce dependency on coal-fired energy production in South Africa. Photovoltaics will become essential for the South African electricity system, but this will need balancing generating capacity, something CSP is flexible enough to provide. A sensible combination of these two technologies in the South African energy mix seems a reasonable prospect.

\clearpage