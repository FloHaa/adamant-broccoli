\chapter{Conclusion and outlook}
This thesis gives an overview of the power supply structure in South Africa and there need for flexible renewable energy sources. It is shown that SA has a high solar radiation which solar power plants in combination with storage applications can use to supply the bottleneck of there daily load curve in the evening hours. 


solar power plants in combination with storage applications are able to support the SA power supply where it is necessary.

For supporting the South African power supply over the full year and especially during the critical evening time the PV system in combination with a EES based on Li-ion technology is much more expansive than a CR or PTC power plant with TES technology. 

State of art in CSP TES molten salt

over the full day and esspasialy where it is essential 

PV state of art is mono and multi Si 

\subsubsection{Conclusion of main findings in this thesis}
\begin{itemize}
\item It was shown, that SA has a great potential for solar power due to high solar irradiation. The annual GHI of Upigton is more than \SI{2280}{\kilo\watt\hour\per\square\metre\per\year}, while  DNI is \SI{2621}{\kilo\watt\hour\per\square\metre\per\year}. Other regions in SA has even higher irradiation values. 
\item In the first half of 2015 SA had a total of \SI{1083}{\mega\watt\hour} unserved energy between 7:00 and 22:00 due to load shedding. This showed South Africas need for flexible power generation during daytime that espacially needs supply in the evening peak. 
\item Currently SA has a total capacity of \SI{1059.05}{\mega\watt} from PV system, which can supply a part of the South African system load during the day but come to a standstill when the evening peak begin, which then needs to be covert by OCGTs or pumped storage.
\item status of solar power generation in SA
\item generating a simplified daily generation schedule to supply at relevant times 
\item marked situation
\item financing 
\item PV, low LCOE, low load covering
\item 
\item CR best option for high load covering over the full year
\item PTC is also a good option but is reduced by cosine effect in Winter
\item LCOE could be higher for CSP if no load shape and turbine runs as long as it can


\end{itemize}


Spinnennetzdiagramm (cost reduction potential, Current LCOE, market maturity, contribution to grid stability, capacity factor, load curve covering factor)
Auf Christophs arbeit verweisen und auf hybridPV:
http://www.belectric.com/de/hybrid/

Sehr optemistischer LCOE bei PV ohne speicher.
Keine kühlung 
verlust durch alterung nur bedingt betrachtet



\subsubsection{Limitation}

\subsubsection{Outlook}

%
% Hier beginnen die Verzeichnisse.
%
\clearpage
