\chapter{Introduction}
... % Hier muss noch was hin.

\section{Background and motivation}
%South Africa's energy demand is constantly strong growing while the electricity generation is mainly based on coal, there major resource. Due to growing electricity demand and insufficient investment in the power system in the last decades, led to serious blackouts in late 2005. This forced Eskom Holdings SOC Ltd. (Eskom), the South African public utility, to implement scheduled load shedding during peak hours from 2008 on. The South African electricity supply needs new supporting generation capacities during the day and especially in the evening hours. 

South Africa's energy demand is growing, while electricity generation is mainly based on coal, a major local resource. That growing demand, coupled with insufficient investment in the electrical system in past decades, has led to serious blackouts, beginning in 2005. This forced Eskom Holdings SOC Ltd. (Eskom), the South African public utility, to implement scheduled load shedding during peak hours from 2008 on. The South African electricity supply needs new generation capacity, especially in the evening hours. 

%The high dependency on and the scarcity of fossil resources, the intention to reduce greenhouse gas emissions and other environmental pollutants, rising fuel prices, and the predicted increase of energy demand are also forcing South Africa to promote investments in renewable energy sources. A key section is thereby the high solar radiation in South Africa.

The high dependency on and the scarcity of fossil resources, the intention to reduce greenhouse gas emissions and other environmental pollutants, rising fuel prices, and the predicted increase of energy demand are also forcing South Africa to promote investments in renewable energy sources, among them, solar energy.

Solar radiation may not be used directly. It must first be captured and converted into other forms of energy, primarily heat and electricity. The conversion from solar radiation into electrical energy is by two mechanisms: photovoltaic conversion and thermal conversion. For solar thermal electricity (STE) generation, it is common to concentrate the solar radiation with reflectors onto a receiver where a heat-transfer fluid (HTF) circulates. The fluid is heated up to high temperatures to be used in a thermodynamic cycle to generate electrical power. This technology is called concentrating solar power (CSP). The direct conversion from solar radiation into electricity is based on the photovoltaic effect, using photovoltaic (PV) panels. 

Solar radiation is present only during the daytime. South Africa needs energy over the whole day and especially in the evening hours, when solar radiation is absent. In order to maintain supply, solar plants need energy storage. 

%For CSP power plants storage application are state of the art and can nowerdays assures a constant power generation 24/7. For PV application there are currently large-scale hybrid systems with fossil backup on the marked \cite{BELECTRIC2015} to assure uninterrupted power supply of a PV based system, but pure large-scale PV power plants in connection with energy storage are currently unusual and was before not profound developed. 

For CSP power plants, modern storage systems can assure availability \enquote{24/7}. For photovoltaics, there are currently large-scale hybrid systems with fossil-fueled backup on the market \cite{BELECTRIC2015} to assure uninterrupted supply, but pure large-scale PV power plants in connection with energy storage remain unusual. In this work, large-scale is equivalent to utility-scale.

%It can be expected that plain PV systems are more favourable then CSP systems. But in combination with a energy storage, which is necessary for long duration generation, this situation might be different because of more convenient storage applications of CSP.

Photovoltaic systems are technologically simpler than concentrating solar power systems. When combined with energy storage, however, the choice is not so clear; CSP systems have storage options which can be more attractive, both from an operational and financial perspective.

%This thesis provides therefore a comparison of large-scale solar power plants to support the South African power supply constantly and enhanced on high demand times.

\section{Objective of this thesis}
%The objective of this thesis is the technical and economic comparison of concentrating solar and photovoltaic power plants and the most reasonable storage systems for both technologies. This comparison is limited to South Africa and there specified needs in power generation.

The objective of this thesis is the technical and economic comparison of concentrating solar and photovoltaic power plants, and the evaluation of storage systems for both technologies. This comparison is limited to South Africa under consideration of its specific circumstances and requirements.

A cost prognosis is also considered, in order to provide infrastructure planning guidance.

The research question is as follows:
\begin{quote}
%Which technology, concentrating solar or photovoltaic power plants, in combination with suitable storage systems, will prevail in South Africa in a technical and economical comparison? What is the forecasted cost development of these technologies in the medium- and long-term?
Which technology, concentrating solar or photovoltaic power plants, in combination with suitable storage systems, will prevail in South Africa in a technical and economical comparison? What is the predicted cost development of these technologies in the medium- and long-term?
\end{quote}


\section{Approach}

This thesis follows a bottom-up design, meaning that all essential information is to be found at the beginning of the study.

%The first step of this work is to describe the power supply situation in South Africa from there demand over the generation and distribution to there major supply problems. Thereby the reader gets a impression of the need in South Africa for constant, but flexible power generation.

The basis for the analysis is a study of the electric power sector in South Africa, from demand, to generation, to distribution, to the country's serious supply problems. The reader should understand the need in South Africa for consistent but flexible power generation.

%The second step is to describe the special solar irradiation situation in South Africa and the located potential for generating power from solar power plants. This will also reveal the downside of  direct solar generation without provide energy in storage for the South African electricity supply structure and thereby the need for storage application of solar power plants. A generation profile for the solar power plants will be defined to support the South African power supply at any time of the day but with a main focus on times with high demand. 

Next, the special solar irradiation characteristics of South Africa are considered and described in detail, in order to show the potential, if any, for generating electricity from solar power plants. Disadvantages of direct solar generation will be highlighted, paying particular attention to lack of night generation in the absence of storage. A model plant output profile will be defined for the purpose of supporting the South African electrical system at any time of day, while paying particular attention to peak demand periods.

%The next step is to describe the currently commercial and mature situation of solar power generation and there individual suitable storage technology. Thereby the exact technology type for the comparison will be selected. In order to compare the solar power plants equally they will be simulated under same conditions and requirements (location, weather data and generation profile) by using the System Advisor Model (SAM) software from the National Renewable Energy Laboratory (NREL) \cite{NREL2015}. SAM is designed to model and simulate performance and financial parameter of different types of renewable energy. It also can model PV and CSP application with belonging storage applications. The solar power plant will be modeled and simulated individually and compared in the following. The main performance and financial metrics for the comparison will be the amount of covering of the generation profile and the resulting LCOE.

After a survey of the state of the art, technologies used in the comparison will be chosen. To make the comparison fair, the model plants will be simulated under the same conditions and requirements (location, weather data and generation profile) using the System Advisor Model (SAM) software from the National Renewable Energy Laboratory (NREL) \cite{NREL2015}. SAM is designed to model and simulate performance and produce financial projections for different types of renewable energy. It can model PV and CSP applications with attached storage. The solar power plant will be modelled and simulated individually, then compared. The main performance and financial metrics for the comparison will be the output profile coverage and the resulting levelized cost of electricity.

%At the end the future cost development of the solar power plants will be discussed and compared with new-built conventional power plant costs in South Africa.

Finally, the future cost development of the solar power plants will be discussed and compared with new-build conventional power plant costs in South Africa.

\section{Sources}
In order to assure a academic work with high quality standard, all of the information and data used in this study were gathered from scientific publications or databases, renowned institutions/departments and libraries.

The following sources were considered to obtain scientific provable information:
%\begin{itemize}
%\item Solar Thermal Energy Research Group (STERG)
%\item Centre for Renewable and Sustainable Energy Studies (CRSES)
%\item Library of Stellenbosch University
%\item Library of FH Technikum Wien (University of Applied Science Vienna)
%\item Library of TU Munich
%\item Library of Science Direct
%\end{itemize}
\begin{itemize}
\item Solar Thermal Energy Research Group (STERG)
\item Centre for Renewable and Sustainable Energy Studies (CRSES)
\item Stellenbosch University Library
\item FH Technikum Wien (University of Applied Science Vienna) Library
\item TU Munich Library
\item Elsevier ScienceDirect
\end{itemize}

%Further information was mainly obtained from Eskom, the Department of Energy (DoE), NREL, BP and the International Energy Agency (IEA).

Further information was mainly obtained from Eskom, the Department of Energy of the Republic of South Africa (DoE), the US National Renewable Energy Laboratory (NREL), BP plc and the International Energy Agency (IEA).

%To assure a high quality of this academic work, all units are described in SI units system and applied after the current state of art \cite{Blankenburg2011}.

All units are SI units.