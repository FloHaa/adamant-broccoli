\chapter{Introduction}
%

\section{Background and motivation}
%It is generally known that the worldwide energy demand is nearly entirely covered by fossil energy sources such as coal, oil, natural gas and uranium. Also, the consequences of fossil fuels to the environmental, social and economic development are globally well known and discussed. Especially the greenhouse gas emissions need to be reduced for the purpose to dilute their effects on climate change. Furthermore, the world population and the energy demand is rising. By 2035 the world population is projected to reach 8.7 billion, this means an additional 1.6 billion people who will need energy. This leads to a rise in primary energy consumption by 37~\% between 2013 and 2035 \cite{BP2015a}. 
Currently, global energy demand is nearly entirely covered by fossil energy sources such as coal, oil, natural gas and uranium. This has consequences for environmental, social and economic development. In particular, greenhouse gas emissions must be reduced to slow climate change. Furthermore, world population and energy demand are rising. By 2035, Earth's population is projected to reach 8.7 billion, meaning an additional 1.6 billion people will need energy. This will lead to a rise in primary energy consumption of \SI{37}{\percent} between 2013 and 2035 \cite{BP2015a}.

%Especially newly industrialized countries with a rapid population growth like China and India showed a strong rising primary energy consumption. Also the prognosticated primary energy demand in the newly industrialized country South Africa (SA), officially the Republic of South Africa, is growing from 1~639.83~TWh in 2012 to 2~163.18~TWh (including solid biomass and waist) in 2040. Coal is the mainstay of the South African energy system, meeting around 70~\% of primary energy demand and accounting for more than 90~\% of domestic electricity output. The prognosticated electricity demand is also rising by more than 70~\% from 212~TWh in 2012 to 364~TWh in 2040 \cite{IEA2014f}.

Newly industrialized countries with rapid population growth, such as China and India, have rapidly rising primary energy consumption. Predicted primary energy demand in the Republic of South Africa, a newly industrialized country, is expected to grow from \SI{1639.83}{\tera\watt\hour} in 2012 to \SI{2163.18}{\tera\watt\hour} (including solid biomass and waste) in 2040. Coal is the mainstay of the South African energy system, meeting around \SI{70}{\percent} of primary energy demand and accounting for more than \SI{70}{\percent} of domestic electricity output. Electricity demand is expected to rise by more than \SI{70}{\percent} from \SI{212}{\tera\watt\hour} in 2012 to \SI{364}{\tera\watt\hour} in 2040 \cite{IEA2014f}.

%Depending on the growing electricity demand and missing investments in the power supply system in the last decades, the reserve margin in electricity capacity declined from almost 40~\% in 1990 to 8~\% at these days \cite{Trollip2014,Eskom2015}. The region of Western Cape experienced serious blackouts in late 2005 dependence on combination of inadequate reserve margin, insufficient reliability, a stressed system and system element failures. Those were the causes which forced Eskom (Eskom Holdings SOC Ltd.), the South African public utility, to start scheduled load shedding during peak hours, from 2007 on \cite{Trollip2014}. The economic and social impacts are clearly noticeable already. Load shedding was also the main trigger for Bureau for Economic Research (BER) senior economist Hugo Pienaar to revise down the 2015 GDP growth forecast from 2.9~\% to 1.9~\% for SA. \cite{Bisseker2015}.

Due to growing electricity demand and insufficient investment in the power system in the last decades, the reserve margin in electricity capacity declined from almost \SI{40}{\percent} in 1990 to \SI{40}{\percent} currently \cite{Trollip2014,Eskom2015}. The region of Western Cape experienced serious blackouts in late 2005 due to a combination of inadequate reserve margin, poor reliability, a stressed system and system element failures. This forced Eskom Holdings SOC Ltd., the South African public utility, to implement scheduled load shedding during peak hours from 2007 on \cite{Trollip2014}. The economic and social impacts are clearly noticeable already. This lack of supply certainty was also the main reason that Bureau for Economic Research (BER) senior economist Hugo Pienaar revised the 2015 GDP growth forecast for South Africa downward from \SI{2.9}{\percent} to \SI{2.9}{\percent} \cite{Bisseker2015}.

%The high dependency and the scarcity of fossil resources, the intention to reduce greenhouse gas emissions and other environmental pollutants, the rising fuel price, the predicted increase of energy demand, are also forcing SA to promote the investments in renewable energy sources. In 2009, the South African government began to implement feed-in tariffs (FITs) for renewable energies. These were later rejected in favor of competitive tenders. The resulting program, the Renewable Energy Independent Power Producer Procurement Program (REIPPPP) is an extensive initiative to install 17.8~GW of electricity generation capacity from renewable energy sources, such as wind, solar, biomass, biogas and hydropower, over the period 2012–2030. \cite{DEA2015,DoE2013,Eberhard2014}

The high dependency on and the scarcity of fossil resources, the intention to reduce greenhouse gas emissions and other environmental pollutants, rising fuel prices, and the predicted increase of energy demand are also forcing South Africa to promote investments in renewable energy sources. In 2009, the South African government began to implement feed-in tariffs (FITs) for renewable energy. These were later rejected in favor of competitive tenders. The resulting program, the Renewable Energy Independent Power Producer Procurement Program (REIPPPP) is an extensive initiative to install \SI{17.8}{\giga\watt} of electricity generation capacity from renewable energy sources such as wind, solar, biomass, biogas and hydropower, over the period 2012–2030 \cite{DEA2015,DoE2013,Eberhard2014}.

%SA has a high level of renewable energy potential, especially the solar irradiation is one of the highest in the world \cite{IRENA2014}. In some parts of the country the direct normal irradiance (DNI),  the direct beam radiation, rises above 3~000 kWh/m\textsuperscript{2} per year. In comparison to that, Austria has ca. one third of DNI per year \cite{SolarGIS2013a,SolarGIS2013}. Also the global horizontal irradiance (GHI), so the direct and diffuse solar radiation, is with about 2~300~kWh/m\textsuperscript{2} per year comparatively high \cite{SolarGIS2011}. The high solar irradiation allows SA a very effective electricity generation from an infinite source of energy.

South Africa has great renewable energy potential. The solar irradiation on its territory is among the highest in the world \cite{IRENA2014}. In some parts of the country, direct normal irradiance (DNI), the direct beam radiation, rises above \SI{3000}{\kilo\watt\hour\per\square\metre\per\year}. In comparison, Austria has one third the DNI per year \cite{SolarGIS2013a,SolarGIS2013}. The global horizontal irradiance (GHI), the combined direct and diffuse solar radiation, is also comparitively high at \SI{2300}{\kilo\watt\hour\per\square\metre\per\year} \cite{SolarGIS2011}. This high solar irradiation offers South Africa the promise of meeting all its energy demand from an infinite source of energy.

%Solar energy cannot be used as such, it has to be captured and converted into higher forms of energy, primarily heat and electricity. The conversion from solar energy into electrical energy is carried out by two mechanisms: the photovoltaic conversion and the thermal conversion. For the generation of solar thermal electricity (STE) it is common to concentrate the solar radiation with reflectors onto a receiver where a heat-transfer fluid (HTF) circulates. The fluid is heated up to high temperatures to be used in a thermodynamic cycle to generate electrical power. This technology is called concentrating solar power (CSP). The direct conversion from solar radiation into electricity is based on the photovoltaic effect using photovoltaic (PV) panels. The main difference between those two solar conversion technologies is the type of radiation that can be converted. The CSP technology can only capture and convert the  component of the solar radiation that is directly hitting the collector i.e. the DNI. The PV technology, instead, can furthermore convert the scattered component by clouds, water vapor and particles in the atmosphere and the reflected component due to the albedo effect. Although the CSP systems can exploit only the direct fraction of the overall solar irradiation, CSP plants allow to store the thermal energy with lower costs and lower environmental impacts than storing electric energy generated by PV systems. \cite{IEA2014e,EASAC2011} The levelized cost of electricity (LCOE) of STE and PV varies widely with the location, technology, design and intended use of plants. According to the U.S. Energy Information Administration the estimated total system LCOE for new generation resources in 2019 for PV is at about 130.0~US\$/MWh and for STE at about 243.1~US\$/MWh \cite{Outlook2014}. 

Solar radiation may not be used directly, it must first be captured and converted into other forms of energy, primarily heat and electricity. The conversion from solar radiation into electrical energy is carried out by two mechanisms: photovoltaic conversion and thermal conversion. For solar thermal electricity (STE) generation, it is common to concentrate the solar radiation with reflectors onto a receiver where a heat-transfer fluid (HTF) circulates. The fluid is heated up to high temperatures to be used in a thermodynamic cycle to generate electrical power. This technology is called concentrating solar power (CSP). The direct conversion from solar radiation into electricity is based on the photovoltaic effect using photovoltaic (PV) panels. The main difference between these two solar conversion technologies is the type of radiation that can be converted. The CSP technology can only capture and convert the component of the solar radiation that directly strikes the collector \textit{i.e.} the DNI. Photovoltaic technology, on the other hand, can convert the component scattered by clouds, water vapor and particles in the atmosphere and the reflected component due to the albedo effect. Although CSP systems can exploit only the direct fraction of the total solar irradiation, CSP plants also allow storage of thermal energy at lower costs and with lower environmental impacts than storing electric energy generated by PV systems \cite{IEA2014e,EASAC2011}. The levelized cost of electricity (LCOE) of STE and PV varies widely with the location, technology, design and intended use of plants. According to estimates of the U.S. Energy Information Administration, the total system LCOE for new generation resources in 2019 for PV will be \SI{130.0}{\usd/\mega\watt\hour} and \SI{243.1}{\usd/\mega\watt\hour} for STE \cite{Outlook2014}. 

%At first it can be summarized that STE technology is more expansive than PV technology, but it allows the use of the cheaper thermal storage technology. The stressed electricity grid in SA needs a constant and controllable energy supply. For a fluctuating energy source a balance mechanism is necessary to guarantee a support and not a pressure for the electricity supply. 

Though STE technology is more expensive than PV technology, it allows the use of the cheaper thermal storage technology. The stressed electricity grid in South Africa needs a constant and controllable energy supply. For a fluctuating energy source, a balance mechanism is necessary to guarantee support and not put the electricity supply under additional pressure.

% Was enthält diese Masterarbeit (Aufbau)
%\section{Description of the objective of the thesis}
\section{Objective of this thesis}

%The objective of this thesis was an technical and economical comparison between an concentrating solar and an photovoltaic power plant. This comparison is limited to the region South Africa and including as well the most reasonable storage systems for both technologies. 
%

The objective of this thesis is the technical and economic comparison of concentrating solar and photovoltaic power plants. This comparison is limited to South Africa and the most reasonable storage systems for both technologies.

%Also the medium- and long-term cost digression potential technologies are considered. This is an necessary indicator for common power plants investments in South Africa.

Also the medium- and long-term cost development of the technologies is considered. This is an important factor in common power plants investments in South Africa.

The research question is as follows:
\begin{quote}
Which technology, concentrating solar or photovoltaic power plants, in combination with suitable storage systems, will prevail in South Africa in a technical and economical comparison? What is the forecasted cost development of these technologies in the medium- and long-term?
\end{quote}

%Specified generation/load profile 
% yo yo my homey where is the rest?