\chapter{Introduction}
%
It is generally known that the worldwide energy demand is nearly entirely covered by fossil energy sources such as coal, oil, natural gas and uranium. Also, the consequences of fossil fuels to the environmental, social and economic development are globally well known and discussed. Especially the greenhouse gas emissions need to be reduced for the purpose to dilute their effects on climate change. Furthermore, the world population and the energy demand is rising. By 2035 the world population is projected to reach 8.7 billion, this means an additional 1.6 billion people who will need energy. This leads to a rise in primary energy consumption by 37~\% between 2013 and 2035 \cite{BP2015a}. 



Especially newly industrialized countries with a rapid population growth like China and India showed a strong rising primary energy consumption. Also the prognosticated primary energy demand in the newly industrialized country South Africa (SA), officially the Republic of South Africa, is growing from 1~639.83~TWh in 2012 to 2~163.18~TWh (including solid biomass and waist) in 2040. Coal is the mainstay of the South African energy system, meeting around 70~\% of primary energy demand and accounting for more than 90~\% of domestic electricity output. The prognosticated electricity demand is also rising by more than 70~\% from 212~TWh in 2012 to 364~TWh in 2040 \cite{IEA2014f}.



Depending on the growing electricity demand and missing investments in the power supply system in the last decades, the reserve margin in electricity capacity declined from almost 40~\% in 1990 to 8~\% at these days \cite{Trollip2014,Eskom2015}. The region of Western Cape experienced serious blackouts in late 2005 dependence on combination of inadequate reserve margin, insufficient reliability, a stressed system and system element failures. Those were the causes which forced Eskom (Eskom Holdings SOC Ltd.), the South African public utility, to start scheduled load shedding during peak hours, from 2007 on \cite{Trollip2014}. The economic and social impacts are clearly noticeable already. Load shedding was also the main trigger for Bureau for Economic Research (BER) senior economist Hugo Pienaar to revise down the 2015 GDP growth forecast from 2.9~\% to 1.9~\% for SA. \cite{Bisseker2015}.



The high dependency and the scarcity of fossil resources, the intention to reduce greenhouse gas emissions and other environmental pollutants, the rising fuel price, the predicted increase of energy demand, are also forcing SA to promote the investments in renewable energy sources. In 2009, the South African government began to implement feed-in tariffs (FITs) for renewable energies. These were later rejected in favor of competitive tenders. The resulting program, the Renewable Energy Independent Power Producer Procurement Program (REIPPPP) is an extensive initiative to install 17.8~GW of electricity generation capacity from renewable energy sources, such as wind, solar, biomass, biogas and hydropower, over the period 2012–2030. \cite{DEA2015,DoE2013,Eberhard2014}



SA has a high level of renewable energy potential, especially the solar irradiation is one of the highest in the world \cite{IRENA2014}. In some parts of the country the direct normal irradiance (DNI),  the direct beam radiation, rises above 3~000 kWh/m\textsuperscript{2} per year. In comparison to that, Austria has ca. one third of DNI per year \cite{SolarGIS2013a,SolarGIS2013}. Also the global horizontal irradiance (GHI), so the direct and diffuse solar radiation, is with about 2~300~kWh/m\textsuperscript{2} per year comparatively high \cite{SolarGIS2011}. The high solar irradiation allows SA a very effective electricity generation from an infinite source of energy.



Solar energy cannot be used as such, it has to be captured and converted into higher forms of energy, primarily heat and electricity. The conversion from solar energy into electrical energy is carried out by two mechanisms: the photovoltaic conversion and the thermal conversion. For the generation of solar thermal electricity (STE) it is common to concentrate the solar radiation with reflectors onto a receiver where a heat-transfer fluid (HTF) circulates. The fluid is heated up to high temperatures to be used in a thermodynamic cycle to generate electrical power. This technology is called concentrating solar power (CSP). The direct conversion from solar radiation into electricity is based on the photovoltaic effect using photovoltaic (PV) panels. The main difference between those two solar conversion technologies is the type of radiation that can be converted. The CSP technology can only capture and convert the  component of the solar radiation that is directly hitting the collector i.e. the DNI. The PV technology, instead, can furthermore convert the scattered component by clouds, water vapor and particles in the atmosphere and the reflected component due to the albedo effect. Although the CSP systems can exploit only the direct fraction of the overall solar irradiation, CSP plants allow to store the thermal energy with lower costs and lower environmental impacts than storing electric energy generated by PV systems. \cite{IEA2014e,EASAC2011} The levelized cost of electricity (LCOE) of STE and PV varies widely with the location, technology, design and intended use of plants. According to the U.S. Energy Information Administration the estimated total system LCOE for new generation resources in 2019 for PV is at about 130.0~US\$/MWh and for STE at about 243.1~US\$/MWh \cite{Outlook2014}. 



At first it can be summarized that STE technology is more expansive than PV technology, but it allows the use of the cheaper thermal storage technology. The stressed electricity grid in SA needs a constant and controllable energy supply. For a fluctuating energy source a balance mechanism is necessary to guarantee a support and not a pressure for the electricity supply. 

% Was enthält diese Masterarbeit (Aufbau)
\section{Description of the objective of the thesis}
The objective of this thesis was an technical and economical comparison between an concentrating solar and an photovoltaic power plant. This comparison is limited to the region South Africa and including as well the most reasonable storage systems for both technologies. 



Also the medium- and long-term cost digression potential technologies are considered. This is an necessary indicator for common power plants investments in South Africa.



The research question is as follows:
\begin{quote}
Which technology, concentrating solar or photovoltaic power plants in combination with suitable storage systems, will prevail in South Africa in a technical and economical comparison? What are the medium- and long-term cost degression potentials of these technologies?
\end{quote}


Specified generation/load profile 


\section{Relevance of results}
Was sagt mein Ergebniss aus und was kann man damit anfangen
\section{Methodology}

\subsection{Information procurement}
used SI-Units  = always 
\subsection{Quality assurance}

\subsection{Implementation of present resources}

\pagebreak

