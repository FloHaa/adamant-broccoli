\chapter{Introduction}
... % Hier muss noch was hin.

\section{Background and motivation}
South Africa's energy demand is constantly strong growing while the electricity generation is mainly based on coal, there major resource. Due to growing electricity demand and insufficient investment in the power system in the last decades, led to serious blackouts in late 2005. This forced Eskom Holdings SOC Ltd. (Eskom), the South African public utility, to implement scheduled load shedding during peak hours from 2008 on. The South African electricity supply needs new supporting generation capacities during the day and especially in the evening hours. 

The high dependency on and the scarcity of fossil resources, the intention to reduce greenhouse gas emissions and other environmental pollutants, rising fuel prices, and the predicted increase of energy demand are also forcing South Africa to promote investments in renewable energy sources. A key section is thereby the high solar radiation in South Africa.

Solar radiation may not be used directly, it must first be captured and converted into other forms of energy, primarily heat and electricity. The conversion from solar radiation into electrical energy is carried out by two mechanisms: photovoltaic conversion and thermal conversion. For solar thermal electricity (STE) generation, it is common to concentrate the solar radiation with reflectors onto a receiver where a heat-transfer fluid (HTF) circulates. The fluid is heated up to high temperatures to be used in a thermodynamic cycle to generate electrical power. This technology is called concentrating solar power (CSP). The direct conversion from solar radiation into electricity is based on the photovoltaic effect using photovoltaic (PV) panels. 

Both technologies can just collect solar radiation during the daytime, while the South African power supply needs support over the whole day and especially in the evening hours without solar radiation. In order to support the supply solar power plants needs assistance by energy storage. 
\section{Objective of this thesis}
The objective of this thesis is the technical and economic comparison of concentrating solar and photovoltaic power plants and the most reasonable storage systems for both technologies. This comparison is limited to South Africa and there specified needs in power generation.

Also the cost development for the next decades of the technologies is considered to give a indicator for common power plants investments in South Africa.

The research question is as follows:
\begin{quote}
Which technology, concentrating solar or photovoltaic power plants, in combination with suitable storage systems, will prevail in South Africa in a technical and economical comparison? What is the forecasted cost development of these technologies in the medium- and long-term?
\end{quote}