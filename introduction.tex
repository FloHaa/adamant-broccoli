\chapter{Introduction}
... % Hier muss noch was hin.

\section{Background and motivation}
%South Africa's energy demand is constantly strong growing while the electricity generation is mainly based on coal, there major resource. Due to growing electricity demand and insufficient investment in the power system in the last decades, led to serious blackouts in late 2005. This forced Eskom Holdings SOC Ltd. (Eskom), the South African public utility, to implement scheduled load shedding during peak hours from 2008 on. The South African electricity supply needs new supporting generation capacities during the day and especially in the evening hours. 

South Africa's energy demand is growing, while electricity generation is mainly based on coal, a major local resource. That growing demand, coupled with insufficient investment in the electrical system in past decades, has led to serious blackouts, beginning in 2005. This forced Eskom Holdings SOC Ltd. (Eskom), the South African public utility, to implement scheduled load shedding during peak hours from 2008 on. The South African electricity supply needs new generation capacity, especially in the evening hours. 

%The high dependency on and the scarcity of fossil resources, the intention to reduce greenhouse gas emissions and other environmental pollutants, rising fuel prices, and the predicted increase of energy demand are also forcing South Africa to promote investments in renewable energy sources. A key section is thereby the high solar radiation in South Africa.

The high dependency on and the scarcity of fossil resources, the intention to reduce greenhouse gas emissions and other environmental pollutants, rising fuel prices, and the predicted increase of energy demand are also forcing South Africa to promote investments in renewable energy sources, among them, solar energy.

Solar radiation may not be used directly. It must first be captured and converted into other forms of energy, primarily heat and electricity. The conversion from solar radiation into electrical energy is by two mechanisms: photovoltaic conversion and thermal conversion. For solar thermal electricity (STE) generation, it is common to concentrate the solar radiation with reflectors onto a receiver where a heat-transfer fluid (HTF) circulates. The fluid is heated up to high temperatures to be used in a thermodynamic cycle to generate electrical power. This technology is called concentrating solar power (CSP). The direct conversion from solar radiation into electricity is based on the photovoltaic effect, using photovoltaic (PV) panels. 

Solar radiation is present only during the daytime. South Africa needs energy over the whole day and especially in the evening hours, when solar radiation is absent. In order to maintain supply, solar plants need energy storage. 

%For CSP power plants storage application are state of the art and can nowerdays assures a constant power generation 24/7. For PV application there are currently large-scale hybrid systems with fossil backup on the marked \cite{BELECTRIC2015} to assure uninterrupted power supply of a PV based system, but pure large-scale PV power plants in connection with energy storage are currently unusual and was before not profound developed. 

For CSP power plants, modern storage systems can assure availability \enquote{24/7}. For photovoltaics, there are currently large-scale hybrid systems with fossil-fueled backup on the market \cite{BELECTRIC2015} to assure uninterrupted supply, but pure large-scale PV power plants in connection with energy storage remain unusual. 

%It can be expected that plain PV systems are more favourable then CSP systems. But in combination with a energy storage, which is necessary for long duration generation, this situation might be different because of more convenient storage applications of CSP.

Photovoltaic systems are technologically simpler than concentrating solar power systems. When combined with energy storage, however, the choice is not so clear; CSP systems have storage options which can be more attractive, both from an operational and financial perspective.

%This thesis provides therefore a comparison of large-scale solar power plants to support the South African power supply constantly and enhanced on high demand times.

\section{Objective of this thesis}
%The objective of this thesis is the technical and economic comparison of concentrating solar and photovoltaic power plants and the most reasonable storage systems for both technologies. This comparison is limited to South Africa and there specified needs in power generation.

The objective of this thesis is the technical and economic comparison of concentrating solar and photovoltaic power plants, and the evaluation of storage systems for both technologies. This comparison is limited to South Africa under consideration of its specific circumstances and requirements.

A cost prognosis is also considered, in order to provide infrastructure planning guidance.

The research question is as follows:
\begin{quote}
%Which technology, concentrating solar or photovoltaic power plants, in combination with suitable storage systems, will prevail in South Africa in a technical and economical comparison? What is the forecasted cost development of these technologies in the medium- and long-term?
Which technology, concentrating solar or photovoltaic power plants, in combination with suitable storage systems, will prevail in South Africa in a technical and economical comparison? What is the predicted cost development of these technologies in the medium- and long-term?
\end{quote}