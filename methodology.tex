\chapter{Methodology}
In order to assure a clearly and comprehensible approach this thesis follows a bottom-up design. Therefore all reasonable information are to find at the beginning of the study. The general approach and the information procurement are described in the following.
\section{Approach}
The first step of this work is to describe the power supply situation in South Africa from there demand over the generation and distribution to there major supply problems. Thereby the reader gets a impression of the need in South Africa for constant, but flexible power generation.

The second step is to describe the special solar irradiation situation in South Africa and the located potential for generating power from solar power plants. This will also reveal the downside of  direct solar generation without provide energy in storage for the South African electricity supply structure and thereby the need for storage application of solar power plants. A generation profile for the solar power plants will be defined to support the South African power supply at any time of the day but with a main focus on times with high demand. 

The next step is to describe the currently commercial and mature situation of solar power generation and there individual suitable storage technology. Thereby the exact technology type for the comparison will be selected. In order to compare the solar power plants equally they will be simulated under same conditions and requirements (location, weather data and generation profile) by using the System Advisor Model (SAM) software from the National Renewable Energy Laboratory (NREL) \cite{NREL2015}. SAM is designed to model and simulate performance and financial parameter of different types of renewable energy. It also can model PV and CSP application with belonging storage applications. The solar power plant will be modeled and simulated individually and compared in the following. The main performance and financial metrics for the comparison will be the amount of covering of the generation profile and the resulting LCOE.

At the end the future cost development of the solar power plants will be discussed and compared with new-built conventional power plant costs in South Africa.
\section{Information procurement}
In order to assure a academic work with high quality standard, all of the information and data used in this study were gathered from scientific publications or databases, renowned institutions/departments and libraries.

The following sources were considered to obtain scientific provable information:
\begin{itemize}
\item Solar Thermal Energy Research Group (STERG)
\item Centre for Renewable and Sustainable Energy Studies (CRSES)
\item Library of Stellenbosch University
\item Library of FH Technikum Wien (University of Applied Science Vienna)
\item Library of TU Munich
\item Library of Science Direct
\end{itemize}
Further information was mainly obtained from Eskom, the Department of Energy (DoE), NREL, BP and the International Energy Agency (IEA).

To assure a high quality of this academic work, all units are described in SI units system and applied after the current state of art \cite{Blankenburg2011}.


