\chapter{Solar energy in South Africa and load curve covering concept}\label{Solar power in South Africa}
Almost all power that we use on our planet comes from the sun. Direct in form of radiation or indirect during wind, water and vegetation. Also the fossil power resources and reserves are stored energy from the sun in from of organic carbon compounds. 

The sun emits a power rate of about 3.83x10\textsuperscript{26}W. Of this total, only a tiny fraction, \SI{1367}{\watt\per\square\metre} (solar constant) reaches the Earth’s atmosphere. The solar radiation is reduced by absorption and reflection effects in the atmosphere.  The reduction is about \SI{30}{\percent} on a clear day and about \SI{90}{\percent} on a very cloudy day. \cite{Stine2001a}

When taking an eye on the world map in Figure~\ref{WorldDNI} it can be noticed that some parts of the world receive much higher direct parts of the sun’s irradiation than others. In particular four regions worldwide are worth mentioning. The Atacama Desert in South America, the Mojave Desert in North America, a huge part of Australia and parts of the southern Africa. Therefore SA is one of the country with the highest potential for generating solar electricity in the world.

\begin{figure}[h!] 
\centering
\includegraphics[width=1\linewidth]{FIG/WorldDNI}
\caption[World map of Direct normal irradiation.]{World map of Direct normal irradiation \cite{SolarGIS2015c}.}\label{WorldDNI}
\end{figure}  
\section{Solar irradiation in South Africa}
As shown above, solar irradiation is highly depending from the location. The solar irradiance of a specific location can be measured on-site by ground measurement devices or site-adapted by interpolated satellite data, which is validated with other ground measurement devices. Thereby is the direct and indirect as well as the total sun irradiance crucial. These solar irradiation parameter are here defined:
\begin{itemize}
\item \textbf{Global Horizontal Irradiance (GHI)} in \si{\watt\hour\per\square\metre\year} or \si{\watt\per\square\metre}: GHI is the total amount of shortwave radiation received from above by a horizontal surface. It includes direct (beam) and a diffuse (scattered) irradiation. This value is of particular interest to PV or solar water heater with a fixed inclined angle.
\item \textbf{Direct Normal Irradiance (DNI)} in \si{\watt\hour\per\square\metre\year} or \si{\watt\per\square\metre}: DNI is the amount of solar radiation received per unit area by a surface that is always held perpendicular (or normal) to the rays that come in a straight line from the direction of the sun at its current position in the sky. Diffuse irradiation is totally excluded from the DNI. This quantity is of particular interest to  installations that track the position of the sun.
\item \textbf{Diffuse Horizontal Irradiance (DHI)} in \si{\watt\hour\per\square\metre\year} or \si{\watt\per\square\metre}: DHI is the amount of radiation received per unit area by a surface that does not arrive on a direct path from the sun, but has been scattered by molecules and particles in the atmosphere and comes equally from all directions.
\end{itemize}
Furthermore is irradiance understood as instantaneous density of solar radiation incident on a given surface, typically expressed in \si{\watt\per\square\metre} and irradiation is the sum of irradiance over a time period expressed in \si{\joule\per\square\metre} or more commonly used in \si{\watt\hour\per\square\metre}. The connection between the solar radiation parameters is shown in Equation \ref{GL_GHI}. The angle $\theta_\text{z}$ is the angle between the direction of the sun and the zenith (directly overhead).
\begin{align}
GHI=DNI*\cos(\theta_{z})+DHI \label{GL_GHI}
\end{align}
The GHI is decisive for the power output of PV systems and the DNI for CSP systems. Figure\ref{irradiation} shows the solar GHI and the DNI data for SA. It is shown, that the ceiling value for GHI can be more than \SI{2300}{\kilo\watt\hour\per\square\metre\year}, whereas in some parts of the country the DNI  value attains about \SI{3200}{\kilo\watt\hour\per\square\metre\year}. This is significantly high than in the most regions worldwide, therefor SA is predestined for using solar technologies. The figure shows, that the southeastern coastline has predominantly the lowest irradiance values. The solar irradiation rise significant in the inland. The highest GHI can be find close to the Namibian boarder in the northeast of the country. The direct beam is also at highest in the western part of SA. The area around Springbok in the province Northern Cape has the highest DNI value of the country.

\begin{figure}[h!]
        \centering
        \begin{subfigure}[b]{0.5\textwidth}
                \centering
                \includegraphics[width=1\textwidth]{FIG/SA_GHI}
                \caption{Global Horizontal Irradiation \cite{SolarGIS2015a}.}\label{fig:bild-links}
        \end{subfigure}%
        ~
        \begin{subfigure}[b]{0.5\textwidth}
                \centering
                \includegraphics[width=1\textwidth]{FIG/SA_DNI}
                \caption{Direct Normal Irradiation \cite{SolarGIS2015b}.}\label{fig:bild-rechts}
        \end{subfigure}
        \caption{Solar radiation maps of South Africa.}\label{irradiation}
\end{figure}
Both maps demonstrates, that the highest values of solar irradiation can be found in the northwestern part of SA, which allocated in the Northern Cape Province. Currently all CSP plants and about two-thirds of the PV systems of SA are developed in the Northern Cape \cite{Forder2015}. Thereby the region around the city of Upington is highly attractive, owning to the high irradiation value in connection with a reliable water access due to the Orange River and the possibility of a close access to the Eskom grid. 

Therefore the location parameter and weather data of Upington was selected for the simulation and calculations in this thesis \cite{WhiteBoxTechnologies2015}. The hourly values of GHI and DNI over a full year for Upington are shown in Figure~\ref{Upington_GHI/DNI}. The highest irradiance value during the summer are \SI{1199}{\watt\hour\per\square\metre} for GHI and \SI{1154}{\watt\hour\per\square\metre} for DNI. At the northern solstice, the highest irradiance is \SI{625}{\watt\hour\per\square\metre} for GHI and \SI{820}{\watt\hour\per\square\metre} for DNI. 

So it is obviously that the irradiation values demonstrates significant seasonal variation of GHI with high values in summer and low irradiation in winter, whereas the DNI shows a more balanced variation throughout the year. This mainly leads from the irradiation angle which is changing constantly during the year. 

\begin{figure}[!htbp]
        \centering
                \begin{subfigure}[b]{1\textwidth}
                \centering
                \includegraphics[width=1\textwidth]{FIG/Upington_GHI}
                \caption{Global horizontal irradiance}\label{Upington_GHI}
        \end{subfigure}%
\par\medskip % Linebreak      
        \begin{subfigure}[b]{1\textwidth}
                \centering
                \includegraphics[width=1\textwidth]{FIG/Upington_DNI}
                \caption{Direct normal irradiance}\label{Upington_DNI}
        \end{subfigure}%

        \caption[Hourly values of irradiance over a full year from Upington used for the simulation.]{Hourly values of irradiance over a full year from Upington used for the simulation.}\label{Upington_GHI/DNI}
\end{figure}
The path of the sun during the year in Upington is characterize in Figure~\ref{SunPathUpington}. The solar path diagram depends on the geographical location by the position of longitude and latitude. The diagram apparent that the longest day in Upington has a duration of \SI{13}{h} and \SI{56}{minutes} with a maximum sun height of \SI{85.05}{\degree} while the shortest day has a duration of just \SI{10}{h} and \SI{19}{minutes} and a maximum sun height of \SI{35.93}{\degree}.

\begin{figure}[htbp]  
\centering
\includegraphics[width=1\linewidth]{FIG/SunPathUpington}
\caption[Solar path diagram for Upington.]{Solar path diagram Upington \cite{PVsystSA2015}.}\label{SunPathUpington}
\end{figure}
\pagebreak
From the hourly irradiation values concludes a annual sum of \SI{2280}{\kilo\watt\hour\per\square\metre\year} by GHI and an annuall DNI amount of \SI{2621}{\kilo\watt\hour\per\square\metre\year}. When comparing this for the simulation used data with the solar irradiation maps in Figure~\ref{irradiation} the annual sum of GHI is corresponding. But it must be noted that the annual sum of DNI from SolarGIS map \cite{SolarGIS2015b} is about \SI{200}{\kilo\watt\hour\per\square\metre\year} higher than the value which was used for the simulation in this thesis.

The weather data for the simulation are in EPW format (EnergyPlus Weather Data) and produced by White Box Technologies, Inc. \cite{WhiteBoxTechnologies2015}. The EPW files are data sets of hourly values of solar radiation and meteorological elements for a typical one-year period. These include air temperature (\si{\celsius}), dew point temperature (\si{\celsius}), relative humidity (\si{\percent}), atmospheric pressure(\si{\milli\bar}), global horizontal solar radiation (\si{\watt\per\square\metre}), diffuse horizontal solar radiation (\si{\watt\per\square\metre}), direct normal radiation (\si{\watt\per\square\metre}), wind speed (\si{\metre\per\second}), wind direction (\si{\degree}) and snow depth (\si{\metre}). The most relevant parameters for the simulation are summarized in Table \ref{tbl: Location}. 
 
\begin{table}[!h]  
  \centering
	\begin{tabular}{  p{4.0cm}  C{4.0cm}  C{3.0cm} } 
	\hline	
\textbf{Item}  & \textbf{Value} & \textbf{Unit} \\ \hline \hline
Location & Upington & -\\ 
Station ID &  684240& -  \\ 
Data source & White Box Technologies, Inc. (31.05.2015) & -\\ \hline
Latitude & -28.40 &$\,^{\circ}$N \\ 
Longitude &  21.27 &$\,^{\circ}$E \\ 
Elevation &  836 & m \\ 
Total GHI per year  &  2~280 & \si{\kilo\watt\hour\per\square\metre}\\ 
Total DNI per year &  2~621 & \si{\kilo\watt\hour\per\square\metre}\\ 
Total DHI per year &  516 & \si{\kilo\watt\hour\per\square\metre}\\ 
Mean temp. &  21 & \si{\celsius}\\ 
Mean wind speed & 3.3 & \si{\metre\per\second}\\ \hline
\end{tabular}
\caption[Location and characteristics for the simulation in SAM.]{Location and characteristics for the simulation in SAM.}\label{tbl: Location}
\end{table}
\pagebreak
\section{System load in SA and predicted solar power generation profil} \label{SystemloadinSA}
%Power plants are forced to supply the system load/demand.
A hard requirement is that plants meet the full electrical demand in the system. Figure~\ref{LoadScenarios} shows the daily average system load/demand in South Africa for the winter and summer period.
% The profiles from both load shapes rises at approx 7:00 in the morning and has there peak demand at 20:00 during the summer period.
Both profiles feature a sharp rise at approx 7:00 and have their peak demand at 20:00 during the summer period.
In winter, there is an initial peak at 9:00 and a second at 19:00.
%In order to supply this system load the simulated solar power plants are forced to generate full power output of \SI{100}{MW} from 7:00 to 22:00. When the system demand comes down during the night also the power plants reduce there output from 22:00 to 7:00 to \SI{50}{MW}. The scenario is called "night-reduction".
In order to match this system load, the simulated solar power plants must operate at full power output of \SI{100}{MW} from 7:00 to 22:00. When the system demand drops during the night, the plants reduce there output from 22:00 to 7:00 to \SI{50}{MW}. The scenario is called "night reduction".
\begin{figure}[htbp]  
\centering
\includegraphics[width=1\linewidth]{FIG/LoadScenarios}
\caption[South Africa daily average system load/demand for summer and winter days, with scheduled power production curve.]{South Africa daily average system load/demand for summer and winter days, with scheduled power production curve.}\label{LoadScenarios}
\end{figure}
%For the comparison the power plants there system design is forced to cover 90~\% of the scheduled electricity production over the first year.  Usually, in feed-in contracts the power output is fixed at specified values. The overproduction is not content of these contracts and is not remunerated. In order to generate exploitable and comparable results and also considering the common feed-in contracts for power plants, the hourly power production is cut to the planed values. So overproduction will not considered in the evaluation and analyses of the systems.
In this comparison, the system design must handle 90~\% of the scheduled electricity production over the first year. Usually, in feed-in contracts, the deliverable energy is fixed. Any overproduction is not covered by such contracts and is not remunerated. In order to generate exploitable and comparable results and considering standard feed-in contracts, the hourly power production is cut to the planned values and thus overproduction is not considered in this analysis.

%The results of the simulated scenarios and the financial analyses will be rated in two selected categories:
The results of the simulation and the financial analyses will be evaluated by the following quantitative measures:
\begin{itemize}
%\item \textbf{Load curve covering factor} [\%]: The Load curve covering factor (LCCF) describes quantitative how effective the power plant follows the required load curve of the scenarios.
\item \textbf{Load curve covering factor} [\%]: The load curve covering factor (LCCF) is an effectiveness measure describing how closely the plant follows the load curve.
%\item \textbf{Levelized cost of electricity} [\textcent /kWh]: The levelized cost of electricity (LCOE) represents the total project lifecycle costs. It is the present value of project costs expressed in cents per kilo-hour of electricity generated by the system over its life. \cite{NREL2015a}
\item \textbf{Levelized cost of electricity} [\textcent /kWh]: The levelized cost of electricity (LCOE) represents the total project life-cycle costs. It is the present value of project costs expressed in cents per kilowatt-hour of electricity generated by the system over its life. \cite{NREL2015a}
\end{itemize}
%There is a huge difference between the mentioned LCCF and the widespread capacity factor (CF). The CF is the ratio of the system's predicted electrical output in the first year of operation to the nameplate output, which is equivalent to the quantity of energy the system would generate if it operated at its nameplate capacity for every hour of the year \cite{NREL2015a}. As it is mentioned above is the LCCF calculated from the sum value of load covering in each hour of the year.
There is an important difference between the LCCF and the widely-used \emph{capacity factor} (CF). The CF is the ratio of the system's electrical output in the first year of operation to the nameplate output, which is equivalent to the quantity of energy the system would generate if it operated at its nameplate capacity for every hour of the year \cite{NREL2015a}. The LCCF is calculated from the sum value of load coverage in each hour of the year. At a covering of 100~\% of the predicted load the solar power plants would produce \SI{711.75}{GWh} per year. But at a 100~\% CF the solar power plants would need to produce \SI{876.00}{GWh} per year. Therefore is in the selected scenario a maximum CF of just 81.25~\% with a full covering of the predicted load possible.



\pagebreak 


