%\chapter{Solar radiation in South Africa and load curve covering concept}\label{Solar power in South Africa}
%South Africa is one of the country with the highest potential for generating solar thermal electricity in the word. Figure \ref{DHI-DIF} shows  the daily sum of global irradiation in Aberdeen, United Kingdom and Upington, South Africa. The yearly share of diffuse irradiation in Aberdeen overlaps 60 \%. In winter month the share reaches 90 \%. The share in Upington is approximately 25 \% and is in total significantly higher than Aberdeen. The direct comparison shows also the difference of the solar irradiation between northern and southern hemisphere during the year. It is obviously, that the seasons are the other way around between both hemispheres.
\chapter{Solar radiation in South Africa and load curve covering}\label{Solar power in South Africa}

The sun emits a power rate of about 3.83x10\textsuperscript{26}W. Of this total, only a tiny fraction, \SI{1367}{\watt\per\square\metre} (solar constant) reaches the Earth’s atmosphere. The solar radiation is reduced by absorption and reflection effects in the atmosphere.  The reduction is about \SI{30}{\percent} on a clear day and about \SI{90}{\percent} on a very cloudy day. \cite{Stine2001a}

When taking an eye on the world map in Figure~\ref{WorldDNI} it can be noticed that some parts of the world receive much higher direct parts of the sun’s irradiation than others. In particular four regions worldwide are worth mentioning. The Atacama Desert in South America, the Mojave Desert in North America, a huge part of Australia and parts of the southern Africa. Therefore SA is one of the country with the highest potential for generating solar electricity in the world.

\begin{figure}[h!] 
\centering
\includegraphics[width=1\linewidth]{FIG/WorldDNI}
\caption[World map of Direct normal irradiation.]{World map of Direct normal irradiation \cite{SolarGIS2015c}.}\label{WorldDNI}
\end{figure}  

South Africa is among the countries with the greatest potential for generating solar thermal electricity. The percentage of total radiation that is diffuse in Aberdeen, United Kingdom is \SI{60}{\percent}; in winter months it can be as high as \SI{90}{\percent} (Figure \ref{DHI-DIF}). In Upington, South Africa, the diffuse share is \SI{25}{\percent} and total radiation is significantly higher than in Aberdeen. The seasonal differences between northern and southern hemispheres are evident.

%This chapter shows energy impact from the sun and how it is distributed in SA. Furthermore the chapter comprised the current situation of solar power plants in SA.
\begin{figure}[h!] % DHI-DFI
\centering
\includegraphics[width=1\linewidth]{FIG/DHI-DIF}
\caption[Long-term monthly variability of direct (DNI) and diffuse (DHI) irradiaton throughout the year for Aberdeen, UK and Upington, RSA.]{Long-term monthly variability of direct (DNI) and diffuse (DHI) irradiaton throughout the year for Aberdeen, UK and Upington, RSA \cite{SolarGIS2015}.}\label{DHI-DIF}
\end{figure} 
\section{Solar radiation}

As shown above, solar irradiation is highly depending from the location. The solar irradiance of a specific location can be measured on-site by ground measurement devices or site-adapted by interpolated satellite data, which is validated with other ground measurement devices. Thereby is the direct and indirect as well as the total sun irradiance crucial.
%The most important solar radiation parameters for designing a solar power plant are here defined:
The solar radiation parameters most critical for power plant design are:
\begin{itemize}
%\item \textbf{GHI} (kWh/m\textsuperscript{2}/a or W/m\textsuperscript{2}): Global Horizontal Irradiance is the total amount of shortwave radiation received from above by a horizontal surface. It includes direct (beam) and a diffuse (scattered) irradiation. This value is of particular interest to PV or solar water heater with a fixed inclined angle.
\item \textbf{GHI} (\si{\kilo\watt\hour\per\squared\metre\year} or \si{\watt\per\squared\metre}): \emph{Global Horizontal Irradiance} is the total amount of shortwave radiation received from above by a horizontal surface. It includes direct (beam) and a diffuse (scattered) irradiation. This value is of particular interest when designing PV or solar water heater systems with a fixed inclined angle.
%\item \textbf{DNI} (kWh/m\textsuperscript{2}/a or W/m\textsuperscript{2}): Direct Normal Irradiance is the amount of solar radiation received per unit area by a surface that is always held perpendicular (or normal) to the rays that come in a straight line from the direction of the sun at its current position in the sky. Diffuse irradiation is totally excluded from the DNI. This quantity is of particular interest to  installations that track the position of the sun.
\item \textbf{DNI} (\si{\kilo\watt\hour\per\squared\metre\year} or \si{\watt\per\squared\metre}): \emph{Direct Normal Irradiance} is the amount of solar radiation received per unit area by a surface that is always held perpendicular (or normal) to the rays that come in a straight line from the direction of the sun at its current position in the sky. Diffuse irradiation is totally excluded from the DNI. This quantity is relevant for installations that track the position of the sun.
%\item \textbf{DHI} (kWh/m\textsuperscript{2}/a or W/m\textsuperscript{2}): Diffuse Horizontal Irradiance is the amount of radiation received per unit area by a surface that does not arrive on a direct path from the sun, but has been scattered by molecules and particles in the atmosphere and comes equally from all directions.
\item \textbf{DHI} (\si{\kilo\watt\hour\per\squared\metre\year} or \si{\watt\per\squared\metre}): \emph{Diffuse Horizontal Irradiance} is the amount of radiation received per unit area by a surface that does not arrive on a direct path from the sun, but has been scattered by molecules and particles in the atmosphere and comes equally from all directions.
\end{itemize}
%Furthermore is irradiance understood as instantaneous density of solar radiation incident on a given surface, typically expressed in W/m\textsuperscript{2} and irradiation is the sum of irradiance over a time period expressed in J/m\textsuperscript{2} or more commonly used in Wh/m\textsuperscript{2}. The connection between the solar radiation parameters is shown in Equation \ref{GL_GHI}.The angle $\theta_\text{z}$ is the angle between the direction of the sun and the zenith (directly overhead).
\emph{Irradiance} is understood as the instantaneous density of solar radiation incident on a given surface and is typically expressed in \si{\watt\per\squared\metre} and \emph{irradiation} is the sum of irradiance over a time period expressed in \si{\joule\per\squared\metre} or the more commonly used \si{\watt\hour\per\squared\metre}. The relationship between solar radiation parameters is shown by Equation \ref{GL_GHI}.The angle $\theta_\text{z}$ is the angle between the direction of the sun and the zenith (directly overhead).
\begin{align}
\text{GHI}=\text{DNI}\cdot\cos(\theta_{z})+\text{DHI}\label{GL_GHI}
\end{align}

%Figure\ref{irradiation} shows the solar GHI and the DNI data for the country. It is shown, that the ceiling value for GHI can be more than 2~300~kWh/m\textsuperscript{2}/a, whereas in some parts of the country the DNI  value attains about 2~900~kWh/m\textsuperscript{2}/a. This is significantly high than in the most regions worldwide, therefor SA is predestined for using solar technologies. The figure shows, that the southeastern coastline has predominantly the lowest irradiance values. The solar irradiation rise significant in the inland. The highest GHI can be find close to the Namibian boarder in the northeast of the country. The direct beam is also at highest in the western part of SA. The area around Springbok in the province Northern Cape has the highest DNI value of the country.
In South Africa, the ceiling for GHI is more than \SI{2300}{\kilo\watt\hour\per\squared\metre\year}, while in some parts of the country DNI is as high as \SI{2900}{\kilo\watt\hour\per\squared\metre\year} (Figure\ref{irradiation}), significantly higher than in most regions of the world. As such, solar power is South Africa is particularly well-suited to solar power. The highest GHI occurs close to the Namibian border in the northwest. The direct beam is also at its highest in western SA. The area around Springbok in Northern Cape province has the highest DNI.

\begin{figure}[h!]
        \centering
        \begin{subfigure}[b]{0.5\textwidth}
                \centering
                \includegraphics[width=1\textwidth]{FIG/SA_GHI}
                \caption{Global Horizontal Irradiation \cite{SolarGIS2015a}.}\label{fig:bild-links}
        \end{subfigure}%
        ~
        \begin{subfigure}[b]{0.5\textwidth}
                \centering
                \includegraphics[width=1\textwidth]{FIG/SA_DNI}
                \caption{Direct Normal Irradiation \cite{SolarGIS2015b}.}\label{fig:bild-rechts}
        \end{subfigure}
        \caption{Solar radiation maps of South Africa.}\label{irradiation}
\end{figure}
Both maps demonstrates, that the highest values of solar irradiation can be found in the northwestern part of SA, which allocated in the Northern Cape Province. Currently all CSP plants and about two-thirds of the PV systems of SA are developed in the Northern Cape \cite{Forder2015}. Thereby the region around the city of Upington is highly attractive, owning to the high irradiation value in connection with a reliable water access due to the Orange River and the possibility of a close access to the Eskom grid. 

Therefore the location parameter and weather data of Upington was selected for the simulation and calculations in this thesis \cite{WhiteBoxTechnologies2015}. The hourly values of GHI and DNI over a full year for Upington are shown in Figure~\ref{Upington_GHI/DNI}. The highest irradiance value during the summer are \SI{1199}{\watt\hour\per\square\metre} for GHI and \SI{1154}{\watt\hour\per\square\metre} for DNI. At the northern solstice, the highest irradiance is \SI{625}{\watt\hour\per\square\metre} for GHI and \SI{820}{\watt\hour\per\square\metre} for DNI. 

So it is obviously that the irradiation values demonstrates significant seasonal variation of GHI with high values in summer and low irradiation in winter, whereas the DNI shows a more balanced variation throughout the year. This mainly leads from the irradiation angle which is changing constantly during the year. 

\begin{figure}[!htbp]
        \centering
                \begin{subfigure}[b]{1\textwidth}
                \centering
                \includegraphics[width=1\textwidth]{FIG/Upington_GHI}
                \caption{Global horizontal irradiance}\label{Upington_GHI}
        \end{subfigure}%
\par\medskip % Linebreak      
        \begin{subfigure}[b]{1\textwidth}
                \centering
                \includegraphics[width=1\textwidth]{FIG/Upington_DNI}
                \caption{Direct normal irradiance}\label{Upington_DNI}
        \end{subfigure}%

        \caption[Hourly values of irradiance over a full year from Upington used for the simulation.]{Hourly values of irradiance over a full year from Upington used for the simulation.}\label{Upington_GHI/DNI}
\end{figure}
The path of the sun during the year in Upington is characterize in Figure~\ref{SunPathUpington}. The solar path diagram depends on the geographical location by the position of longitude and latitude. The diagram apparent that the longest day in Upington has a duration of \SI{13}{h} and \SI{56}{minutes} with a maximum sun height of \SI{85.05}{\degree} while the shortest day has a duration of just \SI{10}{h} and \SI{19}{minutes} and a maximum sun height of \SI{35.93}{\degree}.

\begin{figure}[htbp]  
\centering
\includegraphics[width=1\linewidth]{FIG/SunPathUpington}
\caption[Solar path diagram for Upington.]{Solar path diagram Upington \cite{PVsystSA2015}.}\label{SunPathUpington}
\end{figure}
\pagebreak
From the hourly irradiation values concludes a annual sum of \SI{2280}{\kilo\watt\hour\per\square\metre\year} by GHI and an annuall DNI amount of \SI{2621}{\kilo\watt\hour\per\square\metre\year}. When comparing this for the simulation used data with the solar irradiation maps in Figure~\ref{irradiation} the annual sum of GHI is corresponding. But it must be noted that the annual sum of DNI from SolarGIS map \cite{SolarGIS2015b} is about \SI{200}{\kilo\watt\hour\per\square\metre\year} higher than the value which was used for the simulation in this thesis.

The weather data for the simulation are in EPW format (EnergyPlus Weather Data) and produced by White Box Technologies, Inc. \cite{WhiteBoxTechnologies2015}. The EPW files are data sets of hourly values of solar radiation and meteorological elements for a typical one-year period. These include air temperature (\si{\celsius}), dew point temperature (\si{\celsius}), relative humidity (\si{\percent}), atmospheric pressure(\si{\milli\bar}), global horizontal solar radiation (\si{\watt\per\square\metre}), diffuse horizontal solar radiation (\si{\watt\per\square\metre}), direct normal radiation (\si{\watt\per\square\metre}), wind speed (\si{\metre\per\second}), wind direction (\si{\degree}) and snow depth (\si{\metre}). The most relevant parameters for the simulation are summarized in Table \ref{tbl: Location}. 
 
\begin{table}[!h]  
  \centering
	\begin{tabular}{  p{4.0cm}  C{4.0cm}  C{3.0cm} } 
	\hline	
\textbf{Item}  & \textbf{Value} & \textbf{Unit} \\ \hline \hline
Location & Upington & -\\ 
Station ID &  684240& -  \\ 
Data source & White Box Technologies, Inc. (31.05.2015) & -\\ \hline
Latitude & -28.40 &$\,^{\circ}$N \\ 
Longitude &  21.27 &$\,^{\circ}$E \\ 
Elevation &  836 & m \\ 
Total GHI per year  &  2~280 & \si{\kilo\watt\hour\per\square\metre}\\ 
Total DNI per year &  2~621 & \si{\kilo\watt\hour\per\square\metre}\\ 
Total DHI per year &  516 & \si{\kilo\watt\hour\per\square\metre}\\ 
Mean temp. &  21 & \si{\celsius}\\ 
Mean wind speed & 3.3 & \si{\metre\per\second}\\ \hline
\end{tabular}
\caption[Location and characteristics for the simulation in SAM.]{Location and characteristics for the simulation in SAM.}\label{tbl: Location}
\end{table}
\pagebreak
\section{Current stage of solar power in South Africa}
South Africa started there expansion in the field of solar power plants with the first round of the REIPPPP in 2011. As it was mentioned before in Section~\ref{ElectricitySA} the REIPPPP has currently a capacity of \SI{5237}{\mega\watt} committed inclusive \SI{1899}{\mega\watt} PV systems and \SI{600}{\mega\watt} CSP plants. Additionally comes one CSP project with \SI{100}{\mega\watt} from Eskom which is not a part of the REIPPPP.

Figure \ref{Solar-map} shows the allocation of all currently committed solar power plants of the REIPPPP. Yellow marked are PV-power plants and CSP plants are marked in orange (some marks cover them mutually). The numbers in the single marks expose in which REIPPPP-Round it belongs.

Currently SA has 27 fully operational PV-power plants with a total capacity of \SI{1059.05}{\mega\watt} further six PV-power plants with in total of \SI{442.5}{\mega\watt} are under construction. \cite{Forder2015}

Most of the South African PV-power plants are allocated in the Northern Cape. So 29 out of 45 currently committed PV systems are located there. Five more each in the Western Cape and the North-West Province, three each in the provinces Free State and Limpopo, two in Eastern Cape and one more in the Eastern Cape. \cite{Forder2015}
\pagebreak
\section{Solar power plants in South Africa}

%South Africa started there expansion in the field of solar power plants in the first round of the Renewable Energy Independent Power Producer Procurement Program (REIPPPP) in 2011. Now in 2015 starts the fourth round of the REIPPPP. Figure \ref{Solar-map} shows the allocation of all solar power plants of the REIPPPP. PV-power plants are marked in yellow and CSP-power plants are marked in orange. The numbers in the single marks expose in which REIPPPP-Round it belongs.
South Africa's solar power expansion began with the first round of the Renewable Energy Independent Power Producer Procurement Program (REIPPPP) in 2011, and continues through to the fourth round of the REIPPPP in 2015. Figure \ref{Solar-map} shows the location of all REIPPPP solar power plants. Photovoltaic plants are marked in yellow and CSP plants are marked in orange. The numbers in the marks correspond to the applicable REIPPPP round.

\begin{figure}[h!] % Solar-map
\centering
\includegraphics[width=1\linewidth]{FIG/Solar-map}
\caption[REIPPPP solar power plants in South Africa.]{REIPPPP solar power plants in South Africa \cite{Forder2015}.}\label{Solar-map}
\end{figure}

South Africa has 26 fully operational PV plants with a total capacity of \SI{1048.7}{\mega\watt}, a further seven with \SI{135.35}{\mega\watt} are under construction, four more with \SI{307.5}{\mega\watt} waiting to begin construction (approved and financed) and six more in approvals, planning and financing with a total capacity of \SI{415}{\mega\watt}. Of the total, 26 of 39 of the operational and planned plants are located in Northern Cape, five in the Western Cape, three each in the Free State and Limpopo and one each in Eastern Cape and the North-West Province \cite{Forder2015}.

\pagebreak
\section{System load in SA and predicted solar power generation profile} \label{SystemloadinSA}
%Power plants are forced to supply the system load/demand.
A hard requirement is that plants meet the full electrical demand in the system. Figure~\ref{LoadScenarios} shows the daily average system load/demand in South Africa for the winter and summer period.
% The profiles from both load shapes rises at approx 7:00 in the morning and has there peak demand at 20:00 during the summer period.
Both profiles feature a sharp rise at approx 7:00 and have their peak demand at 20:00 during the summer period.
In winter, there is an initial peak at 9:00 and a second at 19:00.
%In order to supply this system load the simulated solar power plants are forced to generate full power output of \SI{100}{MW} from 7:00 to 22:00. When the system demand comes down during the night also the power plants reduce there output from 22:00 to 7:00 to \SI{50}{MW}. The scenario is called "night-reduction".
In order to match this system load, the simulated solar power plants must operate at full power output of \SI{100}{MW} from 7:00 to 22:00. When the system demand drops during the night, the plants reduce there output from 22:00 to 7:00 to \SI{50}{MW}. The scenario is called "night reduction".
\begin{figure}[htbp]  
\centering
\includegraphics[width=1\linewidth]{FIG/LoadScenarios}
\caption[South Africa daily average system load/demand for summer and winter days, with scheduled power production curve.]{South Africa daily average system load/demand for summer and winter days, with scheduled power production curve.}\label{LoadScenarios}
\end{figure}
%For the comparison the power plants there system design is forced to cover 90~\% of the scheduled electricity production over the first year.  Usually, in feed-in contracts the power output is fixed at specified values. The overproduction is not content of these contracts and is not remunerated. In order to generate exploitable and comparable results and also considering the common feed-in contracts for power plants, the hourly power production is cut to the planed values. So overproduction will not considered in the evaluation and analyses of the systems.
In this comparison, the system design must handle 90~\% of the scheduled electricity production over the first year. Usually, in feed-in contracts, the deliverable energy is fixed. Any overproduction is not covered by such contracts and is not remunerated. In order to generate exploitable and comparable results and considering standard feed-in contracts, the hourly power production is cut to the planned values and thus overproduction is not considered in this analysis.

%The results of the simulated scenarios and the financial analyses will be rated in two selected categories:
The results of the simulation and the financial analyses will be evaluated by the following quantitative measures:
\begin{itemize}
%\item \textbf{Load curve covering factor} [\%]: The Load curve covering factor (LCCF) describes quantitative how effective the power plant follows the required load curve of the scenarios.
\item \textbf{Load curve covering factor} [\%]: The load curve covering factor (LCCF) is an effectiveness measure describing how closely the plant follows the load curve.
%\item \textbf{Levelized cost of electricity} [\textcent /kWh]: The levelized cost of electricity (LCOE) represents the total project lifecycle costs. It is the present value of project costs expressed in cents per kilo-hour of electricity generated by the system over its life. \cite{NREL2015a}
\item \textbf{Levelized cost of electricity} [\textcent /kWh]: The levelized cost of electricity (LCOE) represents the total project life-cycle costs. It is the present value of project costs expressed in cents per kilowatt-hour of electricity generated by the system over its life. \cite{NREL2015a}
\end{itemize}
%There is a huge difference between the mentioned LCCF and the widespread capacity factor (CF). The CF is the ratio of the system's predicted electrical output in the first year of operation to the nameplate output, which is equivalent to the quantity of energy the system would generate if it operated at its nameplate capacity for every hour of the year \cite{NREL2015a}. As it is mentioned above is the LCCF calculated from the sum value of load covering in each hour of the year.
There is an important difference between the LCCF and the widely-used \emph{capacity factor} (CF). The CF is the ratio of the system's electrical output in the first year of operation to the nameplate output, which is equivalent to the quantity of energy the system would generate if it operated at its nameplate capacity for every hour of the year \cite{NREL2015a}. The LCCF is calculated from the sum value of load coverage in each hour of the year. At a covering of 100~\% of the predicted load the solar power plants would produce \SI{711.75}{GWh} per year. But at a 100~\% CF the solar power plants would need to produce \SI{876.00}{GWh} per year. Therefore is in the selected scenario a maximum CF of just 81.25~\% with a full covering of the predicted load possible.


%Mainly PV-power plants are allocated in the Northern Cape. So 26 of 39 of the operational and planed plants are located there. Five more in the Western Cape, three each in the provinces Free State and Limpopo and each one in Eastern Cape and the North-West Province. \cite{Forder2015}

%"KaXu Solar One" is the first fully operational CSP-plant in SA and is shown in Figure \ref{KaXu-solar-field}. It is using parabolic trough technology and a 2.5~h thermal energy storage for generating of 100~MW capacity electricity. The CSP-plants "Khi Solar One" and "Bokpoort CSP Project" with a capacity of each 50~MW are under construction. Further three CSP-plants are awaiting construction, they have all a capacity of each 100~MW. 
\emph{KaXu Solar One} is the first fully operational CSP plant in South Africa (Figure \ref{KaXu-solar-field}). It uses parabolic trough technology with \SI{2.5}{\hour} thermal energy storage for \SI{100}{\mega\watt} generating capacity. The CSP plants \emph{Khi Solar One} and \emph{Bokpoort CSP Project} with a capacity of \SI{50}{\mega\watt} each are now under construction. An additional three CSP plants, each with a generating capacity of \SI{100}{\mega\watt} are awaiting construction. 

%All six CSP-plants are located in the region around the cites to Upington or Pofadder in the Northern Cape. This region is predestined for CSP-plants, because it has high DNI (around~2~900~kWh/m\textsuperscript{2}/a) and a water connection through the Orange River. \cite{Forder2015}
All of the operating and planned CSP plants are located around Upington and Pofadder in the Northern Cape, a region particularly suited to CSP plants because of its high DNI (\SI{2900}{\kilo\watt\hour\per\squared\meter\year}) and its water access via the Orange River \cite{Forder2015}.

\begin{figure}[!h]
\centering
\includegraphics[width=1\linewidth]{FIG/KaXu-solar-field}
\caption[KaXu Solar One, a \SI{100\{\mega\watt} parabolic trough plant with 2.5 hours of thermal storage in molten salts.]{KaXu Solar One, a \SI{100\{\mega\watt} parabolic trough plant with 2.5 hours of thermal storage in molten salts \cite{AbengoaSolar2015}.}\label{KaXu-solar-field}
\end{figure}
\pagebreak 
