% FH Technikum Wien
% !TEX encoding = UTF-8 Unicode
%
% Erstellung von Master- und Bachelorarbeiten an der FH Technikum Wien mit Hilfe von LaTeX und der Klasse TWBOOK
%
% Um ein eigenes Dokument zu erstellen, müssen Sie folgendes ergänzen:
% 1) Mit \documentclass[..] einstellen: Master- oder Bachelorarbeit, Studiengang und Sprache
% 2) Mit \newcommand{\FHTWCitationType}.. Zitierstandard festlegen (wird in der Regel vom Studiengang vorgegeben - bitte erfragen)
% 3) Deckblatt, Kurzfassung, etc. ausfüllen
% 4) und die Arbeit schreiben (die verwendeten Literaturquellen in Literatur.bib eintragen)
%
% Getestet mit TeXstudio mit Zeichenkodierung ISO-8859-1 (=ansinew/latin1) und MikTex unter Windows
% Zu beachten ist, dass die Kodierung der Datei mit der Kodierung des paketes inputenc zusammen passt!
% Die Kodierung der Datei twbook.cls MUSS ANSI betragen!
% Bei der Verwendung von UTF8 muss dnicht nur die Kodierung des Dokuments auf UTF8 gestellt sein, sondern auch die des BibTex-Files!
%
% Bugreports und Feedback bitte per E-Mail an latex@technikum-wien.at
%
% Versionen
% *) V0.7: 9.1.2015, RO: Modeline angepasst und verschoben
% *) V0.6: 10.10.2014, RO: Weitere Anpassung an die UK
% *) V0.5: 8.8.2014, WK: Literaturquellen überarbeitet und angepasst
% *) V0.4: 4.8.2014, WK: Initalversion in SVN eingespielt
%
\documentclass[Master,MEE,english]{twbook}%\documentclass[Bachelor,BMR,german]{twbook}
%\usepackage[utf8]{inputenc}
%\usepackage[T1]{fontenc}
\usepackage{textcomp}
\usepackage{graphicx}
\usepackage{caption}
\usepackage{subcaption}
\usepackage{pgfplots}
\pgfplotsset{compat=1.10}
\usepackage{pgfplotstable}
\usepackage{siunitx}                       
\usepgfplotslibrary{units}
\usepackage{csquotes}
\usepackage{booktabs}
\usepackage{array}
\usepackage{rotating}
%
% Bitte in der folgenden Zeile den Zitierstandard festlegen
\newcommand{\FHTWCitationType}{HARVARD} % IEEE oder HARVARD möglich - wenn Sie zwischen IEEE und HARVARD wechseln, bitte die temorären Dateien (aux, bbl, ...) löschen
%
\ifthenelse{\equal{\FHTWCitationType}{HARVARD}}{\usepackage{harvard}}{\usepackage{bibgerm}}

% Definition Code-Listings Formatierung:
\usepackage[final]{listings}
\lstset{captionpos=b, numberbychapter=false,caption=\lstname,frame=single, numbers=left, stepnumber=1, numbersep=2pt, xleftmargin=15pt, framexleftmargin=15pt, numberstyle=\tiny, tabsize=3, columns=fixed, basicstyle={\fontfamily{pcr}\selectfont\footnotesize}, keywordstyle=\bfseries, commentstyle={\color[gray]{0.33}\itshape}, stringstyle=\color[gray]{0.25}, breaklines, breakatwhitespace, breakautoindent}
\lstloadlanguages{[ANSI]C, C++, [gnu]make, gnuplot, Matlab}

%Formatieren des Quellcodeverzeichnisses
\makeatletter
% Setzen der Bezeichnungen für das Quellcodeverzeichnis/Abkürzungsverzeichnis in Abhängigkeit von der eingestellten Sprache
\providecommand\listacroname{}
\@ifclasswith{twbook}{english}
{%
    \renewcommand\lstlistingname{Code}
    \renewcommand\lstlistlistingname{List of Code}
    \renewcommand\listacroname{List of Abbreviations}
}{%
    \renewcommand\lstlistingname{Quellcode}
    \renewcommand\lstlistlistingname{Quellcodeverzeichnis}
    \renewcommand\listacroname{Abkürzungsverzeichnis}
}
% Wenn die Option listof=entryprefix gewählt wurde, Definition des Entyprefixes für das Quellcodeverzeichnis. Definition des Macros listoflolentryname analog zu listoflofentryname und listoflotentryname der KOMA-Klasse
\@ifclasswith{scrbook}{listof=entryprefix}
{%
    \newcommand\listoflolentryname\lstlistingname
}{%
}
\makeatother
\newcommand{\listofcode}{\phantomsection\lstlistoflistings}

% Die nachfolgenden 2 Pakete stellen sonst nicht benötigte Features zur Verfügung
\usepackage{blindtext,dtklogos}

%
% Einträge für Deckblatt, Kurzfassung, etc.
%
\title{Technology comparison and profitability analysis of PV and CSP for large-scale electricity production in South Africa}
\author{Florian Haag, BEng}
\studentnumber{1310578031}
\supervisor{FH-Prof. Dipl.-Ing. Hubert Fechner, MSc, MAS}
\secondsupervisor{Prof. Dr.-Ing. Frank Dinter}
\place{Stellenbosch, South Africa}
\kurzfassung{Hier ist die Kurzfassung}
\schlagworte{Schlagwort1, Schlagwort2, Schlagwort3, Schlagwort4}
\outline{This is the Abstract BLABLA}
\keywords{Keyword1, Keyword2, Keyword3, Keyword4}
%\acknowledgements{"I'd put my money on the sun and solar energy. What a source of power! I hope we don't have to wait till oil and coal run out before we tackle that."
%\par
%- Thomas Edison (1931)}

% siunitx-Konfiguration
\sisetup{detect-all}
\DeclareSIQualifier\electric{el}
\DeclareSIQualifier\thermal{th}
\DeclareSIUnit\wattel{\watt\electric}
\DeclareSIUnit\wattth{\watt\thermal}
\DeclareSIUnit\usd{USD}
\DeclareSIUnit\eur{EUR}

%\usepackage{stdpage}
\KOMAoption{parskip}{half}
\begin{document}

\newcolumntype{C}[1]{>{\centering\arraybackslash}m{#1}} 

%Festlegungen für den HARVARD-Zitierstandard
\ifthenelse{\equal{\FHTWCitationType}{HARVARD}}{
\bibliographystyle{Harvard_FHTW_MR}%Zitierstandard FH Technikum Wien, Studiengang Mechatronik/Robotik, Version 1.2e
\citationstyle{dcu}%Correct citation-style (Harvardand, ";" between citations, "," between author and year)
\citationmode{abbr}%use "et al." with first citation
\iflanguage{ngerman}{
    %Deutsch Neue Rechtschreibung
    \newcommand{\citepic}[1]{(Quelle: \protect\cite{#1})}%Zitat: Bild
    \newcommand{\citefig}[2]{(Quelle: \protect\cite{#1}, S. #2)}%Zitat: Bild aus Dokument
    \newcommand{\citefigm}[2]{(Quelle: modifiziert "ubernommen aus \protect\cite{#1}, S. #2)}%Zitat: modifiziertes Bild aus Dokument
    \newcommand{\citep}{\citeasnoun}%In-Line Zitiat entweder mit \citep{} oder \citeasnoun{}
    \newcommand{\acessedthrough}{Verf{\"u}gbar unter:}%Für URL-Angabe
    \newcommand{\acessedthroughp}{Verf{\"u}gbar bei:}%Für URL-Angabe (Geschützte Datenbank, Zugriff durch FH)
    \newcommand{\acessedat}{Zugang am}%Für URL-Datum-Angabe
    \newcommand{\singlepage}{S.}%Für Seitenangabe (einzelne Seite)
    \newcommand{\multiplepages}{S.}%Für Seitenangabe (mehrere Seiten)
    \newcommand{\chapternr}{K.}%Für Kapitelangabe
    \renewcommand{\harvardand}{\&}%Harvardand in Zitaten
    \newcommand{\abstractonly}{ausschließlich Abstract}
    \newcommand{\edition}{. Auflage}%Angabe der Auflage
}{
\iflanguage{german}{
    %Deutsch
    \newcommand{\citepic}[1]{(Quelle: \protect\cite{#1})}%Zitat: Bild
    \newcommand{\citefig}[2]{(Quelle: \protect\cite{#1}, S. #2)}%Zitat: Bild aus Dokument
    \newcommand{\citefigm}[2]{(Quelle: modifiziert "ubernommen aus \protect\cite{#1}, S. #2)}%Zitat: modifiziertes Bild aus Dokument
    \newcommand{\citep}{\citeasnoun}%In-Line Zitiat entweder mit \citep{} oder \citeasnoun{}
    \newcommand{\acessedthrough}{Verf{\"u}gbar unter:}%Für URL-Angabe
    \newcommand{\acessedthroughp}{Verf{\"u}gbar bei:}%Für URL-Angabe (Geschützte Datenbank, Zugriff durch FH)
    \newcommand{\acessedat}{Zugang am}%Für URL-Datum-Angabe
    \newcommand{\singlepage}{S.}%Für Seitenangabe (einzelne Seite)
    \newcommand{\multiplepages}{S.}%Für Seitenangabe (mehrere Seiten)
    \newcommand{\chapternr}{K.}%Für Kapitelangabe
    \renewcommand{\harvardand}{\&}%Harvardand in Zitaten
    \newcommand{\abstractonly}{ausschließlich Abstract}
    \newcommand{\edition}{. Auflage}%Angabe der Auflage
}{
    %Englisch
    \newcommand{\citepic}[1]{(Source: \protect\cite{#1})}%Zitat: Bild
    \newcommand{\citefig}[2]{(Source: \protect\cite{#1}, p. #2)}%Zitat: Bild aus Dokument
    \newcommand{\citefigm}[2]{(Source: taken with modification from \protect\cite{#1}, p. #2)}%Zitat: modifiziertes Bild aus Dokument
    \newcommand{\citep}{\citeasnoun}%In-Line Zitiat entweder mit \citep{} oder \citeasnoun{}
    \newcommand{\acessedthrough}{Available at:}%Für URL-Angabe
    \newcommand{\acessedthroughp}{Available through:}%Für URL-Angabe (Geschützte Datenbank, Zugriff durch FH)
    \newcommand{\acessedat}{Accessed}%Für URL-Datum-Angabe	
    \newcommand{\singlepage}{p.}%Für Seitenangabe (einzelne Seite)
    \newcommand{\multiplepages}{pp.}%Für Seitenangabe (mehrere Seiten)
    \newcommand{\chapternr}{Ch.}%Für Kapitelangabe
    \renewcommand{\harvardand}{\&}%Harvardand in Zitaten
    \newcommand{\abstractonly}{Abstract only}
    \newcommand{\edition}{~edition}%Edition -> note, that you have to write "edition = {2nd},"!
}}}


\maketitle
%% Einführung 

\renewcommand{\dictumwidth}{0.8\textwidth}
\renewcommand{\dictumauthorformat}[1]{#1}
%\renewcommand{\raggeddictumtext}{\raggedleft}
\renewcommand{\dictumrule}{\vskip1ex\par}
\setkomafont{dictumtext}{\itshape}
\setkomafont{dictumauthor}{\bfseries\upshape}
\cleardoublepage
\thispagestyle{empty}
\begin{minipage}[t]{\textwidth}
\vspace{10cm}
\dictum[Thomas Edison]{We are like tenant farmers chopping down the fence around our house for fuel when we should be using Nature's inexhaustible sources of energy \textemdash sun, wind and tide\textellipsis I'd put my money on the sun and solar energy. What a source of power! I hope we don't have to wait until oil and coal run out before we tackle that.}
\end{minipage}
\cleardoublepage


%% Grundlagen
\mainmatter

\chapter{Introduction}
%
It is generally known that the worldwide energy demand is nearly entirely covered by fossil energy sources such as coal, oil, natural gas and uranium. Also, the consequences of fossil fuels to the environmental, social and economic development are globally well known and discussed. Especially the greenhouse gas emissions need to be reduced for the purpose to dilute their effects on climate change. Furthermore, the world population and the energy demand is rising. By 2035 the world population is projected to reach 8.7 billion, this means an additional 1.6 billion people who will need energy. This leads to a rise in primary energy consumption by 37~\% between 2013 and 2035 \cite{BP2015a}. 



Especially newly industrialized countries with a rapid population growth like China and India showed a strong rising primary energy consumption. Also the prognosticated primary energy demand in the newly industrialized country South Africa (SA), officially the Republic of South Africa, is growing from 1~639.83~TWh in 2012 to 2~163.18~TWh (including solid biomass and waist) in 2040. Coal is the mainstay of the South African energy system, meeting around 70~\% of primary energy demand and accounting for more than 90~\% of domestic electricity output. The prognosticated electricity demand is also rising by more than 70~\% from 212~TWh in 2012 to 364~TWh in 2040 \cite{IEA2014f}.



Depending on the growing electricity demand and missing investments in the power supply system in the last decades, the reserve margin in electricity capacity declined from almost 40~\% in 1990 to 8~\% at these days \cite{Trollip2014,Eskom2015}. The region of Western Cape experienced serious blackouts in late 2005 dependence on combination of inadequate reserve margin, insufficient reliability, a stressed system and system element failures. Those were the causes which forced Eskom (Eskom Holdings SOC Ltd.), the South African public utility, to start scheduled load shedding during peak hours, from 2007 on \cite{Trollip2014}. The economic and social impacts are clearly noticeable already. Load shedding was also the main trigger for Bureau for Economic Research (BER) senior economist Hugo Pienaar to revise down the 2015 GDP growth forecast from 2.9~\% to 1.9~\% for SA. \cite{Bisseker2015}.



The high dependency and the scarcity of fossil resources, the intention to reduce greenhouse gas emissions and other environmental pollutants, the rising fuel price, the predicted increase of energy demand, are also forcing SA to promote the investments in renewable energy sources. In 2009, the South African government began to implement feed-in tariffs (FITs) for renewable energies. These were later rejected in favor of competitive tenders. The resulting program, the Renewable Energy Independent Power Producer Procurement Program (REIPPPP) is an extensive initiative to install 17.8~GW of electricity generation capacity from renewable energy sources, such as wind, solar, biomass, biogas and hydropower, over the period 2012–2030. \cite{DEA2015,DoE2013,Eberhard2014}



SA has a high level of renewable energy potential, especially the solar irradiation is one of the highest in the world \cite{IRENA2014}. In some parts of the country the direct normal irradiance (DNI),  the direct beam radiation, rises above 3~000 kWh/m\textsuperscript{2} per year. In comparison to that, Austria has ca. one third of DNI per year \cite{SolarGIS2013a,SolarGIS2013}. Also the global horizontal irradiance (GHI), so the direct and diffuse solar radiation, is with about 2~300~kWh/m\textsuperscript{2} per year comparatively high \cite{SolarGIS2011}. The high solar irradiation allows SA a very effective electricity generation from an infinite source of energy.



Solar energy cannot be used as such, it has to be captured and converted into higher forms of energy, primarily heat and electricity. The conversion from solar energy into electrical energy is carried out by two mechanisms: the photovoltaic conversion and the thermal conversion. For the generation of solar thermal electricity (STE) it is common to concentrate the solar radiation with reflectors onto a receiver where a heat-transfer fluid (HTF) circulates. The fluid is heated up to high temperatures to be used in a thermodynamic cycle to generate electrical power. This technology is called concentrating solar power (CSP). The direct conversion from solar radiation into electricity is based on the photovoltaic effect using photovoltaic (PV) panels. The main difference between those two solar conversion technologies is the type of radiation that can be converted. The CSP technology can only capture and convert the  component of the solar radiation that is directly hitting the collector i.e. the DNI. The PV technology, instead, can furthermore convert the scattered component by clouds, water vapor and particles in the atmosphere and the reflected component due to the albedo effect. Although the CSP systems can exploit only the direct fraction of the overall solar irradiation, CSP plants allow to store the thermal energy with lower costs and lower environmental impacts than storing electric energy generated by PV systems. \cite{IEA2014e,EASAC2011} The levelized cost of electricity (LCOE) of STE and PV varies widely with the location, technology, design and intended use of plants. According to the U.S. Energy Information Administration the estimated total system LCOE for new generation resources in 2019 for PV is at about 130.0~US\$/MWh and for STE at about 243.1~US\$/MWh \cite{Outlook2014}. 



At first it can be summarized that STE technology is more expansive than PV technology, but it allows the use of the cheaper thermal storage technology. The stressed electricity grid in SA needs a constant and controllable energy supply. For a fluctuating energy source a balance mechanism is necessary to guarantee a support and not a pressure for the electricity supply. 

% Was enthält diese Masterarbeit (Aufbau)
\section{Description of the objective of the thesis}
The objective of this thesis was an technical and economical comparison between an concentrating solar and an photovoltaic power plant. This comparison is limited to the region South Africa and including as well the most reasonable storage systems for both technologies. 



Also the medium- and long-term cost digression potential technologies are considered. This is an necessary indicator for common power plants investments in South Africa.



The research question is as follows:
\begin{quote}
Which technology, concentrating solar or photovoltaic power plants in combination with suitable storage systems, will prevail in South Africa in a technical and economical comparison? What are the medium- and long-term cost degression potentials of these technologies?
\end{quote}


Specified generation/load profile 


\section{Relevance of results}
Was sagt mein Ergebniss aus und was kann man damit anfangen
\section{Methodology}

\subsection{Information procurement}
used SI-Units  = always 
\subsection{Quality assurance}

\subsection{Implementation of present resources}

\pagebreak


\chapter{The South African energy sector}
The Republic of South Africa (SA) is one of the most developed country in Sub-Saharan Africa and acording to the Human Development Index (HDI) it is growing constand since the 1980's, nevertheless it counts as medium developed country \cite{UNDP2014}. Also the population and energy demand is constantly growing \cite{TheWorldBank2015,Agency2015}.

SA has also one of the strongest econemys in Africa, therefore it is accounting for about 30~\% of the primary energy consumption of the entire continent Africa in 2014 \cite{BP2015b}. 
\section{Primary energy consumption}
\begin{figure}[!b]
        \centering                
        \begin{subfigure}[b]{0.45\textwidth}
                \centering
                \includegraphics[width=1\textwidth]{FIG/PrimWorld}
                \caption{Worldwide allocation of primary energy consuption.}\label{PrimWorld}
        \end{subfigure}
        ~
        \begin{subfigure}[b]{0.45\textwidth}
                \centering
                \includegraphics[width=1\textwidth]{FIG/PrimSA}
                \caption{SA's allocation of primary energy consuption.}\label{PrimSA}
        \end{subfigure}
\caption[Comparision of primary energy consumption by fuel in 2014.]{Comparision of primary energy consumption by fuel (excl. biomass and waste) in 2014 \cite{BP2015b}.}\label{PEKreis}
\end{figure}
The primary energy consumption of SA was in 2014 about 1~473.52~TWh \cite{BP2015b}. This consumption is mainly based on fossil energy resources. More than 96~\% of the primary energy consumption was in 2014 based fossil fuels and further 2.8~\% on nuclear. Thereby is the share on primary energy consumption predominant coming from coal. Figure \ref{PEKreis} shows the primary energy mix of SA in comparision with the worldwide primary energy mix. \cite{BP2015b}

It can be seen that coal is with about 71~\% the main primary energy source. Also crude oil (23~\%) is a very important energy source for SA. Therefor is the primary energy consumption from renewable energies in SA just about 0.7~\%. Comparing to this, the global share of  renewable primary energy consumption was about 9.3~\% in 2015. But it must be said that the share on renewable energy growth almost five times from 2013 to 2014. \cite{BP2015b}

Figure \ref{PrimEnergyDevelopment} shows the growing South African primary energy consumption. Between 1965 and 2014 the annual primary energy consumption in SA has risen from 351.96~TWh up to 1~473.52~TWh. Consequently a avarage anual growing rate in primary energy consumption in SA of 8.5~\% in the past half century. \cite{BP2015c}

\begin{figure}[htbp]  
\centering
\includegraphics[width=1\linewidth]{FIG/PrimEnergyDevelopment}
\caption[Evolution of primary energy consuption of SA.]{Evolution of primary energy consuption of SA \cite{BP2015c}.}\label{PrimEnergyDevelopment}
\end{figure}
The spread in consumption of primary energy is defined by three major consumption groups, namly the industry sector with about 34.9~\%, the transport sector which consumes about 28.6~\% and other sectors with about 36.5~\%, which includes agriculture, commerce and public services, residential and non-specified consumers \cite{DepartmentofEnergy2012}. So it can be said that the sectors industy and transport are the main energy consumer in SA. 
\pagebreak
\section{Electricity supply and demand}
The electricity market in SA is regulated by the National Energy Regulator of South Africa (NERSA) in terms of the National Energy Regulatory Act from 2004. NERSA's area of responsibility includes the national grid codes, licences, provides, regulations of tariff increases and more. \cite{Eskom2015a}

SA has a fully state-owned and vertically integrated electricity supplier named Eskom (Eskom Holdings SOC Ltd.). Eskom supplies approximately 95~\% of SA's electricity and more than 45~\% of Africa \cite{EskomGenerationDivision2014}. In 2014 SA has a 1.1~\% share of the worldwide electricity consumption with a gross electricity output of 252.6~TWh \cite{BP2015c}. 92.6~\% of SA's primary energy consumption for electricity generation was based on coal fired power plants in 2013 and further 5.5~\% came by nuclear power plant \cite{Agency2015}. Therefore was Eskom in 2009 with 215.91~Mt~CO\textsubscript{2} also worldwide number five of the power companies with the highest CO\textsubscript{2} emissions \cite{CARMA2015}.

Figure~\ref{Electr} compares the South African and worldwide primary energy consumption for electricity generation in 2013. As it is shown, besides coal and nuclear based power generation makes just hydroelectric generation any significant part. \cite{Agency2015}

\begin{figure}[!htbp]
        \centering                
        \begin{subfigure}[b]{0.45\textwidth}
                \centering
                \includegraphics[width=1\textwidth]{FIG/ElectrWorld}
                \caption{World's allocation of primary energy consumption for electricity generation.}\label{ElectrWorld}
        \end{subfigure}
        ~
        \begin{subfigure}[b]{0.45\textwidth}
                \centering
                \includegraphics[width=1\textwidth]{FIG/ElectrSA}
                \caption{SA allocation of primary energy consumption for electricity generation.}\label{ElectrSA}
        \end{subfigure}
\caption[Comparision of primary energy consumption for electricity generation by fuel in 2013.]{Comparision of primary energy consumption for electricity generation by fuel in 2013 \cite{Agency2015}.}\label{Electr}
\end{figure}
As mentioned in 2014 the gross electricity generation was about 252.58~TWh in SA. But when taking an eye on the evolution of the gross electricity generation of the last three decades in Figure~\ref{electrGross} it can be noted that the generation peaked in 2007 with 263,48~TWh. \cite{BP2015c} 

\begin{figure}[htbp]  
\centering
\includegraphics[width=1\linewidth]{FIG/electrGross}
\caption[Evolution of gross electricity generation in SA.]{Evolution of gross electricity generation in SA \cite{BP2015c}.}\label{electrGross}
\end{figure}
The leak on growth momentum of gross electricity generation in SA after the global financial crisis in 2009 is not leading from decreasing request in demand, but more from a power plant maintainence behind schedule and investment backlog. The delayed maintainence schedule was a results from Eskom's “keeping the lights on” philosophy what they are calling nowadays a "not sustainable approach" \cite{Eskom2014}. Eskom’s base load fleet has a average age of about 34 years with a plant availability of about 73~\% \cite{Eskom2015c}. This facts be associated with the rising annual unplanned capability loss factor (UCLF) of Eskom’s power plant fleet. Figure~\ref{UCLF} shows that the UCLF rises from 2008/09 on significantly. In 2014/15 the UCLF reached their maximum of 15.22~\%. The rising UCLF comes in hand with implementing of “load shedding” by Eskom in 2008, which is a planned rolling blackouts based on a schedule in order to protect the power system from a total blackout \cite{Eskom2015d}. Eskom implemted a load shedding schedule in four stages, wich allows Eskom at Stage 4 to dropping of up to 4~000~MW of the national load to balance electricity supply and demand. \cite{Eskom2015e}

\begin{figure}[htbp]  
\centering
\includegraphics[width=1\linewidth]{FIG/UCLF}
\caption[Evolution of Eskom's annual unplanned capability loss factor.]{Evolution of Eskom's annual unplanned capability loss factor \cite{Eskom2015b,Eskom2015d}.}\label{UCLF}
\end{figure}
SA has currently a total nominal installed capacity of 45~699~MW, therefrom Eskom owns and manage 42~090~MW power station capacity in March 2015 \cite{Eskom2015b}. Almost 85~\% of Eskom's power plant capacity are coal-fired (35~721~MW), 4.4~\% nuclear (1~860~MW) and 5.7~\% gas-fired (2~409~MW) which is shown in Figure~\ref{PgenerationEskom}. Besides these fossil and nuclear based capacities Eskom owns also capacities in pumped storage (1~400~MW), hydro (600~MW) and wind (100~MW). Further 3~609~MW capacity coming from independent power producers (IPP) which also include 1~795~MW renewable generation trough the Renewable Energy Independent Power Producer Procurement Program (REIPPPP). \cite{Eskom2015a}

\begin{figure}[htbp]  
\centering
\includegraphics[width=0.45\linewidth]{FIG/PgenerationEskom}
\caption[Eskom's nominal installed power station capacity allocation in 2015.]{Eskom's nominal installed power station capacity allocation in 2015 \cite{Eskom2015a}.}\label{PgenerationEskom}
\end{figure}
The REIPPPP is one of the instruments that the SA Government uses to reaching there ambitious set target for a total installed capacity of 81~350~MW in 2030. From the aspired goal 17~430~MW are planed from wind and solar power. More precisely 9~770~MW by PV and 3~300~MW by CSP. \cite{DoE2013}

After five bid windows (incl. extended bid window in March 2014) the REIPPPP reached a committed capacity of 5~237~MW wherefrome 1~899~MW are commited to PV and 600~MW to CSP. \cite{DoE2015}

Currently most of the power generation is located in the north east of the country. Near to Johannesburg the capital of the province Gauteng are most of the country’s coal-fired power stations and coal mines allocated, as well as the country’s economic and industrial hub. SA is a country of wide open spaces and cities and towns are separated by large distance. Therefore long transmission lines are needed to transport electricity from the large power plants of the Gauteng province to the coastal areas. Figure~\ref{transmissionprojekts} gives an overview of the current situation of the allocations of the power stations and the transmission lines. Also Eskoms future transmission and power station projects are featured in the figure.

\begin{figure}[htbp]
\centering
\includegraphics[width=1\linewidth]{FIG/transmissionprojekts}
\caption[Eskom’s transmission projects as at 31 March 2015.]{Eskom’s transmission projects as at 31 March 2015 \cite{Eskom2015a}.}\label{transmissionprojekts}
\end{figure}
Eskom operates, manages and maintains a total distance of 368~331~km power lines, whereof 31~107~km are transmission power lines at 765~kV to 132~kV and 48~278~km are distribution power lines at 132~kV to 33~kV. The residual power lines are 281~510~km reticulation power lines which are below 22kV and 7~436~km underground cables. Thereby is the country's total transformer capacity 239~490~MVA. The whole power supply system is controlled to a frequency of 50~Hz. \cite{Eskom2015b}

When taking again an eye on Figure~\ref{transmissionprojekts} it can be noted, that the area of Upington in the Northern Cape province is not yet accessed with high voltage transmission lines. Considering that the area of Upington provides the country’s highest solar irradiation value and is therefore a highly attractive site for solar power plants, new high voltage transmission lines to regions with higher load factors are necessary and are currently in its planning phase. Eskom is currently planing additional 3~940~km  transmission lines and 12~815~MVA transformer capacities till 2021 \cite{Eskom2015a}.

Through this electricity network SA covers a electrification rate of around 85~\% and which is the highest on mainland sub-Saharan Africa. About 11~\% of households don't have access to electricity and a further 4~\% rely on illegal access (non-paying) or obtain access informally (from one household to another but paying). \cite{IEA2014f}

In the financial year 2014/15 Eskom sales about 216~274~GWh. The spread of the main customers is shown in Figure~\ref{ElectricityShare}. From this it appears that the Municipalities are Eskom's biggest consumers. But also the Industrial and Mining sector makes together a considerable part of 38.6~\% of the electricity consumption. The residential part of the electricity consumption was just about 5.36~\% and the commercial part was below 5~\%. The main part of the exported electricity with about 70~\% went to the neighboring country Mosambique and further 10~\% to Botswana. Namibia and Swaziland mades just 8 or 7~\% of the exported electricity. \cite{Eskom2015b}  
\begin{figure}[!h] 
\centering
\includegraphics[width=0.4\linewidth]{FIG/ElectricityShare}
\caption[Eskom's electricity sales per customer category.]{Eskom's electricity sales per customer category \cite{Eskom2015b}.}\label{ElectricityShare}
\end{figure}
\pagebreak

\chapter{Solar energy in South Africa and load curve covering concept}\label{Solar power in South Africa}
Almost all power that we use on our planet comes from the sun. Direct in form of radiation or indirect during wind, water and vegetation. Also the fossil power resources and reserves are stored energy from the sun in from of organic carbon compounds. 

The sun emits a power rate of about 3.83x10\textsuperscript{26}W. Of this total, only a tiny fraction, \SI{1367}{\watt\per\square\metre} (solar constant) reaches the Earth’s atmosphere. The solar radiation is reduced by absorption and reflection effects in the atmosphere.  The reduction is about \SI{30}{\percent} on a clear day and about \SI{90}{\percent} on a very cloudy day. \cite{Stine2001a}

When taking an eye on the world map in Figure~\ref{WorldDNI} it can be noticed that some parts of the world receive much higher direct parts of the sun’s irradiation than others. In particular four regions worldwide are worth mentioning. The Atacama Desert in South America, the Mojave Desert in North America, a huge part of Australia and parts of the southern Africa. Therefore SA is one of the country with the highest potential for generating solar electricity in the world.

\begin{figure}[h!] 
\centering
\includegraphics[width=1\linewidth]{FIG/WorldDNI}
\caption[World map of Direct normal irradiation.]{World map of Direct normal irradiation \cite{SolarGIS2015c}.}\label{WorldDNI}
\end{figure}  
\section{Solar irradiation in South Africa}
As shown above, solar irradiation is highly depending from the location. The solar irradiance of a specific location can be measured on-site by ground measurement devices or site-adapted by interpolated satellite data, which is validated with other ground measurement devices. Thereby is the direct and indirect as well as the total sun irradiance crucial. These solar irradiation parameter are here defined:
\begin{itemize}
\item \textbf{Global Horizontal Irradiance (GHI)} in \si{\watt\hour\per\square\metre\year} or \si{\watt\per\square\metre}: GHI is the total amount of shortwave radiation received from above by a horizontal surface. It includes direct (beam) and a diffuse (scattered) irradiation. This value is of particular interest to PV or solar water heater with a fixed inclined angle.
\item \textbf{Direct Normal Irradiance (DNI)} in \si{\watt\hour\per\square\metre\year} or \si{\watt\per\square\metre}: DNI is the amount of solar radiation received per unit area by a surface that is always held perpendicular (or normal) to the rays that come in a straight line from the direction of the sun at its current position in the sky. Diffuse irradiation is totally excluded from the DNI. This quantity is of particular interest to  installations that track the position of the sun.
\item \textbf{Diffuse Horizontal Irradiance (DHI)} in \si{\watt\hour\per\square\metre\year} or \si{\watt\per\square\metre}: DHI is the amount of radiation received per unit area by a surface that does not arrive on a direct path from the sun, but has been scattered by molecules and particles in the atmosphere and comes equally from all directions.
\end{itemize}
Furthermore is irradiance understood as instantaneous density of solar radiation incident on a given surface, typically expressed in \si{\watt\per\square\metre} and irradiation is the sum of irradiance over a time period expressed in \si{\joule\per\square\metre} or more commonly used in \si{\watt\hour\per\square\metre}. The connection between the solar radiation parameters is shown in Equation \ref{GL_GHI}. The angle $\theta_\text{z}$ is the angle between the direction of the sun and the zenith (directly overhead).
\begin{align}
GHI=DNI*\cos(\theta_{z})+DHI \label{GL_GHI}
\end{align}
The GHI is decisive for the power output of PV systems and the DNI for CSP systems. Figure\ref{irradiation} shows the solar GHI and the DNI data for SA. It is shown, that the ceiling value for GHI can be more than \SI{2300}{\kilo\watt\hour\per\square\metre\year}, whereas in some parts of the country the DNI  value attains about \SI{3200}{\kilo\watt\hour\per\square\metre\year}. This is significantly high than in the most regions worldwide, therefor SA is predestined for using solar technologies. The figure shows, that the southeastern coastline has predominantly the lowest irradiance values. The solar irradiation rise significant in the inland. The highest GHI can be find close to the Namibian boarder in the northeast of the country. The direct beam is also at highest in the western part of SA. The area around Springbok in the province Northern Cape has the highest DNI value of the country.

\begin{figure}[h!]
        \centering
        \begin{subfigure}[b]{0.5\textwidth}
                \centering
                \includegraphics[width=1\textwidth]{FIG/SA_GHI}
                \caption{Global Horizontal Irradiation \cite{SolarGIS2015a}.}\label{fig:bild-links}
        \end{subfigure}%
        ~
        \begin{subfigure}[b]{0.5\textwidth}
                \centering
                \includegraphics[width=1\textwidth]{FIG/SA_DNI}
                \caption{Direct Normal Irradiation \cite{SolarGIS2015b}.}\label{fig:bild-rechts}
        \end{subfigure}
        \caption{Solar radiation maps of South Africa.}\label{irradiation}
\end{figure}
Both maps demonstrates, that the highest values of solar irradiation can be found in the northwestern part of SA, which allocated in the Northern Cape Province. Currently all CSP plants and about two-thirds of the PV systems of SA are developed in the Northern Cape \cite{Forder2015}. Thereby the region around the city of Upington is highly attractive, owning to the high irradiation value in connection with a reliable water access due to the Orange River and the possibility of a close access to the Eskom grid. 

Therefore the location parameter and weather data of Upington was selected for the simulation and calculations in this thesis \cite{WhiteBoxTechnologies2015}. The hourly values of GHI and DNI over a full year for Upington are shown in Figure~\ref{Upington_GHI/DNI}. The highest irradiance value during the summer are \SI{1199}{\watt\hour\per\square\metre} for GHI and \SI{1154}{\watt\hour\per\square\metre} for DNI. At the northern solstice, the highest irradiance is \SI{625}{\watt\hour\per\square\metre} for GHI and \SI{820}{\watt\hour\per\square\metre} for DNI. 

So it is obviously that the irradiation values demonstrates significant seasonal variation of GHI with high values in summer and low irradiation in winter, whereas the DNI shows a more balanced variation throughout the year. This mainly leads from the irradiation angle which is changing constantly during the year. 

\begin{figure}[!htbp]
        \centering
                \begin{subfigure}[b]{1\textwidth}
                \centering
                \includegraphics[width=1\textwidth]{FIG/Upington_GHI}
                \caption{Global horizontal irradiance}\label{Upington_GHI}
        \end{subfigure}%
\par\medskip % Linebreak      
        \begin{subfigure}[b]{1\textwidth}
                \centering
                \includegraphics[width=1\textwidth]{FIG/Upington_DNI}
                \caption{Direct normal irradiance}\label{Upington_DNI}
        \end{subfigure}%

        \caption[Hourly values of irradiance over a full year from Upington used for the simulation.]{Hourly values of irradiance over a full year from Upington used for the simulation.}\label{Upington_GHI/DNI}
\end{figure}
The path of the sun during the year in Upington is characterize in Figure~\ref{SunPathUpington}. The solar path diagram depends on the geographical location by the position of longitude and latitude. The diagram apparent that the longest day in Upington has a duration of \SI{13}{h} and \SI{56}{minutes} with a maximum sun height of \SI{85.05}{\degree} while the shortest day has a duration of just \SI{10}{h} and \SI{19}{minutes} and a maximum sun height of \SI{35.93}{\degree}.

\begin{figure}[htbp]  
\centering
\includegraphics[width=1\linewidth]{FIG/SunPathUpington}
\caption[Solar path diagram for Upington.]{Solar path diagram Upington \cite{PVsystSA2015}.}\label{SunPathUpington}
\end{figure}
\pagebreak
From the hourly irradiation values concludes a annual sum of \SI{2280}{\kilo\watt\hour\per\square\metre\year} by GHI and an annuall DNI amount of \SI{2621}{\kilo\watt\hour\per\square\metre\year}. When comparing this for the simulation used data with the solar irradiation maps in Figure~\ref{irradiation} the annual sum of GHI is corresponding. But it must be noted that the annual sum of DNI from SolarGIS map \cite{SolarGIS2015b} is about \SI{200}{\kilo\watt\hour\per\square\metre\year} higher than the value which was used for the simulation in this thesis.

The weather data for the simulation are in EPW format (EnergyPlus Weather Data) and produced by White Box Technologies, Inc. \cite{WhiteBoxTechnologies2015}. The EPW files are data sets of hourly values of solar radiation and meteorological elements for a typical one-year period. These include air temperature (\si{\celsius}), dew point temperature (\si{\celsius}), relative humidity (\si{\percent}), atmospheric pressure(\si{\milli\bar}), global horizontal solar radiation (\si{\watt\per\square\metre}), diffuse horizontal solar radiation (\si{\watt\per\square\metre}), direct normal radiation (\si{\watt\per\square\metre}), wind speed (\si{\metre\per\second}), wind direction (\si{\degree}) and snow depth (\si{\metre}). The most relevant parameters for the simulation are summarized in Table \ref{tbl: Location}. 
 
\begin{table}[!h]  
  \centering
	\begin{tabular}{  p{4.0cm}  C{4.0cm}  C{3.0cm} } 
	\hline	
\textbf{Item}  & \textbf{Value} & \textbf{Unit} \\ \hline \hline
Location & Upington & -\\ 
Station ID &  684240& -  \\ 
Data source & White Box Technologies, Inc. (31.05.2015) & -\\ \hline
Latitude & -28.40 &$\,^{\circ}$N \\ 
Longitude &  21.27 &$\,^{\circ}$E \\ 
Elevation &  836 & m \\ 
Total GHI per year  &  2~280 & \si{\kilo\watt\hour\per\square\metre}\\ 
Total DNI per year &  2~621 & \si{\kilo\watt\hour\per\square\metre}\\ 
Total DHI per year &  516 & \si{\kilo\watt\hour\per\square\metre}\\ 
Mean temp. &  21 & \si{\celsius}\\ 
Mean wind speed & 3.3 & \si{\metre\per\second}\\ \hline
\end{tabular}
\caption[Location and characteristics for the simulation in SAM.]{Location and characteristics for the simulation in SAM.}\label{tbl: Location}
\end{table}
\pagebreak
\section{Current stage of solar power in South Africa}
South Africa started there expansion in the field of solar power plants with the first round of the REIPPPP in 2011. As it was mentioned before in Section~\ref{ElectricitySA} the REIPPPP has currently a capacity of \SI{5237}{\mega\watt} committed inclusive \SI{1899}{\mega\watt} PV systems and \SI{600}{\mega\watt} CSP plants. Additionally comes one CSP project with \SI{100}{\mega\watt} from Eskom which is not a part of the REIPPPP.

Figure \ref{Solar-map} shows the allocation of all currently committed solar power plants of the REIPPPP. Yellow marked are PV-power plants and CSP plants are marked in orange (some marks cover them mutually). The numbers in the single marks expose to which REIPPPP-Round it belongs.

Currently SA has 27 fully operational PV-power plants with a total capacity of \SI{1059.05}{\mega\watt} further six PV-power plants with in total of \SI{442.5}{\mega\watt} are under construction. \cite{Forder2015}

Most of the South African PV-power plants are allocated in the Northern Cape. So 29 out of 45 currently committed PV systems are located there. Five more each in the Western Cape and the North-West Province, three each in the provinces Free State and Limpopo, two in Eastern Cape and one more in the Eastern Cape. \cite{Forder2015}
\pagebreak

\begin{figure}[h!]
\centering
\includegraphics[width=1\linewidth]{FIG/Solar-map}
\caption[Allocation of all REIPPPP solar power plants in SA.]{Allocation of all REIPPPP solar power plants in SA \cite{Forder2015}.}\label{Solar-map}
\end{figure}

"KaXu Solar One" is the first and currently only full operational CSP plant in SA. It is using parabolic trough technology and a 2.5~h thermal energy storage for generating \SI{100}{\mega\watt} capacity. The CSP-plants "Khi Solar One" and "Bokpoort CSP Project" with a capacity of each 50~MW are under construction as well as the \SI{100}{\mega\watt} "Xina CSP South Africa" project. Further three CSP-plants are awaiting construction, they have all a capacity of each \SI{100}{\mega\watt}. As mentioned before, all eight CSP projects are located in the Northern Cape Province of SA. \cite{Forder2015}
\pagebreak
\section{System load in SA and prescribed solar power generation profil} \label{SystemloadinSA}
The South African supply structure is based on coal-fired thermal power plants, as it was shown in Section~\ref{ElectricitySA} before. But also the renewable energy supply is getting noteworthy for the supply structure in SA. When taking an eye on Figure~\ref{systemload} it can be noted that wind and PV power also makes a small share of the daily electricity supply. It has to be said that by the time of spring in 2014 the KaXu Solar One CSP plant was not completed and so CSP supply in SA doesn't exist. The reveals that the power demand in SA rises in the morning hours and is coming to a peak demand in the evening hours. The surplus energy during the night got compensated by pump storage and reduction by supply imports. 

\begin{figure}[htbp]  
\centering
\includegraphics[width=1\linewidth]{FIG/systemload}
\caption[Actual South African supply structure for a spring day, the 17. October 2014.]{Actual South African supply structure for a spring day, the 17. October 2014 \cite{CSIR2015}.}\label{systemload}
\end{figure}
It is shown that the PV supply can support the South African demand during the day and reduces thereby the generation of coal-fired thermal power plants. But by the time of the daily peak demand, in the evening hours, the PV supply is coming to standstill, whereby this demand needs to be covered by pumped storage or expansive diesel driven open cycle gas turbines (OCGTs).

When taking an eye on the total unserved energy due to load shedding history of the first half year of 2015 in Figure~\ref{Load_shedding_sum} it is obviously that the capacity bottleneck in SA is mostly during the daytime and particularly during the evening peak hours till 22:00. This makes clear in which time period the support of new power plants is necessary for a secure electricity supply in SA.

\begin{figure}[htbp]  
\centering
\includegraphics[width=1\linewidth]{FIG/Load_shedding_sum}
\caption[Total unserved energy due to load shedding for all hours per month Jan-Jun 2015 in GWh.]{Total unserved energy due to load shedding for all hours per month Jan-Jun 2015 in GWh \cite{CSIREnergyCentre2015}.}\label{Load_shedding_sum}
\end{figure}
It is a  hard requirement that plants meet the full electrical demand in the system. Figure~\ref{LoadScenarios} shows the daily average system load/demand in South Africa for the winter and summer period (2011). It can be seen that the summer profile has a particularly lower demand than the winter profile (up to \SI{15}{\percent} difference) and after a sharp rise at approx 7:00 the demand has a constant system load during the day due to commercial, agricultural and residential demand caused by air-conditioning, pool pumps and geysers, which comes to their peak demand at 20:00. In winter, there is an initial peak at 9:00 and a second at 19:00 mainly caused by residential customers due to electrical heating, lighting, cooking, geysers and pool pumps.

\begin{figure}[htbp]  
\centering
\includegraphics[width=1\linewidth]{FIG/LoadScenarios}
\caption[South Africa daily average system load/demand for summer and winter days, with prescribed generation profile.]{South Africa daily average system load/demand for summer and winter days, with prescribed generation profile.}\label{LoadScenarios}
\end{figure}

Considering the system load, the supply structure and the point of time by load shedding it is unequivocally that the South African power supply needs adjustable power plants which can support the demand during the evening hours till 22:00 in particular. 

In order to support the South African power supply were it is needed, the for the comparison selected solar power plants are forced to cover a prescribed generation profile. The solar power plants must operate at full power output of \SI{100}{\mega\watt} from 7:00 to 22:00. When the system demand drops during the night, the plants reduce there output from 22:00 to 7:00 to \SI{50}{\mega\watt}. At a covering of \SI{100}{\percent} of the prescribed generation the solar power plants would produce \SI{711.75}{\giga\watt\hour} per year.

With consideration to solar energy is depending on weather and seasonal conditions the individual system design must handle \SI{90}{\percent} of the prescribed generation profile and with regard to the PV system whose module efficiency deteriorates over there lifetime it is just defined for the first year of the plants. It signifies that the power plants are forced to produce more than \SI{640575}{\mega\watt\hour} in the first year.

Usually, in feed-in contracts, the deliverable maximum net power output of power plants is fixed. Any overproduction is not covered by such contracts and is not remunerated. In order to generate exploitable and comparable results and considering standard feed-in contracts, the hourly power production is cut to the planned values and thus overproduction is not considered in this analysis.

The results of the solar power plant simulations will therefore evaluated and compared by the following quantitative measures:
\begin{itemize}
\item \textbf{Generation curve covering} [\si{\percent}]: The generation curve covering (GCC) is an effectiveness measure describing how closely the plant follows the prescribed generation curve over the full year. It is defined as:

\begin{equation}
\mbox{GCC} = \sum\limits_{t=1}^{8760} \frac{\mbox{prescribed generation at }t\mbox{ in MW}}{\mbox{actual plant net output at }t\mbox{ in MW}} \label{GL_GCC}
\end{equation} 

\item \textbf{Levelized cost of electricity} [USD/\si{\mega\watt\hour}]: The levelized cost of electricity (LCOE) represents the total project life-cycle costs. It is the present value of project costs expressed in USD per megawatt-hour of electricity generated by the system over its life.
\end{itemize}
These performance and financial parameter are the only releavant indicator for the evaluation in this thesis, other indicators can be seen as supplement. 

It must be noted, that there is an important difference between the GCC and the widely-used \emph{capacity factor} (CF). The CF is the ratio of the system's electrical net output in the first year of operation to the nameplate output (CSP) or peak output (PV), which is equivalent to the quantity of energy the system would generate if it operated at its nameplate capacity for every hour of the year \cite{NREL2015a}. The GCC is calculated from the sum value of prescribed generation coverage in each hour of the year. 
\pagebreak 

\chapter{Large scale solar power plants}
%Almost all power that we use on our planet comes from the sun. Direct in form of radiation or indirect during wind, water and vegetation. Also the fossil power resources and reserves are stored energy from the sun in from of organic carbon compounds. There are two main technologies for generating electricity out of direct sun radiation. One is the direct conversion of solar irradiance to electrical energy while using photovoltaic. The other is to generate heat and convert it to electrical power. Figure~\ref{OverviewSTP} gives an abstract of the common technologies using direct solar power to generating electric power. This chapter has the focus on large-scale solar power plants and describes parts from both technologies.
Virtually all energy consumed on Earth comes from the sun, whether directly in the form of solar radiation, or indirectly through wind, the water cycle or vegetation. There are two main technologies for generating electricity out of direct solar radiation. One is the direct conversion of solar irradiance to electrical energy via the photovoltaic effect. The other is to first convert that solar irradiance into heat and use that heat to generate electricity. Figure~\ref{OverviewSTP} graphically shows the relationships between common solar technologies for generating generating electric power. This Chapter introduces the solar technologies to cover the in Section~\ref{SystemloadinSA} defined power plant net output load curve. Therefore the technologies which are the focus of this work are highlighted in red.

\begin{figure}[!h] 
\centering
\includegraphics[width=0.8\linewidth]{FIG/OverviewSTP}
\caption[Overview of Solar Power Technologies.]{Overview of Solar Power Technologies.}\label{OverviewSTP}
\end{figure}



%Therefore it is split in the technology fields of large scale concentrated solar power plants (\ref{Large scale concentrated solar power (CSP) plants}) and large scale photo voltaic power plants (\ref{Large scale photo voltaic (PV) power plants}).
%In detail the technologies with a large-scale power plant potential -- parabolic trough, central receiver and non-concentrating photovoltaic -- are described. Also the storage systems for both systems. 
%Therefore it is split in the technology fields of large scale concentrated solar power plants (\ref{Large scale concentrated solar power (CSP) plants}) and large scale photo voltaic power plants (\ref{Large scale photo voltaic (PV) power plants}).
%In detail the technologies with a large-scale power plant potential -- parabolic trough, central receiver and non-concentrating photovoltaic -- are described. Also the storage systems for both systems. 

\section{Large-scale CSP plants}\label{Large scale concentrated solar power (CSP) plants}
%Concentrating solar power (CSP) systems use combinations of mirrors or lenses to concentrate direct beam solar radiation to produce forms of useful energy such as heat, electricity and others. This happens by use of various downstream technologies. Generally the CSP technology includes not only the concentrating solar thermal (CST) technology, but also concentrating photovoltaic (CPV) energy conversion. However, there is no focus on CPV in this thesis. That is why the term CST is put on a level with CSP.
Concentrating solar power (CSP) systems use combinations of mirrors or lenses to concentrate direct beam solar radiation to produce forms of useful energy such as heat and electricity. This happens by use of various downstream technologies. Generally, CSP technology includes not only concentrating solar thermal (CST) technology, but also concentrating photovoltaic (CPV) energy conversion. However, CPV will not be treated in this thesis.
%A CSP plant comprises four main sub-systems: concentrating system, solar receiver, storage and power block. Also supplementary firing is used in some cases, but is basically not necessary nowadays. A graphic scheme of such a sub-system is shown in Figure~\ref{MainComp}. The separate components are linked together by energy flow in mostly radiation transfer or fluid transport. This chapter describes the function and gives an application overview of the individually components. 
A CSP plant comprises four main sub-systems: concentrating system, solar receiver, storage and power block (Figure~\ref{MainComp}). (Supplementary firing was used in the past, but is not necessary nowadays.) The separate components are energetically linked via radiation or fluid transport.
\begin{figure}[!h] 
\centering
\includegraphics[width=0.85\linewidth]{FIG/MainComp}
\caption[Main components of a CSP plant.]{Main components of a CSP plant.}\label{MainComp}
\end{figure}

%The main advantage of CSP in opposite to other renewable energy producers is the thermal storage to provide power for cloudy days or during night time. Therefore the concentrating system and solar receiver have to produce more thermal energy then the power block can use directly. The ratio of the power capacity of the collector field to the capacity of the power block is defined as Solar multiple (SM). For CSP systems with storage, the number of hours of storage is based on the capacity of the power block. Chapter \ref{Subsection_storage_system} describes technical possibilities and application of thermal storage system for CSP.
The main advantage of CSP when compared to other renewable energy producers is its thermal storage, which continues to provide power on cloudy days or at night. This means that the concentrating system and solar receiver must produce more thermal energy then the power block can use directly. The ratio of the output of the collector field to the gross output of the power block is defined as \emph{solar multiple} (SM). For CSP systems with storage, the number of hours of storage is based on the capacity of the power block.

%The solar receiver or concentrating system is eponymously for the main CST technologies. The two most common CSP plant technologies are parabolic trough collector (PTC) and central receiver (CR) systems (also known as solar power towers). Further types of CSP plant are linear Fresnel reflectors (LFR) and parabolic dish. The main difference of the technologies is the concentrating system. Thereof results the differences in optical design, shape of receiver, nature of the transfer fluid and capability to store heat before it is turned into electricity. In systems with a line focus (PTC trough and LFR) the mirrors track the sun along one axis. In those with a point focus (CR and parabolic dish), the mirrors track the sun along two axes. The receiver may be fixed, as in LFR and CR, or mobile as in PTC and parabolic dish systems. An overview of the technologies and there differences in relation to the focus and the receiver  is shown in Table \ref{tbl: CSPtech}.
The two most common CSP plant technologies are parabolic trough collector (PTC) and central receiver (CR) systems (also known as solar power towers). Other types include linear Fresnel reflector (LFR) and parabolic dish. The main difference between these technologies is the concentrating system, which lead to differences in optical design, shape of receiver, nature of the transfer fluid and capability to store heat before it is turned into electricity. In systems with a line focus (PTC and LFR) the mirrors track the sun along one axis. In those with a point focus (CR and parabolic dish), the mirrors track the sun along two axes. The receiver may be fixed, as in LFR and CR, or mobile as in PTC and parabolic dish systems (see Table \ref{tbl: CSPtech}).

\begin{table}[h!] % Main technologies 
  \centering
  \begin{tabular}{  m{5cm}  m{5cm}  m{5cm}  }
    \hline
    & \textbf{Line focus} & \textbf{Point focus} \\ 
    & Collectors track the sun along a single axis and focus irradiance on a linear receiver. This makes tracking the sun simpler. & Collectors track the sun along two axes and focus irradiance at a single point receiver. This allows for good receiver efficiency at higher temperatures.\\ \hline \hline
    \textbf{Fixed receiver} & &\\

    Fixed receivers are stationary devices that remain independent of the plant's focusing device. This eases the transport of collected heat to the power block.
    &
    \begin{minipage}[t]{5cm}
      \centering
	 \includegraphics[height=55mm]{FIG/SUM/LF}
    \end{minipage}
    & 
    \begin{minipage}[t]{5cm}
      \centering
	  \includegraphics[height=55mm]{FIG/SUM/ST}
    \end{minipage}
    \\ \hline
    \textbf{Mobile receiver} & & \\
    Mobile receivers move together with the focusing device. In both line focus and point focus designs, mobile receivers collect more energy.
    &
    \begin{minipage}{5cm}
      \centering
	  \includegraphics[height=55mm]{FIG/SUM/PT}
    \end{minipage}
    & 
    \begin{minipage}{5cm}
      \centering
	  \includegraphics[height=55mm]{FIG/SUM/PD}
    \end{minipage}    
    \\ \hline
  \end{tabular}
  \caption[CSP technology families.]{CSP technology families \cite{IEA2014b}.}\label{tbl: CSPtech}
\end{table}


%As mentioned above is main difference between the CSP technology families how they concentrate the solar radiation. This strongly affects their overall efficiency. The parabolic dish has the best annual optical efficiency (about 90\%) because the concentrator axis is always parallel to the sun rays. The worst (about 50\%) is observed for linear Fresnel systems because of poor performance in the morning and in the evening. Intermediate values (65-75\%) are obtained for parabolic trough and tower systems. For each family the actual efficiency varies with the location, the time of day and the season of the year. \cite{EASAC2011} 
The main difference between the CSP technology families is in how they concentrate the solar radiation, which affects their overall efficiency. The parabolic dish has the best annual optical efficiency (\sim\SI{90}{\percent}) because the concentrator axis is always parallel to the sun's rays. The worst (\sim\SI{50}{\percent}) is observed for linear Fresnel systems because of poor performance at low solar irradiation angles. Intermediate efficiencies (\SIrange{65}{75}{\percent}) are obtained for parabolic trough and central receiver systems. For each family, actual efficiency varies with location, time of day and season \cite{EASAC2011}.

%The capacity range of an CSP-plant is also strongly affected by the concentration ratio. The most common definition of concentration ratio is the ratio of the area of reflector aperture ($A_a$) to the area of receiver ($A_r$). The area concentration is:
The capacity range of a CSP plant is also strongly affected by the concentration ratio, which is defined as the ratio of the area of reflector aperture ($A_a$) to the area of receiver ($A_r$):

\begin{align}
C=\frac{A_{a}}{A_{r}} \label{GL_concentration}
\end{align}
%The concentration ratio from Equation \ref{GL_concentration} has an upper limit that depends on whether the concentration is a three dimensional (point focus) concentrator such as a parabolic dish and central receiver solar tower or a two-dimensional (linear focus) concentrator such as parabolic trough and linear Fresnel reflector. The maximum concentration ratio is based on the second law of thermodynamics applied to radiative heat exchange between the sun and the receiver. The maximum possible concentration ratio for circular concentrators is 45~000, and for linear concentrators is the maximum 212. \cite{Duffie2013}
The concentration ratio from Equation \ref{GL_concentration} has an upper limit that depends on whether the concentration is a three-dimensional (point focus) concentrator such as a parabolic dish or central receiver, or a two-dimensional (linear focus) concentrator such as a parabolic trough or linear Fresnel reflector. The maximum concentration ratio is based on the second law of thermodynamics applied to radiative heat exchange between the sun and the receiver. The maximum possible concentration ratio for circular concentrators is \num{45000}, for linear concentrators it is \num{212} \cite{Duffie2013}.

\begin{table}[h!]  
  \centering
	\begin{tabular}{  p{3.0cm}  C{2.0cm}  C{2.2cm}  C{2.0cm}  C{2.0cm}  C{2.0cm}} 
\hline
\textbf{Technology} & \textbf{Capacity range} (\si{\mega\watt}) & \textbf{Concent- ration} & \textbf{Peak system efficiency} (\si{\percent}) & \textbf{Annual system efficiency} (\si{\percent}) & \textbf{Thermal cycle efficiency} (\si{\percent}) \\ \hline \hline
Parabolic trough & 10-280$^1$ & 70-100 & 21 & 10-16 & 35-42 ST  \\ \hline
Fresnel reflector & 10-200 & 25-100 & 20 & 9-13 & 30-42 ST  \\ \hline
Solar tower & 10-200 &  300-1~000 & 23 & 8-23 & 0-45 ST  \\ \hline
Dish-Stirling & 0.01-0.4 & 1~000-3~000 & 29 & 16-28 & 30-40  \\ \hline
\multicolumn{2}{l}{ST = Steam Turbine}
\end{tabular}
\caption[Performance characteristics CSP technology families.]{Performance characteristics CSP technology families \cite{Pitz-Paal.2013} \cite{AbengoaSolar2013a}$^1$.}\label{tbl: CSPCharacteristics}
\end{table}

%But actually the technical implementation of concentration ratio is the main parameter for the capacity range of a CSP plant. Table~\ref{tbl: CSPCharacteristics} gives an overview of some of the performance characteristics of the concentrating solar power concepts. More details are listed in Annexure I, Part A, Figure~\ref{CSPOverview1} on Page~\pageref{CSPOverview1} and Figure~\ref{CSPOverview2} on Page~\pageref{CSPOverview2}. PTC, LFR, and CR can be coupled to steam cycles of 10-280~MW electric capacity (and more), with thermal cycle efficiencies of 30-45~\%. Also the applies for stirling engines which are coupled to dish systems have similar efficiency ranges. The conversion efficiency of the power block remains essentially the same as in conventional-fired power plants. The annual system efficiency are the net power generation over incident beam radiation. They are lower than the conversion efficiencies of conventional steam or combined cycles, because they include the conversion of solar radiative energy to heat within the collector and the conversion of the heat to electricity in the power block. \cite{Pitz-Paal.2013}
% I do not understand the intended meaning of this sentence:
%But actually the technical implementation of concentration ratio is the main parameter for the capacity range of a CSP plant.

Table~\ref{tbl: CSPCharacteristics} gives an overview of performance characteristics. PTC, LFR, and CR systems can be coupled to steam cycles of \SIrange{10}{280}{\mega\watt} electric capacity (and more), with thermal cycle efficiencies of \SIrange{30}{45}{\percent}. Stirling engines that are coupled to dish systems have similar efficiency ranges. The conversion efficiency of the power block is consistent with conventional power plants. The annual system efficiency is the net power generation over incident beam radiation. This is lower than the conversion efficiencies of conventional steam or combined cycles because it includes include the conversion of solar radiative energy to heat within the collector and the conversion of the heat to electricity in the power block \cite{Pitz-Paal.2013}.

\begin{figure}[!h] 
\centering
\includegraphics[width=0.65\linewidth]{FIG/CSP_technology_development}
\caption[Historical development of CSP technologies.]{Historical development of CSP technologies \cite{Abbas2015}.}\label{CSP_technology_development}
\end{figure}

%The development of the actual CSP plant technology goes back in the 1970's and 1980's, which is a consequent from the first two Oil crises. Figure~\ref{CSP_technology_development} shows the historical development of PTC, LFR in the Figure called linear Fresnel collector (LFC) and the CR (tower). The parabolic dishes have not succeeded at all and aren't shown here. The reason for that is mainly due to the high structural costs of moving an large diameter dish with two-axes tracking. One might observe that the predominant technology is the PTC, with an installed capacity well above 3~GW. CR have started the exponential development some years later compared to PTC, which explains why the operational installed capacity is much lower. Similarly the development for LFR have started even later, which drives to the lowest installed capacity of the three technologies. The difference in the timing of the three successful technologies is very influenced by the CSP development in its first golden period, the 1980's. In such period important central tower prototypes were built in USA (Solar One, Solar Two, CESA-1) and a 365~MW PTC solar plant was installed in the Mojave Desert. When the oil prices dropped at the end of the second oil crises interest on renewable energies was lost until the last decade. 
The development of the CSP technology began in earnest in the 1970s, a consequence of the first two oil price shocks. Figure~\ref{CSP_technology_development} shows the historical development of PTC and LFR (called linear Fresnel collector (LFC) here) technologies and the CR (tower). Parabolic dish systems have not been widely adopted and are not shown here; this is mainly due to the high costs associated with moving a large diameter dish with two-axis tracking. The predominant technology remains PTC, with an installed capacity well above \SI{3}{GW}. The first commercial scaled CR systems started decades later, and that of linear Fresnel reflector systems later still; the operational installed capacity is proportional to the number of years the technology has been available. In the 1980s, important central tower prototypes were built in the USA (\emph{Solar One}, \emph{Solar Two}, \emph{CESA-1}) and the \SI{365}{\mega\watt} PTC solar plant was built at Kramer Junction in the Mojave Desert. When the oil prices dropped after the second oil crisis, interest in renewable energies was lost and it did not return until the beginning of the new millenium. 

With regard to the past development in the different CSP technology deals this theses with the PTC as well as with the CR technology. The most relevant facts to the separate technologies and there technical components are described in the following shortly in order to describe there technical possibilities to cover the prescribed load curve. 

\subsubsection{Overview of CR system technology}
As it was mentioned above, a CSP system basically consists of five main part, namely concentration system, solar receiver, storage, power block and heat transfer fluid (HTF) system. Which are in case of a CR power plant a solar field (heliostat field) consisting by several two-axis tracking heliostats, a thermal receiver which is mounted on a tower, a thermal storage usually using molten salt or a steam accumulator (depending on HTF), a power block mostly based on Rankine cycle technology and steam or molten salt as HTF. Figure~\ref{towerdirecttwotank} shows the simplified CR power plant scheme of the demonstration project \emph{Solar Two} (1995-2001). 

Currently CR systems can be separated by two receiver types, external (tube) receiver and cavity receiver \cite{Hoffschmidt2014}. The \emph{Solar Two} project used a external receiver to heat up molten salt at a top temperature of \SI{565}{\celsius} \cite{Reilly2001}. This project concept was the basic research for the current state of art in CR power plants using molten salt as HTF. Typical solar heat fluxes in this type of receiver are up to \SI{1}{\mega\watt\per\square\metre} \cite{Pitz-Paal.2013}. Cavity receiver are currently mostly used in connection with direct saturated or super heated steam production, which is not easy to store over longer duration and therefore not suitable for the defined load profile in thesis \cite{Hoffschmidt2014,Steinmann2015}.

\begin{figure}[htbp]  
\centering
\includegraphics[width=0.45\linewidth]{FIG/towerdirecttwotank}
\caption[Simplified scheme of \emph{Solar Two} CR power plant with direct storage of molten salt used as heat transfer fluid.]{Simplified scheme of \emph{Solar Two} CR power plant with direct storage of molten salt used as heat transfer fluid \cite{Richter2013}.}\label{towerdirecttwotank}
\end{figure}

Classical two-axis tracking heliostats design is dominated by mirrors brought into position by steel structures and drives that guarantee high accuracies underwind loads and thermal stress situations \cite{Alexopoulos2013}. The reflective surface of can therefore reach sizes from \SIrange{1.1}{320}{\square\metre} \cite{Blackmon2012,Tyner2014}. There is not really one specific size pushed through for commercial scale application, but one of the marked leader \emph{Abengoa Solar} currently uses \SI{140}{\square\metre} heliostats for CR projects in SA \cite{Abengoa2014}.

The main advantage for using molten salt as HTF is the comparatively high thermal range and its good manageability storage characteristics. Molten salt in CR systems can be stored directly in two tanks as its shown in the scheme. For the storage capacity, the thermal range of the receiver temperature and the storage volume is decisive. The storage capacity is usually measured in hours of storage capacity at full load turbine output. Commercial direct molten salt thermal energy storages (TES) in CR systems currently reaches storage capacities of \SI{17.5}{\hour} \cite{NREL2015b}.

\subsubsection{Overview of PTC system technology} 
The concentration system of a PTC plant consist out of parabolic shaped mirrors which is focusing the solar beam on to a receiver tube in the focus point of the parabola. Commonly synthetic oil is used as HTF which flows through the receiver tube (also known as heat collecting element (HCE)). From the receiver, the heat up synthetic oil gets either transported to the power block or to a TES. The PTC system technology currently also uses mostly molten salt as storage fluid, but indirect due to the different used fluids of storage and solar field. Figure~\ref{troughtindirecttwotank} shows a simplified scheme of the currently most used PTC system.

\begin{figure}[htbp]  
\centering
\includegraphics[width=0.65\linewidth]{FIG/troughtindirecttwotank}
\caption[Simplified scheme of PTC power plant with indirect storage system.]{Simplified scheme of PTC power plant with indirect storage system \cite{Steinmann2012}.}\label{troughtindirecttwotank}
\end{figure}
The parabolic shaped mirrors and the HCE are typically assembled on a support structure mounted on a series of aligned pylons. The center pylon is equipped with a hydraulic drive system to allow tracking of the total solar collector assembly (SCA). Figure~\ref{SCA_EuroTrough} shows a SCA exemplary by using the EuroTrough system. The solar field of a PTC system consists of one or more parallel loops of SCAs. A common header pipe provides each loop with an equal flow rate of HTF. A second header collects the hot HTF to return it either directly to the power cycle or to the TES. \cite{Lupfert2013,Maccari2015}

\begin{figure}[htbp] 
\centering
\includegraphics[width=1\linewidth]{FIG/SCA_EuroTrough}
\caption[Typical solar collector assembly composed of 12 EuroTrough collector elements.]{Typical solar collector assembly (SCA) composed of 12 EuroTrough collector elements \cite{VonReeken2014}.}\label{SCA_EuroTrough}
\end{figure}
Current collector sizes varied from \SIrange{2.55}{8}{\metre}, but in order to reduce the total costs of PTC systems the trend for commercial scaled collector sizes is increasing. \cite{AbengoaSolar2013b,Pitz-Paal.2013,VonReeken2014}

All commercial scale PTC systems uses actually oil based HTF. The first commercial used PTC power plant \emph{SEGS-1}, built 1984 in California, uses a mineral oil as HTF in the HCE, which heats up from \SIrange{240}{305}{\celsius}. The upper temperature end for a stable use of mineral oil is around \SI{300}{\celsius}. In order to increase the maximum process temperature actual PTC systems uses synthetic oil with a maximum working temperature of \SI{400}{\celsius}. \cite{Gil2010,Richter2013,Therminol2015}

Synthetic oil is quite expansive therefore it is not used as fluid for TES aplication. Molten salt is more than ten times less expansive, that's the main reason why it is currently used as storage medium in the TES of PTC systems \cite{Gil2010}. 

There is a huge research and development expenses for using molten salt also as HTF in the HCE for PTC systems, but currently there are no commercial scale PTC power plants projects with molten salt applications as HTF known. \cite{Maccari2015}

At this stage it can be noted, that PTC uses a lower thermal process temperature than the CR systems, which effects on the one hand the efficiency of the thermal process (Rankine cycle) and on the other hand the required dimensions of the TES medium. 

Furthermore must be noted that the power output of a PTC plant is more effected by the seaseonal variation of the irradiance angle due to not traking the sun in two axis and a thereby resulting higher cosine efficinetcy loss. \cite{Jorgenson2013}

\pagebreak
\section{Large scale PV power plants}\label{Large scale photo voltaic (PV) power plants}

two axis tracking PV

\subsection{Large-scale PV power plants}

\subsection{Large-scale electrical energy storage systems}
electrical energy storage (EES)
EESSchema
\begin{figure}[htbp]  
\centering
\includegraphics[width=0.65\linewidth]{FIG/EESSchema}
\caption[Schema of an electrical energy storage (EES) system and there energy losses.]{Schema of an electrical energy storage (EES) system and there energy losses.}\label{TCC_EES}
\end{figure}

\begin{equation}
\textrm{Overall storage efficiency (AC-to-AC)} =\frac{E_{out} \textrm{ (kWh)} }{E_{in} \textrm{ (kWh)}}
\end{equation}

\begin{figure}[htbp]  
\centering
\includegraphics[width=0.75\linewidth]{FIG/TCC_EES}
\caption[Total capital cost in EUR of large-scale EES systems per unit of nominal power rating including costs of power electronics, storage part, fixed obtain and maintainence and maybe incidental replacement costs.]{Total capital cost in EUR of large-scale EES systems per unit of nominal power rating including costs of power electronics, storage part, fixed obtain and maintainence and maybe incidental replacement costs \cite{Zakeri2015}.}\label{TCC_EES}
\end{figure}
\chapter{Simulation-based comparison of solar powered power plants}
To compare CSP and PV technologies, a simulated case study was performed on the basis of a CR system and PTC system. 
%Special attention was paid in this comparison for a high run time of the system during the day and the covering of a prescribed demand curve.
Specific attention was given to maximizing daytime operation and meeting the demand curve. 
% For this comparison the PV system was expanded with an electrical energy storage.
The photovoltaic system was extended with battery storage. 
% For the simulation an battery storage was selected.
% The dimension of the stored energy in this simulation went much beyond the actual technical capacity of individual electrical storage units and reaches more than one-sixt of the actual world battery storage capacity 690~GW \cite{IEA2015}.
The storage capacity allocated for the simulation considerably exceeds the real capacity of electrical storage units currently available and is more than one-sixth the globally installed battery storage capacity of 690 GW.\cite{IEA2015}
 %Therefor must be said at this point that this is a theoretical comparison from the viewpoint of the PV, in particular for this plant and storage scale.
Due to the scale and particularly with respect to the photovoltaic plant, the comparison is theoretical in nature.
\pagebreak
\section{General assumptions}
With the aim of producing quantifiable and comparable results, the different solar supplied power plants will be simulated under different input  parameters. After that, selected comparable output parameters will be analyzed, evaluated and rated.

The plant technologies selected for comparison are: 
\begin{itemize}
\item CSP molten salt central receiver with thermal energy storage
\item CSP synthetic oil parabolic trough with thermal energy storage
\item PV fixed elevated flat plate collectors with adapted electrical energy storage
\end{itemize}
The PV plant has been extended with a lithium-ion battery storage for the simulation, while the thermal energy storage of the CSP plant uses molten salt technology.
%All power plants got laid out for a maximum power output of 100~MW$_{el}$.
All plants were laid out for a maximum power output of 100 MW$_{el}$.
% For the comparison the power plants are forced to cover a selected load scenario.
The plants are driven to match a selected load scenario.
% In order to find an individual suitable power plant design to cover the scheduled output of the scenario, different layout conditions, using various storage and collecting field sizes, was tried.
In order to find an appropriate power plant design to match the load of the scenario, different layout parameters, using various storage and collecting field sizes, were tested.
% The scenario and there goals are discussed and defined in Section~\ref{Overall simulated configuration}. The location and related weather data is defined in Section~\ref{Location and weather data}.
The scenario and requirements are defined and discussed in Section~\ref{Overall simulated configuration}. The location and related weather data are described in Section~\ref{Location and weather data}.
%The solar power plants are implemented and simulated in NREL’s System Advisor Model (SAM) version SAM 2015.6.30 r3 for OS X \cite{NREL2015}.
The solar power plants are implemented and simulated in NREL’s System Advisor Model (SAM) version SAM 2015.6.30 r3 for OS X \cite{NREL2015}. 
% SAM is designed to simulate performance and financial models of different types of renewable energies. For the simulation the performance part of the software was used. The financial analysis was made separately. 
SAM is designed to simulate performance and financial models of different types of renewable energy. For this simulation, only the performance component was used. The financial analysis was done separately. 

The financial parameters and the resulting levelized cost of electricity (LCOE) are calculated separately for all power plants in Microsoft Excel 2011 (vers. 14.5.7) for Mac, using a simplified method which is documented in Appendix~\ref{ChapterLCOE} on page \pageref{ChapterLCOE} using a lifetime of 25~years for each plant.

\subsection{Simulation scenario} \label{Overall simulated configuration}
%Power plants are forced to supply the system load/demand.
A hard requirement is that plants meet the full electrical demand in the system. Figure~\ref{LoadScenarios} shows the daily average system load/demand in South Africa for the winter and summer period.
% The profiles from both load shapes rises at approx 7:00 in the morning and has there peak demand at 20:00 during the summer period.
Both profiles feature a sharp rise at approx 7:00 and have their peak demand at 20:00 during the summer period.
In winter, there is an initial peak at 9:00 and a second at 19:00.
%In order to supply this system load the simulated solar power plants are forced to generate full power output of 100~MW from 7:00 to 22:00. When the system demand comes down during the night also the power plants reduce there output from 22:00 to 7:00 to 50~MW. The scenario is called "night-reduction".
In order to match this system load, the simulated solar power plants must operate at full power output of 100~MW from 7:00 to 22:00. When the system demand drops during the night, the plants reduce there output from 22:00 to 7:00 to 50~MW. The scenario is called "night reduction".
\begin{figure}[htbp]  
\centering
\includegraphics[width=1\linewidth]{FIG/LoadScenarios}
\caption[South Africa daily average system load/demand for summer and winter days, with scheduled power production curve.]{South Africa daily average system load/demand for summer and winter days, with scheduled power production curve.}\label{LoadScenarios}
\end{figure}
%For the comparison the power plants there system design is forced to cover 90~\% of the scheduled electricity production over the first year.  Usually, in feed-in contracts the power output is fixed at specified values. The overproduction is not content of these contracts and is not remunerated. In order to generate exploitable and comparable results and also considering the common feed-in contracts for power plants, the hourly power production is cut to the planed values. So overproduction will not considered in the evaluation and analyses of the systems.
In this comparison, the system design must handle 90~\% of the scheduled electricity production over the first year. Usually, in feed-in contracts, the deliverable energy is fixed. Any overproduction is not covered by such contracts and is not remunerated. In order to generate exploitable and comparable results and considering standard feed-in contracts, the hourly power production is cut to the planned values and thus overproduction is not considered in this analysis.

%The results of the simulated scenarios and the financial analyses will be rated in two selected categories:
The results of the simulation and the financial analyses will be evaluated by the following quantitative measures:
\begin{itemize}
%\item \textbf{Load curve covering factor} [\%]: The Load curve covering factor (LCCF) describes quantitative how effective the power plant follows the required load curve of the scenarios.
\item \textbf{Load curve covering factor} [\%]: The load curve covering factor (LCCF) is an effectiveness measure describing how closely the plant follows the load curve.
%\item \textbf{Levelized cost of electricity} [\textcent /kWh]: The levelized cost of electricity (LCOE) represents the total project lifecycle costs. It is the present value of project costs expressed in cents per kilowatt-hour of electricity generated by the system over its life. \cite{NREL2015a}
\item \textbf{Levelized cost of electricity} [\textcent /kWh]: The levelized cost of electricity (LCOE) represents the total project life-cycle costs. It is the present value of project costs expressed in cents per kilowatt-hour of electricity generated by the system over its life. \cite{NREL2015a}
\end{itemize}
%There is a huge difference between the mentioned LCCF and the widespread capacity factor (CF). The CF is the ratio of the system's predicted electrical output in the first year of operation to the nameplate output, which is equivalent to the quantity of energy the system would generate if it operated at its nameplate capacity for every hour of the year \cite{NREL2015a}. As it is mentioned above is the LCCF calculated from the sum value of load covering in each hour of the year.
There is an important difference between the LCCF and the widely-used \emph{capacity factor} (CF). The CF is the ratio of the system's electrical output in the first year of operation to the nameplate output, which is equivalent to the quantity of energy the system would generate if it operated at its nameplate capacity for every hour of the year \cite{NREL2015a}. The LCCF is calculated from the sum value of load coverage in each hour of the year.

\subsection{Location and weather data} \label{Location and weather data}
%For the simulation of the solar power plants the locations and weather parameter of Upington, Northern Cape was used. This location is situated in a region with one of the highest irradiation values of the country, but also has a good water access by the Orange River. This locations was mentioned before in Chapter~\ref{Solar power in South Africa} and is marked in the GHI- and DNI-maps of SA in Figure \ref{irradiation} on Page \pageref{irradiation}.
In this simulation, locations and weather data for Upington, Northern Cape were used. This region has among the highest irradiation values in the country, but also has good water access due to the Orange River. This location was mentioned before in Chapter~\ref{Solar power in South Africa} and is marked in the GHI- and DNI-maps of SA in Figure \ref{irradiation} on Page \pageref{irradiation}. 
 
\begin{table}[!h]  
  \centering
	\begin{tabular}{  p{4.0cm}  C{4.0cm}  C{3.0cm} } 

	\hline	
\textbf{Item}  & \textbf{Value} & \textbf{Unit} \\ \hline \hline
Location & Upington & -\\ 
Station ID &  684240& -  \\ 
Data source & White Box Technologies, Inc. (31.05.2015) & -\\ \hline
Latitude & -28.40 &$\,^{\circ}$N \\ 
Longitude &  21.27 &$\,^{\circ}$E \\ 
Elevation &  836 & m \\ 
Total GHI per year  &  2~280 & kWh/m\textsuperscript{2}\\ 
Total DNI per year &  2~621 & kWh/m\textsuperscript{2}\\ 
Total DHI per year &  516 & kWh/m\textsuperscript{2}\\ 
Mean temp. &  21 & $\,^{\circ}\mathrm{C}$\\ 
Mean wind speed & 3.3 & m/s\\ \hline
\end{tabular}
\caption[Location and characteristics for the simulation in SAM.]{Location and characteristics for the simulation in SAM.}\label{tbl: Location}
\end{table}


%The input weather data for the simulation with SAM are in the EPW-format (EnergyPlus Weather Data) and produced by White Box Technologies, Inc. \cite{WhiteBoxTechnologies2015}. The EPW-files are data sets of hourly values of solar radiation and meteorological elements for a typical one-year period. This includes air temperature, dew point temperature, relative Humidity, atmospheric pressure, global horizontal solar radiation, diffuse Horizontal solar radiation, direct normal radiation, wind Speed, wind Direction and cloud cover. The for the simulation most relevant values are summarized in Table \ref{tbl: Location}. The hourly values of global horizontal and direct normal irradiance from the EPW-file is shown in Figure~\ref{Upington_GHI/DNI}. The highest irradiance value during the summer time is 1~199~Wh/m\textsuperscript{2} for GHI and 1~154~Wh/m\textsuperscript{2} for DNI. At the shortest day the highest irradiance is 625~Wh/m\textsuperscript{2} for GHI and 820~Wh/m\textsuperscript{2} for DNI.
The weather data for the simulation in SAM are in EPW format (EnergyPlus Weather Data) and produced by White Box Technologies, Inc. \cite{WhiteBoxTechnologies2015}. The EPW files are data sets of hourly values of solar radiation and meteorological elements for a typical one-year period. These include air temperature, dew point temperature, relative humidity, atmospheric pressure, global horizontal solar radiation, diffuse horizontal solar radiation, direct normal radiation, wind speed, wind direction and cloud cover. The most relevant parameters for this simulation are summarized in Table \ref{tbl: Location}. The hourly values of global horizontal and direct normal irradiance from the EPW file are shown in Figure~\ref{Upington_GHI/DNI}. The highest irradiance value during the summer is 1~199~Wh/m\textsuperscript{2} for GHI and 1~154~Wh/m\textsuperscript{2} for DNI. At the northern solstice, the highest irradiance is 625~Wh/m\textsuperscript{2} for GHI and 820~Wh/m\textsuperscript{2} for DNI.


\begin{figure}[!htbp]
        \centering
                \begin{subfigure}[b]{1\textwidth}
                \centering
                \includegraphics[width=1\textwidth]{FIG/Upington_GHI}
                \caption{Global horizontal}\label{Upington_GHI}
        \end{subfigure}%
\par\medskip % Linebreak      
        \begin{subfigure}[b]{1\textwidth}
                \centering
                \includegraphics[width=1\textwidth]{FIG/Upington_DNI}
                \caption{Direct normal }\label{Upington_DNI}
        \end{subfigure}%

        \caption[Hourly values of irradiance over a full year from Upington used in the simulation.]{Hourly values of irradiance over a full year from Upington used in the simulation.}\label{Upington_GHI/DNI}
\end{figure}
\newpage \noindent
%Most relevant for the simulation is also the position of the sun. SAM calculates the path of the sun by the position of longitude and latitude. Figure~\ref{SunPathUpington} shows the sun path diagram of Upington. From this it appears that the longest day in Upington has 13~h and 56~minutes with a maximum sun height of 85.05$\,^{\circ}$ while the shortest day has just 10~h and 19~minutes and a maximum sun height of 35.93$\,^{\circ}$.
SAM calculates the path of the sun by the position of longitude and latitude. Figure~\ref{SunPathUpington} shows the solar path diagram at Upington. It is apparent that the longest day in Upington has a duration of 13~h and 56~minutes with a maximum sun height of 85.05$\,^{\circ}$ while the shortest day has a duration of just 10~h and 19~minutes and a maximum sun height of 35.93$\,^{\circ}$.

\begin{figure}[htbp]  
\centering
\includegraphics[width=0.95\linewidth]{FIG/SunPathUpington}
\caption[Solar path diagram for Upington.]{Solar path diagram Upington \cite{PVsystSA2015}.}\label{SunPathUpington}
\end{figure}
\pagebreak 
\section{CR power plant}
\subsection{Design  and simulation} \label{CR power plant design  and simulation}
%For the CR power plant simulation in SAM the “CSP power tower molten salt" model was used. The EPW weather file for Upington from Section~\ref{Location and weather data} was used as an input file to specify the hourly atmospheric conditions. SAM uses the following input data for the simulation:
For the CR plant simulation in SAM, the \enquote{CSP power tower molten salt} model was used. The EPW weather file for Upington from Section~\ref{Location and weather data} was used as an input file to specify the hourly atmospheric conditions. SAM uses the following input data for the simulation:
\begin{itemize}
\item Latitude ($\,^{\circ}$)
\item Longitude ($\,^{\circ}$)
\item Elevation above sea level (m)
\item DNI (W/m\textsuperscript{2})
\item Atmospheric pressure (mbar)
\item Dry bulb temperature ($\,^{\circ}\mathrm{C}$)
\item Wet bulb temperature ($\,^{\circ}\mathrm{C}$)
\item Relative humidity (\%)
\item Wind velocity (m/s)
\end{itemize}
This Chapter describes in detail the crucial inputs of the CR power plant components, namely the power cycle, heliostat field, tower and receiver and thermal energy storage (TES).

As already mentioned, financial parameters and the LCOE are calculated separately in Microsoft Excel using a simplified method which is documented in Appendix~\ref{ChapterLCOE} on Page \pageref{ChapterLCOE}.

\subsubsection{Simulated configurations}
%The simulated configurations had the goal to reach 90~\% of the scheduled production curve by using variation of solar multiple and full load hours of TES. Also the simulated configurations covers a broad view on the technology possibilities. As already mentioned in Section~\ref{Large scale concentrated solar power (CSP) plants} is the SM the ratio of the receivers thermal output to the power cycles thermal input at design point. So a CR system with a SM of 1 has a receiver and a heliostat field which provides the thermal power needed for the power block to run at full load at the system design point. A receiver and collector field with a SM of 1 don't produce enough thermal power to store energy in a TES while feeding the turbine. A CR system with a SM of 1 is just suitable for systems without TES. To covering the scheduled load the solar multiple was varied from 2 to 3.5 in steps of 0.5. Also the storage full load hours were varied from 8 to 16~h in steps of 2~h. The target of 100~MW net capacity was reached with a gross capacity of 111~MW with an estimated gross-to-net conversion factor of 0.90. Table~\ref{tbl: CR_OverallConfig} summarizes the simulated configurations.
The simulated configurations were to meet 90~\% of the scheduled production curve by using variation of solar multiples and full load hours of TES. As mentioned in Section~\ref{Large scale concentrated solar power (CSP) plants} the solar multiple (SM) is the ratio of the receiver's thermal output to the power cycles thermal input at the design point. Thus, a CR system with a SM of 1 has a receiver and a heliostat field which provides the thermal power needed for the power block to run at full load at the system design point. A receiver and collector field with an SM of 1 do not produce enough thermal power to store energy in a TES while also feeding the turbine. A CR system with a SM of 1 is just sufficient for systems without TES. In order to cover the expected load, the solar multiple was varied from 2 to 3.5 in steps of 0.5, while storage full load hours were varied from 8 to 16~h in steps of 2~h. The target of \SI{100}{MW} net capacity was reached with a gross capacity of 111~MW with an estimated gross-to-net conversion factor of 0.90. Table~\ref{tbl: CR_OverallConfig} summarizes the simulated configurations.

\begin{table}[!h]  
  \centering
	\begin{tabular}{ p{4.0cm}  C{1.0cm} C{0.3cm} C{0.3cm} C{0.3cm} C{0.3cm} C{0.3cm} | C{0.3cm} C{0.3cm} C{0.3cm} C{0.3cm} C{0.3cm} } 
	\hline	
\textbf{Item} & \textbf{Unit} & \multicolumn{10}{c}{\textbf{Value}} \\ \hline \hline
Net turbine capacity & MW\textsubscript{el} & \multicolumn{10}{c}{100} \\
Gross turbine capacity & MW\textsubscript{el} & \multicolumn{10}{c}{111} \\ \hline
Solar multiple & - & \multicolumn{5}{c}{2.0} & \multicolumn{5}{c}{2.5} \\
TES capacity & h &  8 & 10 & 12 & 14 & 16 &  8 & 10 & 12 & 14 & 16 \\ \hline 
Solar multiple & - & \multicolumn{5}{c}{3.0} & \multicolumn{5}{c}{3.5} \\
TES capacity & h &   8 & 10 & 12 & 14 & 16 &  8 & 10 & 12 & 14 & 16 \\ \hline 
\end{tabular}
\caption[Simulated CR solar multiple and thermal energy storage configurations.]{Simulated CR solar multiple and thermal energy storage  configurations.}\label{tbl: CR_OverallConfig}
\end{table}
\subsubsection{Power cycle}
The power cycle of the simulated CR system features a Rankine-cycle steam engine, two open feed-water heaters, a pre-heater, boiler and super-heater \cite{NREL2015a}. As mentioned above has the turbine a gross capacity of 111~MW\textsubscript{el} and a nameplate (net) capacity of 100 MW\textsubscript{el}. 
\begin{table}[!h]  
  \centering
	\begin{tabular}{  p{7.0cm}  C{2.0cm}  C{2.0cm} } 
	\hline	
\textbf{Item} & \textbf{Value} & \textbf{Unit} \\ \hline \hline
Turbine design capacity, gross  & 111 & MW\textsubscript{el} \\ 
Turbine design capacity, net & 100 & MW\textsubscript{el} \\ 
Boiler operating pressure & 125 & bar \\ 
Design inlet temperature & 288 & $\,^{\circ}\mathrm{C}$ \\ 
Design outlet temperature & 566 & $\,^{\circ}\mathrm{C}$ \\ 
Cycle conversion efficiency & 41.2 & \% \\ 
Steam generator design thermal power & 269.42 & MW\textsubscript{th}  \\
Power block start-up time & 0.5 & h \\ 
Minimum required start-up temperature & 500 & $\,^{\circ}\mathrm{C}$ \\
Plant availability  & 96 & \\
Condenser type & air-cooled & - \\ 
\hline
\end{tabular}
\caption[CR power block and condecer input parameter in SAM.]{CR power block and condecer input parameter in SAM.}\label{tbl: CRPowerplant}
\end{table}
The steam generator has a HTF inlet temperature of 566$\,^{\circ}\mathrm{C}$ and outlet temperature of 288$\,^{\circ}\mathrm{C}$ at design point and operates at a pressure of 125 bar. In combination with a air-cooled condenser the CR power cycle system was simulated with a cycle gross efficiency of 41.2\%. A wet-cooled condenser would reaches some higher efficiency, but because of the lack of water in the area of Upington and the requirement by the South African government for CSP plants, a air-cooled condenser was selected. The HTF inlet temperature, pressure and efficiency values where adapted for the configuration from \cite{Kolb2011a}. For starting up the system 
needs 30 minutes and a min. required temperature of 500$\,^{\circ}\mathrm{C}$. A plant availability of 96~\% was adapted from \cite{Morin2012} in order to simulate system down-times for outages or scheduled maintenance.
\subsubsection{Heliostat field}
The heliostat field design was done in SAM using its heliostat field layout optimization tool. For the design, optimization and simulation process, heliostat data from the Sanlúcar 120 heliostat was used \cite{Noone2012}. The Sanlúcar 120 is used in the Planta Solar 10 (PS10) near Seville, Spain and is the origin of Abengoas actual heliostat ASUP 140. For a simulation with the ASUP 140 no sufficient data was available. 
\begin{table}[!htbp]  
  \centering
	\begin{tabular}{ p{4.5cm}  C{1.5cm} C{1.2cm} C{1.2cm} C{1.2cm} C{1.2cm} } 
	\hline	
\textbf{Item} & \textbf{Unit} & \multicolumn{4}{c}{\textbf{Value}} \\ \hline \hline
racking & - &  \multicolumn{4}{c}{two-axis} \\
Height & m  &  \multicolumn{4}{c}{9.45} \\
Width & m  &  \multicolumn{4}{c}{12.84} \\
Reflective area & m\textsuperscript{2} &  \multicolumn{4}{c}{120.00} \\
Heliostat availability& \% &  \multicolumn{4}{c}{99} \\
Design-point DNI & W/m\textsuperscript{2} &  \multicolumn{4}{c}{950} \\
\hline
\textbf{Solar multiple}& \textbf{-} & \textbf{2.0} & \textbf{2.5} & \textbf{3.0} & \textbf{3.5}\\ \hline 
Number of heliostats & - & 9~131 & 11~530 & 13~976 & 16~658 \\
Max. distance from tower & m & 1~528 & 1~708 & 1~883 & 2010 \\
Reflective area  & ha & 109.58 & 138.36 & 167.72 & 199.90 \\
Field land area & ha & 655.59 & 821.92 & 997.95 & 1~190.99 \\ 
\hline
\end{tabular}
\caption[CR heliostat parameter.]{CR heliostat parameter.}\label{tbl: CRHeliostats}
\end{table}
The Sanlúcar 120 heliostats have an effective reflective area of 120~m\textsuperscript{2} and are tracked on 2 axes. The heliostat field layout optimization and there design depends on the turbine gross and the SM. The heliostat field was optimized therefore for all four considered configurations. SAM first generates a coarse layout and optimizes the number of heliostats and their position. The goal of the optimization is the maximum flux with minimum power constraints according to the specified design-point DNI which represents the DNI at which the power plant achieves its rated thermal capacity. Table~\ref{tbl: CRHeliostats} summarizes the heliostat datas and the values of the optimized field design for all considered solar multiples.



The optimized heliostat field layouts of the four considered SM are depict in Figure~\ref{SM}. In order to avoid mutual shading from the heliostats the heliostat density decreases with distance from the tower. The maximum heliostat distance to the tower rises with the higher SM which can be seen in the field diameter. 
\begin{figure}[!htbp]
        \centering   
        \begin{subfigure}[b]{0.5\textwidth}
                \centering
                \includegraphics[width=0.95\textwidth]{FIG/SM20}
                \caption{SM:~2.0}\label{SM2.0}
        \end{subfigure}%
        ~
        \begin{subfigure}[b]{0.5\textwidth}
                \centering
                \includegraphics[width=0.95\textwidth]{FIG/SM25}
                \caption{SM:~2.5}\label{SM2.5}
        \end{subfigure}
        
\par\medskip % Linebreak
                
        \begin{subfigure}[b]{0.5\textwidth}
                \centering
                \includegraphics[width=0.95\textwidth]{FIG/SM30}
                \caption{SM:~3.0}\label{SM3.0}
        \end{subfigure}%
        ~
        \begin{subfigure}[b]{0.5\textwidth}
                \centering
                \includegraphics[width=0.95\textwidth]{FIG/SM35}
                \caption{SM:~3.5}\label{SM3.5}
        \end{subfigure}
        \caption[Simulated heliostat field layout at diferent solar multiples (SM).]{Simulated heliostat field layout at diferent solar multiples (SM).}\label{SM}
\end{figure}
\subsubsection{Tower and receiver}
The receiver collects the concentrated irradiance from the surrounding heliostat field. The simulated receiver is build as an external cylindrical receiver and is configurated with 24 panels of thin walled (1.25 mm) receiver tubes with an outer diameter of 60 mm arranged in a circle around the tower. The receiver tubes are made from 316H stainless steel and the external surfaces of the tubes are coated with a black Pyromark paint. The paint is resistant to high temperatures and thermal cycling, and absorbed 95\% of the incident sunlight. This configuration is similar to the receiver of the  Solar Two Project accept of the outer diameter \cite{Bradshaw2002}. The panels containing molten salt (60~\% NaNO\textsubscript{3} and 40~\% KNO\textsubscript{3}) as heat transfer fluid (HTF). The receiver design inlet temperature of 288$\,^{\circ}\mathrm{C}$ and the outlet temperature of 566$\,^{\circ}\mathrm{C}$ was set before in the power block configuration. In order to achieve the desired receiver thermal power output, the height of the tower and receiver dimensions got also adjust with the optimization of the heliostat field design. The results of the optimization is shown in Table~\ref{tbl: CRSolarfield}, as well as the fixed tower and receiver parameter. The heights of the tower range from 179.77 to 236.50~m. The receiver proportions rises with the SM and the field sizes and results in a receiver thermal power range from 538.8 to 943.0 MW\textsubscript{th}.
\begin{table}[!h]  
  \centering
	\begin{tabular}{ p{5.5cm}  C{1.5cm} C{1.2cm} C{1.4cm} C{1.4cm} C{1.4cm} } 
	\hline	
\textbf{Item} & \textbf{Unit} & \multicolumn{4}{c}{\textbf{Value}} \\ \hline \hline
Receiver configuration & - &  \multicolumn{4}{c}{external cylindrical receiver}\\
Heat transfer fluid & - &  \multicolumn{4}{c}{60~\% NaNO\textsubscript{3} and 40~\% KNO\textsubscript{3}}\\
Inlet temperatures & $\,^{\circ}\mathrm{C}$  &  \multicolumn{4}{c}{288}\\
Outlet temperatures & $\,^{\circ}\mathrm{C}$  &  \multicolumn{4}{c}{566}\\
Receiver tube material & - &  \multicolumn{4}{c}{AISI316 stainless steal}\\
Receiver tube outer diameter & mm &  \multicolumn{4}{c}{60}\\
Receiver tube wall thickness & mm &  \multicolumn{4}{c}{1.25}\\
Number of panels & - &  \multicolumn{4}{c}{24}\\
Absorption factor  & - &  \multicolumn{4}{c}{0.95}\\
\hline
\textbf{Solar multiple} &  & \textbf{2.0} & \textbf{2.5} & \textbf{3.0} & \textbf{3.5}\\ \hline 
Tower height & m & 179.77 & 200.96 & 221.57 &  236.50\\
Receiver height  & m & 22.55 & 25.45 & 27.74 &  29.94\\
Receiver diameter & m & 14.85 & 16.66 & 17.91 & 18.77\\ 
Receiver aperture area & m\textsuperscript{2} & 1~052 & 1~332 & 1~561 & 1~765 \\ 
Receiver thermal power & MW\textsubscript{th} & 538.8 & 673.5 & 808.3 & 943.0 \\
\hline
\end{tabular}
\caption[CR heliostat field parameter.]{CR heliostat field parameter.}\label{tbl: CRSolarfield}
\end{table}
\subsubsection{Thermal energy storage (TES)}
The CR system was simulated using a direct two-tank molten salt thermal energy storage (TES). The storage uses directly the HTF (60~\% NaNO\textsubscript{3} and 40~\% KNO\textsubscript{3}) from the receiver as storage medium. Figure~\ref{towerdirecttwotank} on Page~\pageref{towerdirecttwotank} shows a schema of the direct storage for CR system. The design temperature of the hot storage tank (566$\,^{\circ}\mathrm{C}$) and the cold storage tank (288$\,^{\circ}\mathrm{C}$) depends also from the set value in the power block settings as before the design receiver temperatures. So the design temperature difference between the tanks is 278$\,^{\circ}\mathrm{C}$.



As defined at the begining of this Chapter, the full load hours of TES are varied in steps of 2~h from 8 to 16~h. The TES full load hours represents the time in which the storage can supply enough energy to the steam turbine and the power block to run at full design capacity. So a higher value of TES full load hours extended the time that the power plant can run during nights or cloudy days. Table~\ref{tbl: CRTES} shows that with higher TES full load hours the thermal capacity and tank volume increases as well. The simulated storage capacity is in a range of 2~155~MWh\textsubscript{th} using 8 full load hours and 4~311~MWh\textsubscript{th} using 16 full load hours. 
\begin{table}[!h]  
  \centering
	\begin{tabular}{ p{3.9cm}  C{1.0cm} C{1.2cm} C{1.2cm} C{1.2cm} C{1.2cm} C{1.2cm} } 
	\hline	
\textbf{Item} & \textbf{Unit} & \multicolumn{5}{c}{\textbf{Value}} \\ \hline \hline
Storage type & - &  \multicolumn{5}{c}{direct two-tank molten salt}\\
Storage fluid & - &  \multicolumn{5}{c}{60~\% NaNO\textsubscript{3} and 40~\% KNO\textsubscript{3}}\\
Hot tank design temp. & $\,^{\circ}\mathrm{C}$ & \multicolumn{5}{c}{566}\\
Cold tank design temp. & $\,^{\circ}\mathrm{C}$ & \multicolumn{5}{c}{288}\\
\hline
\textbf{TES full load hours} & \textbf{h} & \textbf{8} & \textbf{10} & \textbf{12} & \textbf{14} & \textbf{16}\\ \hline 
Thermal capacity & MWh\textsubscript{th} & 2~155 & 2~694 & 3~233 & 3~772 &  4~311\\
Storage volume  & m\textsuperscript{3} & 10~229 & 12~787 & 15~345 & 17~902 & 20~460\\
\hline
\end{tabular}
\caption[CR system TES parameter.]{CR system TES parameter.}\label{tbl: CRTES}
\end{table}

SAM 2015.6.30 r3 don't have a power control to follow prescribed hourly net output values. The turbine output of each day needs to be scheduled in front for the simulation. Thereby the parasitic consumer needs highly attention. Figure~\ref{CR_turbineoutput} shows the for the simulation scheduled turbine output matrix. The individual values was fixed during experimental trial. It can be seen that the turbine output is reduced to 50\% during the night as it is specified in the scenario. The power plant follows the schedule till there is no solar irradiance and no more thermal capacity in the storage left to generate power. 
\begin{figure}[htbp]  
\centering
\includegraphics[width=0.95\linewidth]{FIG/CR_turbineoutput}
\caption[TES dispatch control matrix for turbine output fraction of CR simulation in SAM.]{TES dispatch control matrix for turbine output fraction of CR simulation in SAM.}\label{CR_turbineoutput}
\end{figure}
\pagebreak
\subsubsection{Financial parameter}
The financial parameter for the LCOE calculation of the CR power plant are seperated in different specific cost parts and are sumerized in Table~\ref{tbl: CRFinance}. The method for calculating the LCOE is documented in Appendix~\ref{ChapterLCOE} on Page \pageref{ChapterLCOE} using a lifetime of 25~years and a real interest rate of 7.5~\% \cite{FraunhoferISE2013}. The interest rate could be reduced with a rising market potential.



The specific costs of the heliostat field of 180\$/m\textsuperscript{2} coming from J. B. Blackmon \cite{Blackmon2012}. He analyzed the costs of heliostat as a function of area for a representative solar central receiver power plant and calculated the total installed costs for a field of 5~000 heliostats with 148~m\textsuperscript{2} reflective area of approximately 180~\$/m\textsuperscript{2}. The analyze results showed that the costs are increasing with the size of the heliostats from 40~m\textsuperscript{2} on. So the adopted value of 180~\$/m\textsuperscript{2} reflective area for 140~m\textsuperscript{2} heliostats can be assumed as conservative.

The remaining investment costs are based on the "Power Tower Technology Roadmap and Cost Reduction Plan" \cite{Kolb2011} from 2011.

On top of the investment costs a came 15 \% once-off surcharge for EPC and contingencies \cite{Platzer2014}.

Fichtner analyzed the annual O\&M costs with 1.84-1.96~\% of the total investment costs for CR power plants in SA \cite{Fichtner2010}. The value of 2~\% can so also be assumed as conservative.

The costs for the land purchase comes from the "African Agriculture Review" report of the Nedbank Capital, which reported the prices for farmland in SA. For the LCOE calculation of the CSP and PV system was the land purchase costs of \$3,000/ha assumed which is based on these report \cite{Cassell2012}.

\begin{table}[!h]  
  \centering
	\begin{tabular}{  p{5.0cm} C{2.0cm} C{1.5cm}  C{1.5cm}  C{4.0cm} } 
	\hline	
\textbf{Item} & \textbf{Symbol}& \textbf{Value} & \textbf{Unit} & \textbf{Source}\\ \hline \hline
Heliostat field &$c_{HF}$ & 180 & \$/m\textsuperscript{2} & \cite{Blackmon2012}\\ 
Power block & $c_{PB,CR}$ & 1000 & \$/kW\textsubscript{el} & \cite{Kolb2011}\\ 
Thermal energy storage&$c_{TES,CR}$ & 30 & \$/kWh\textsubscript{th}  & \cite{Kolb2011}\\ 
Tower and receiver& $c_{T+R}$& 200 & \$/kW\textsubscript{th}  & \cite{Kolb2011}\\ 
Annual O\&M & $f_{O\&M,CR}$ & 2 & \% &\cite{Fichtner2010}\\
Land purchase&$c_{LP}$ & 3~000 & \$/ha & \cite{Cassell2012}\\ \hline
Lifetime &$n$ & 25 & years & \cite{FraunhoferISE2013} \\ 
Interest rate & $i_{CR}$ & 7.5 & \% & \cite{FraunhoferISE2013} \\ 
Surcharge for EPC, project management and risk & $f_{EPC,CR}$& 15 & \% & \cite{Platzer2014} \\
Total plant availability & $f_{avail,plant,CR}$ & 96 & \% & \cite{Morin2012} \\ 
\hline
\end{tabular}
\caption[Finacial input parameter for CR-simulation in SAM.]{Finacial input parameter for CR-simulation in SAM.}\label{tbl: CRFinance}
\end{table}
\subsection{Results of CR power plant simulation}
The following sections illustrates and discuss the obtained simulation and calculation results of the above defined CR power plant. Therefore the load curve covering performance and LCOE as well as the belonging load profiles and duration curves are described and analyzed. The configuration of the power plants to reach 90~\% of the prescribed load while using the best belonging LCOE is defined at the end of each section.
\subsubsection{Load curve covering}
To find the suitable power plant configuration of the CR system to reach the target of 90~\% load curve covering, there was 20 various configurations simulated. This section presets and compare mainly the results of the simulation with the lowest SM and hours of TES configuration (SM: 2.0 \& TES: 8~h) with the simulation using the highest SM and hours of TES configuration (SM: 3.5 \& TES: 16~h) representative for all in between. All simulated configurations are shown in Table~\ref{tbl: CR_OverallConfig} on Page~\pageref{tbl: CR_OverallConfig}.

\begin{figure}[htbp]  
\centering
\includegraphics[width=0.8\linewidth]{FIG/CR_annual_profil}
\caption[Annual average load profile of selected CR power plant configurations.]{Annual average load profile of selected CR power plant configurations.}\label{CR_annual_profil}
\end{figure}
Figure~\ref{CR_annual_profil} shows the annual load curve covering of the above mentioned CR power plant high and low configurations with the prescribed load curve. The Figure shows, that the CR power plant using a SM of 3.5 and 16~h of TES for the configuration can cover almost at any time of the year the prescribed load curve. The decreasing of the supplied electricity 8~am leads from the winter duration and will be shown in the following. The lower CR power plant configuration can just cover between 11~am and 5~pm almost the same amount electricity during the year than the high configuration. In the morning times between 6 and 7~am the energy production is coming to standstill the whole year.

At this point it might be necessary to remind, that it is just the electricity used for the annual load covering as well as for the LCOE calculation which is actually supplied at the requested time step. So the electricity which is overproduced at any time of the year is not considered. This can also be seen in Figure~\ref{CR_winter_load}. 
\begin{figure}[htbp]  
\centering
\includegraphics[width=1\linewidth]{FIG/CR_winter_load}
\caption[CR load profile during the time of winter solstice (15. June - 25. June).]{CR load profile during the time of winter solstice (15. June - 25. June).}\label{CR_winter_load}
\end{figure}
This figure shows exemplary the load curve behavior of the CR power plants during the time with the least hours of sunlight during a day of the year. The DNI is around 830~W/m\textsuperscript{2} in peak times at these days and there are also days with less or almost no direct irradiance. As mentioned, it can be seen that the net electricity production is in both cases at some times across the prescribed load curve and is not consulted further. This reaches mainly from the described dispatch control matrix from Figure~\ref{CR_turbineoutput} on Page~\pageref{CR_turbineoutput} to control the turbine output. This is not accurate at all to follow a specific load exactly during each day of the year. 

However, Figure~\ref{CR_winter_load} mainly shows the load curve behavior of the CR power plants in low and high configuration. At a SM of 2.0 with 8~h of TES the power plant can't produce enough power during the day to fill up the TES for generating power during the night. Compared to that, the CR power plant with a SM of 3.5 and 16~h of TES can almost anytime generate electricity over the whole night. But with the rising load in the morning hours the energy production stops due to the empty storage. 
\begin{figure}[htbp]  
\centering
\includegraphics[width=1\linewidth]{FIG/CR_summer_load}
\caption[CR load profile during the time of summer solstice (16. December - 26. December).]{CR load profile during the time of summer solstice (16. December - 26. December).}\label{CR_summer_load}
\end{figure}
This leads also to the above mentioned decreasing in supply at 8~am in the annual average load profile. The difference between the gross and net electricity production is coming from the parasitic consume within the power plant.

Figure~\ref{CR_summer_load} describes the load behavior of the  same power plants during the longest days of the year. It is clearly to see, that the irradiation is higher and longer during the days compare to the winter solstice. The peaks of DNI are about 1~150~W/m\textsuperscript{2} during that days. At a SM of 2.0 and 8~h of TES the CR power plant is not able to follow the prescribed electricity load over the whole day. It's coming to standstill between 6 and 7~am at least. Therefrom also the annual average load profile is coming to stand still at these time of the day. The CR power plant with a SM of 3.5 and 16~h of TES can follow the prescribed load mostly without any problems. Outstanding for the high productivity of the plant configuration is the electricity output while having low direct irradiance. At the 21. of December the plant can supply almost the whole prescribed load using reserves from the TES.

A detailed look to the parasitic behaviors of the CR power plant with a SM of 3.5 and 16~h of TES reveals a strong decreasing after a moderate increasing during the midday. This happens when the storage is full loaded and the steam turbine don't need more power. At this point a specified amount of heliostats defocusing the receiver and the power of the receivers HTF pump is getting reduced. 

\begin{figure}[!htbp]
        \centering   
        \begin{subfigure}[b]{0.65\textwidth}
                \centering
                \includegraphics[width=1\textwidth]{FIG/CR_parasitics_low}
                \caption{SM: 2.0 TES: 8 h}\label{CR_parasitics_low}
        \end{subfigure}
\par\medskip % Linebreak              
        \begin{subfigure}[b]{0.65\textwidth}
                \centering
                \includegraphics[width=1\textwidth]{FIG/CR_parasitics_high}
                \caption{SM: 3.5 TES: 16 h}\label{CR_parasitics_high}
        \end{subfigure}
        \caption[Share of annual gross energy output of selected CR power plant configurations.]{Share of annual gross energy output of selected CR power plant configurations.}\label{CR_parasitics}
\end{figure}
The parasitic load is composed from various electrical loads of the power plant, namely pumping loads, cooling loads, fixed loads and loads depending from the power generation. The share of these electrical parasitic loads and the total share of the gross energy output of the selected CR power plans can be seen in Figure~\ref{CR_parasitics}. The share of the net energy production out of the gross energy production is about 90~\%. The unused net energy output share of all simulated CR power plants is between 1.6 and 3.2~\% and went not into the LCOE calculation.

Figure~\ref{CR_parasitics_low} shows, that the net outputs share of the lowest CR power plant configuration is about 483.60~GWh.  The total sum of the annual prescribed load is 711.75~GWh. So the power plant covers the prescribed load for 67.9~\%. The net outputs share of the highest CR power plant configuration from Figure~\ref{CR_parasitics_high} is 671.69~GWh and is covering the prescribed load for 94.4~\% over the year.

Figure~\ref{CR_LCCF} summarizes the results of the load curve covering for all simulated CR configurations. The chart shows that at a SM of 2.0 all TES variations besides 8~h a covering more than 70~\% of the prescribed load. At a SM of 2.5 the and a TES about 10~h the results showed over 80~\% load covering. A load curve covering of about 90~\% is reached at a SM of 3.0 and more than 14~h of TES. The configuration with a 12~h TES reaches just 89.1~\% at a SM of 3.0. 

\begin{figure}[htbp]  
\centering
\includegraphics[width=1\linewidth]{FIG/CR_LCCF}
\caption[Load curve covering result of simulated CR systems.]{Load curve covering result of simulated CR systems.}\label{CR_LCCF}
\end{figure}
The results of the load curve covering shows that there is no appreciable difference between the 12, 14 and 16~h of TES. At a SM of 3.5 the difference is just 1.4~\% load curve covering between 12 and 16~h of TES. There is just a addition of 0.6~\% for the 8~h TES and the SM 3.0 to 3.5 configurations. 
\subsubsection{Levelized costs of electricity}
The results of the LCOE claculation for the simulated CR configurations can be seen in Figure~\ref{CR_LCOE}. They are calculated using the finacial input parameter in Table~\ref{tbl: CRFinance} and a simplified method which is documented in Appendix~\ref{ChapterLCOE} on Page \pageref{ChapterLCOE}.

\begin{figure}[htbp]  
\centering
\includegraphics[width=1\linewidth]{FIG/CR_LCOE}
\caption[LCOE calculation results for CR simulation.]{LCOE calculation results for CR simulation.}\label{CR_LCOE}
\end{figure}
The results of the calculation shows that the lowest LCOE of 13.75~\$cent/kWh is reached at a SM of 2.0 and 10~h of TES and is just  marginal cheaper then using 8~h of TES with a LCOE of 13.76~\$cent/kWh. The LCOE addition from SM 2.0 to 2.5 is goes minimal up to 13.8~\$cent/kWh for 10~h of TES. The highest LCOE result of the simulated configurations is 17.90~\$cent/kWh at a SM of 3.5 and 8~h of TES. 

The behavior of the 12, 14 and 16~h of TES results for all SM of the LCOE is almost identical. Just the individual value is differently.

\begin{figure}[!htbp]
        \centering                
        \begin{subfigure}[b]{0.5\textwidth}
                \centering
                \includegraphics[width=1\textwidth]{FIG/CR_LCOE_lowinvest_BreakDown}
                \caption{LCOE break-down for SM~2.0 and 8~h~TES.}\label{CR_LCOE_lowinvest_BreakDown}
        \end{subfigure}%
        ~
        \begin{subfigure}[b]{0.5\textwidth}
                \centering
                \includegraphics[width=1\textwidth]{FIG/CR_LCOE_highinvest_BreakDown}
                \caption{LCOE break-down for SM~3.5 and 16~h~TES.}\label{CR_LCOE_highinvest_BreakDown}
        \end{subfigure}
        \caption[Break-down of selected CR-LCOE calculation results.]{Break-down of selected CR-LCOE calculation results.}\label{CR_LCOE_BreakDown}
\end{figure}
The calculated LCOEs of the simulated CR power plants is composed by seven cost parts. The investment costs of power block, tower and receiver, heliostat field, land purchase, TES, additional EPC, project management (PM) and risk as well as the annual costs for operation and maintain (O\&M). The brake-down of the LCOE components can be seen in Figure~\ref{CR_LCOE_BreakDown} for the simulation of the lowest and the highest configuration.  It can be seen that the investment costs of the land purchase don't have any influence to the LCOE of the simulated CR configurations. Also the share of the EPC, PM and risk as well as the share of O\&M don't varies much.

When comprising the results of the load curve covering with the LCOE calculation the lowest LCOE for reaching the target of 90~\% is the configuration of a SM of 3.0 and 14~h of TES. The LCOE for these configuration is 14.96~\$cent/kWh. 

For reaching 80~\% of the prescribed load curve the lowest LCOE is 13.83~\$cent/kWh using a SM of 2.5 and 10~h of TES. The lowest LCOE for covering 70~\% of the prescribed load curve is also the lowest LCOE of all and is mentioned above.

\pagebreak
\section{PTC power plant}
\subsection{Design  and simulation} \label{PTC power plant design  and simulation}
The PTC power plant was simulation in the “CSP parabolic trough (physical)" model in SAM using the option "no financial model". As before the input the EPW weather file to specify the hourly atmospheric conditions from Section~\ref{Location and weather data} was used. For the PTC simulation SAM uses the following input data:
\begin{itemize}
\item Latitude ($\,^{\circ}$)
\item Longitude ($\,^{\circ}$)
\item Elevation above sea level (m)
\item DNI (W/m\textsuperscript{2})
\item Atmospheric pressure (mbar)
\item Dry bulb temperature ($\,^{\circ}\mathrm{C}$)
\item Wetbulbtemperature($\,^{\circ}\mathrm{C}$)
\item Relative humidity (\%)
\item Wind velocity (m/s)
\end{itemize}
This Chapter describes in detail the input data of the PTC power plant simulation by there components, namely the  power cycle, solar collector, solar receiver, solar field and thermal energy storage (TES).

Also for the simulation of the PTC are the financial parameters and the LCOE calculated separately Microfoft Excel using a simplified method which is documented in Appendix~\ref{ChapterLCOE} on Page \pageref{ChapterLCOE}.
\subsubsection{Simulated configurations}
In order to reaching the goal of covering 90 \% of the scheduled production curve also the simulated PTC power plant used a variation of solar multiple and full load hours of TES. To covering the scheduled load the solar multiple was varied from 2 to 5.0 in steps of 5.0. This is significantly higher SM compared the simulated CR system was necessary to reach the 90 \% covering factor. As before with the simulation of the CR, the storage full load hours were varied from 8 to 16 h in steps of 2 h. The target of 100 MW net capacity was reached with a gross capacity of 120 MW with an estimated gross-to-net conversion factor of 0.83. The comparatively high gross turbine output is nessasary for the high parasitic burden of the system. Table~\ref{tbl: PTC_OverallConfig} summarizes the simulated configurations.
\begin{table}[!h]  
  \centering
	\begin{tabular}{ p{4.0cm}  C{1.0cm}  C{0.3cm} C{0.3cm} C{0.3cm} C{0.3cm} C{0.3cm} |C{0.3cm}  C{0.3cm} C{0.3cm} C{0.3cm} C{0.3cm} } 
	\hline	
\textbf{Item} & \textbf{Unit} & \multicolumn{10}{c}{\textbf{Value}} \\ \hline \hline
Net turbine capacity & MW\textsubscript{el} & \multicolumn{10}{c}{100} \\
Gross turbine capacity & MW\textsubscript{el} & \multicolumn{10}{c}{120} \\ \hline
Solar multiple & - & \multicolumn{5}{c}{2.0} & \multicolumn{5}{c}{2.5} \\
TES capacity & h & 8 & 10 & 12 & 14 & 16 & 8 & 10 & 12 & 14 & 16 \\ \hline 
Solar multiple & - & \multicolumn{5}{c}{3.0} & \multicolumn{5}{c}{3.5} \\
TES capacity& h & 8 & 10 & 12 & 14 & 16 & 8 & 10 & 12 & 14 & 16 \\ \hline 
Solar multiple & - & \multicolumn{5}{c}{4.0} & \multicolumn{5}{c}{4.5} \\
TES capacity& h & 8 & 10 & 12 & 14 & 16 & 8 & 10 & 12 & 14 & 16 \\ \hline 
Solar multiple & - & \multicolumn{5}{c}{5.0} & \multicolumn{5}{c}{ } \\
TES capacity& h & 8 & 10 & 12 & 14 & 16 &  \multicolumn{5}{c}{ } \\ \hline 
\end{tabular}
\caption[Simulated PTC solar multiple and thermal energy storage  configurations.]{Simulated PTC solar multiple and thermal energy storage  configurations.}\label{tbl: PTC_OverallConfig}
\end{table}

\subsubsection{Power cycle}
The PTC system also uses the steam Rankine cycle technology. In the modeled cycle, feedwater is heated in open (mixed stream) feedwater heaters with two intermediate turbine extractions - once for high pressure and once for low pressure, and the steam generation equipment consists of a preheater, boiler, and superheater. The HTF temperature at the field outlet is bound primarily by HTF stability, so maximum HTF temperatures for oil troughs typically range between 370$\,^{\circ}\mathrm{C}$ and 410$\,^{\circ}\mathrm{C}$. The turbine has a gross capacity of 120~MW\textsubscript{el}. So the net turbine capacity output at design (nameplate) is 100~MW\textsubscript{el}. The design inlet temperature of the HTF in the steam generator is 393$\,^{\circ}\mathrm{C}$ and outlet temperature of 293$\,^{\circ}\mathrm{C}$ at design point and operates at a pressure of 100 bar.



Table~\ref{tbl: PTCPowerplant} shows the input parameter for the power block design in SAM. Besides the capacity of the turbine and the condecer type, the parameters are coming from \cite{Wagner2011}. It is obviously, that the cycle conversion efficiency of the PTC power plant is lower than that one from the CR power plant. This is trace back to the lower cycle temperatures and the thereby resulting lower steam pressure in the turbine.



The air-cooled condenser was selected here as well, because water is an valuable resource in the region of Upington. The cooling system is designed to covers the stem generator thermal power. 
\begin{table}[!h]  
  \centering
	\begin{tabular}{  p{7.0cm}  C{2.0cm}  C{2.0cm} } 
	\hline	
\textbf{Item} & \textbf{Value} & \textbf{Unit} \\ \hline \hline
Turbine design capacity, gross  & 110 & MW\textsubscript{el} \\ 
Turbine design capacity, net & 100 & MW\textsubscript{el} \\ 
Boiler operating pressure & 100 & bar \\ 
Design inlet temperature & 393 & $\,^{\circ}\mathrm{C}$ \\ 
Design outlet temperature & 291 & $\,^{\circ}\mathrm{C}$ \\ 
Cycle conversion efficiency & 37.74 & \% \\ 
Steam generator design thermal power & 318.0 & MW\textsubscript{th}  \\
Power block start-up time & 0.5 & h \\ 
Minimum required start-up temperature & 300 & $\,^{\circ}\mathrm{C}$ \\
Maximum turbine over design operation & 90 & \\
Condenser type & air-cooled & - \\ 
\hline
\end{tabular}
\caption[PTC power block and condecer input parameter in SAM.]{PTC power block and condecer input parameter in SAM.}\label{tbl: PTCPowerplant}
\end{table}
\subsubsection{Solar collector (SCA)}
For the simulation of the PTC system the Ultimate Trough was selected as solar collector assembly (SCA). This collector is currently not in commercial use, but comparable troughs with similar characteristics are under construction. So the Ultimate Trough can be adopted  without technical modifications and no increase of risk. Figure~\ref{PTC_Ultimate_config} shows the information of the technical input parameter for SAM. These coming directly from a publication of the developer of the Ultimate Trough, Flabeg GmbH and sbp sonne GmbH \cite{Riffelmann2014}. 
\begin{figure}[bhtp]
\centering
\includegraphics[width=0.95\linewidth]{FIG/PTC_Ultimate_config}
\caption[Screenshot of Ultimate Trough SCA input parameter for SAM.]{Screenshot of Ultimate Trough SCA input parameter for SAM.}\label{PTC_Ultimate_config}
\end{figure}
The section "Optical Calculation" shows, that the total optical efficiency at design of the Ultimate Trough is 89.9~\%. The dimensions and main characteristics of the HCE is shown in the section "Receiver Geometry".
\subsubsection{Solar receiver (HCE)}
As heat collecting element (HCE) Schott's PTR80 was selected. This receiver tube has with 0.08~m an extended tube outer diameter. Thereby the tube can contain more HTF and reduce the mass flow in the tubes. Figure~\ref{PTC_HCE} shows the for the simulation inserted parameter. The values are the results of measurements of outdoor optical efficiency and indoor receiver heat loss of parabolic trough collector from researchers of National Renewable Energy Laboratory \cite{Kutscher2012}.
\begin{figure}[htbp]  
\centering
\includegraphics[width=0.95\linewidth]{FIG/PTC_HCE}
\caption[Screenshot of Schott PTR80 input parameter for SAM.]{Screenshot of Schott PTR80 input parameter for SAM.}\label{PTC_HCE}
\end{figure}
The section "Parameter and Variations" shows the different conditions of the HCE. Variation 1 shows the optimal condition of the HCE with annulus pressure of 0.0001 torr (0.000133~hPa), so an vacuum. The estimated average heat lost is 190~W/m. This Variation counts for 98.5~\% of all HCEs. In Variation 2 the vacuum is lost and which strongly affects the heat loss. It is 1~270~W/m and counts for 1~\% of all HCEs. In Variation 3 also the protection glass is broken, so the heat loss increase to 1~500~W/m. The total weighted losses is 207.35~W/m heat loss at design and about 0.85 optical derate and is shown at the last section most below.
\subsubsection{Solar field}
PTC solar fields are separated in sections. In the simulation of the PTC system the solar field is designed in two sections. Each section carries a feed pipe and a return pipe to transport the thermal energy to the power block. Attached to the header pipes are many loops, which contains the SCAs, the HCEs and the HTF. Figure~\ref{PTC_Field_ultimate} shows an typical layout for PTC system with two field subsections. As in the Figure is the simulated PTC system designed by four SCAs per loop \cite{Riffelmann2014}.
\begin{figure}[htbp]  
\centering
\includegraphics[width=0.9\linewidth]{FIG/PTC_Field_ultimate}
\caption[Typical configuration of a solar field layout with two field subsections for a Ultimate Trough.]{Typical configuration of a solar field layout with two field subsections for a Ultimate Trough \cite{Riffelmann2014}.}\label{PTC_Field_ultimate}
\end{figure}


As mentioned before synthetic oil is used as HTF for the simulation of the PTC system. The current standard for HTF in PTC systems is Terminol VP-1. The input parameter for the HTF are shown in Figure~\ref{PTC_HTF} and are limited by the performance characteristics of Terminol VP-1 of 12 to 400$\,^{\circ}\mathrm{C}$ \cite{Therminol2015}. The velocity range of Therminol VP-1 should be in the range of 0.36 and 4.97 m/s \cite{Wagner2014} and is affected by the inner tube diameter of the HCE, the HTF density and the loop flow rate \cite{NREL2015a}. A freeze protection temperature for the HTF of 150$\,^{\circ}\mathrm{C}$ is typically and also assumed for the simulation \cite{Kearney2002}.
\begin{figure}[htbp]  
\centering
\includegraphics[width=0.6\linewidth]{FIG/PTC_HTF}
\caption[Screenshot of HTF input parameter for SAM.]{Screenshot of HTF input parameter for SAM.}\label{PTC_HTF}
\end{figure}


The size of the solar field is strongly effected by the solar multiple and the thermal demand of the steam Rankine cycle. At a solar multiple of 1 (design thermal power of the steam generator) the solar field requires 453003~m\textsuperscript{2} ($\approx$45.3~ha) reflective aperture area of SCA. 65.86 ($\approx$66) loops are required at a SM of 1. The number of loops and so also the field size multiplies by the value of the SM.



Also the the solar field area and thereby the total land area depending by the SM. Equation~\ref{GL_PTCSolarfieldarea} shows the influence on the solar field area of the ratio between the row spacing and the SCA width. The roe spacing is assumed by 18~m between the parallel SCAs. The total land area results by Equation~\ref{GL_PTCtotallandarea} and is the solar field area multiplies by the factor of non-solar field area. The factor is assumed by 1.4 in the simulation \cite{NREL2015a}.
\begin{align}
\textrm{solar field area }(m^2) =\textrm{aperture area }(m^2) \times \frac{\textrm{row spacing }(m)}{ \textrm{SCA width }(m)} \label{GL_PTCSolarfieldarea}
\end{align}
\begin{align}
\textrm{total land area }(m^2) =\textrm{solar field area }(m^2) \times  \textrm{non-solar field multiplier}\label{GL_PTCtotallandarea}
\end{align}
Table~\ref{tbl: PTCSolarfield} shows the solar field simulation parameter of the PTC system for the simulation configuration values of the solar multiple. The parameters of the Table comes from the above mentioned relations between the Items. The design power of the steam generator (SG) stays at 318.0~MW\textsubscript{th} while the solar field thermal output rises proportional through the SM. 
\begin{table}[!h]  
  \centering
	\begin{tabular}{ p{3.3cm} C{1.1cm} C{1.1cm} C{1.1cm} C{1.1cm} C{1.1cm} C{1.1cm} C{1.1cm} C{1.1cm} } 
	\hline	
\textbf{Item} & \textbf{Unit} & \multicolumn{7}{c}{\textbf{Value}} \\ \hline \hline
SG Design power & MW\textsubscript{th} &  \multicolumn{7}{c}{318.0}\\
Design-point DNI & W/m &  \multicolumn{7}{c}{950}\\
\hline
\textbf{Solar multiple} &  & \textbf{2.0} & \textbf{2.5} & \textbf{3.0} & \textbf{3.5} & \textbf{4.0} & \textbf{4.5} & \textbf{5.0}\\ \hline 
Field th. output & MW\textsubscript{th} & 636 & 795 & 954 & 1~113 & 1~272 & 1~431 & 1~590\\
Number of loops  & - & 132 & 165 & 198 & 231 & 264 & 297 & 330\\ 
Aperture refl. area & ha & 90.6 & 113.3 & 135.9 & 158.6 & 181.2 & 203.9 & 226.5\\ 
Total land area & ha & 675 & 845 & 1013 & 1~182 & 1~351 &1~540 & 1~689\\ 
\hline
\end{tabular}
\caption[PTC solar field parameter.]{PTC solar field parameter.}\label{tbl: PTCSolarfield}
\end{table}
\pagebreak
\subsubsection{Thermal energy storage (TES)}
The thermal energy storage (TES) of the simulated PTC system uses a indirect two-tank molten salt system with "Hitec Solar Salt" as storage fluid. This storage fluid is made from sodium nitrate (60~\% NaNO\textsubscript{3}) and potassium nitrate (40~\% KNO\textsubscript{3}). Solar Salt needs an minimum operating temperature of 238$\,^{\circ}\mathrm{C}$ and and has a maximum operating temperature of 593$\,^{\circ}\mathrm{C}$. \cite{Suite2011,Kearney2003}

\begin{table}[htbp]  
  \centering
	\begin{tabular}{ p{3.9cm}  C{1.0cm} C{1.2cm} C{1.2cm} C{1.2cm} C{1.2cm} C{1.2cm} } 
	\hline	
\textbf{Item} & \textbf{Unit} & \multicolumn{5}{c}{\textbf{Value}} \\ \hline \hline
Storage type & - &  \multicolumn{5}{c}{indirect two-tank molten salt}\\
Storage fluid & - &  \multicolumn{5}{c}{Hitec Solar Salt}\\
Hot tank design temp. & $\,^{\circ}\mathrm{C}$ & \multicolumn{5}{c}{391}\\
Cold tank design temp. & $\,^{\circ}\mathrm{C}$ & \multicolumn{5}{c}{293}\\
\hline
\textbf{TES full load hours} & \textbf{h} & \textbf{8} & \textbf{10} & \textbf{12} & \textbf{14} & \textbf{16}\\ \hline 
Thermal capacity & MWh\textsubscript{th}  & 2~544 & 3~180 & 3~816 & 4~452 & 5~087 \\
Storage volume  & m\textsuperscript{3} & 34~407 & 43008 & 51~610 & 60~212 & 68~813\\
\hline
\end{tabular}
\caption[PTC system TES parameter.]{PTC system TES parameter.}\label{tbl: PTCTES}
\end{table}

The storage design temperatures depends from the solar field design temperatures, so the designed temperature difference is 98~K. As it is shown in Table~\ref{tbl: PTCTES} the operating temperature limits fits with the storage tank design temperatures. The heater set point is 265$\,^{\circ}\mathrm{C}$ for both storage tanks. The TES full load hours goes from 8 to 16 in steps of 2. The thermal capacity and also the storage volume rises with the TES full load hours. The simulated stored thermal capacity reaches from 2~544~MWh\textsubscript{th}  at 8 full load hours up to 5~087~MWh\textsubscript{th} at 16 full load hours. It is obviously, that the storage volume of the PTC needs to be more than 3 times that much than the storage volume of the CR system. This is the result of lower temperature difference of the HTF and the turbine design capacity.

Also for the simulation of the PTC system the dispatch control of the turbine output fraction in the storage settings was used. Figure~\ref{PTC_turbineoutput} shows the dispatch control matrix for the turbine output fraction.
\begin{figure}[htbp]  
\centering
\includegraphics[width=0.95\linewidth]{FIG/PTC_turbineoutput}
\caption[TES dispatch control matrix for turbine output fraction of PTC simulation in SAM.]{TES dispatch control matrix for turbine output fraction of PTC simulation in SAM.}\label{PTC_turbineoutput}
\end{figure}
\subsubsection{Financial parameter}
The financial parameter for calculating the LCOE of the simulated configuration are shown in Table~\ref{tbl: PTCFinance}. As before at the CR is the PTC power plant calculated over a lifetime of 25~years using a real interest rate of 7.5~\% \cite{FraunhoferISE2013}. Also the total plant availability of  15~\% once-off surcharge for EPC and contingencies on the total investment costs are equal to the CR \cite{Platzer2014}.


The specific costs for the collector field inclusive HTF-system is assumed with 275~\$/m\textsuperscript{2} from \cite{Morin2012}. These specific cost could also be reduced by 25~\% by using the assumed cost reduction from Flabeg \cite{FLABEG_FE_GmbH2015}. So the for the simulation assumed costs are highly conservative.

The specific costs for the power block are the same as at the CR but from other sources \cite{Platzer2014}. 

The specific costs of 50~\$/kWh\textsubscript{th} for the TES of the PTC are up to 50\% higher than the CR costs \cite{Platzer2014}. This is reduce to the lower energy density using lower storage temperatures \cite{Steinmann2015}. Other expected specific costs between 35 and 50~\$/kWh\textsubscript{th}  \cite{Steinmann2012}.

As before Fichtner analyzed also the annual O\&M costs for PTC power plants in SA and results costs of 1.96-1.97~\% of the total investment costs \cite{Fichtner2010}. The value of 2~\% can so also be assumed as conservative.

The costs for the land purchase comes from the "African Agriculture Review" report of the Nedbank Capital, which reported the prices for farmland in SA. For the LCOE calculation of the CSP and PV system was the land purchase costs of \$3,000/ha assumed which is based on these report \cite{Cassell2012}.
\begin{table}[!h]  
  \centering
	\begin{tabular}{  p{5.0cm} C{2.0cm} C{1.5cm}  C{1.5cm}  C{4.0cm} } 
	\hline	
\textbf{Item} & \textbf{Symbol}& \textbf{Value} & \textbf{Unit} & \textbf{Source}\\ \hline \hline
Collector field/HTF-system & $c_{CF}$ & 275 & \$/m\textsuperscript{2} & \cite{Morin2012}\\ 
Power block &$c_{PB,PTC}$ & 1000 & \$/kW\textsubscript{el} & \cite{Platzer2014}\\ 
Thermal energy storage & $c_{TES,PTC}$ & 50 & \$/kWh\textsubscript{th} & \cite{Platzer2014}\\ 
Land purchase & $c_{LP}$ & 3~000 & \$/ha & \cite{Cassell2012} \\ 
Annual O\&M & $f_{O\&M,PTC}$ &2&\% &\cite{Fichtner2010}\\ 
\hline
Lifetime&$n$ & 25 & years & \cite{FraunhoferISE2013} \\ 
Interest rate& $i_{PTC}$& 7.5 & \% & \cite{FraunhoferISE2013} \\ 
Surcharge for EPC, project management and risk & $f_{EPC,PTC}$ & 15 & \% & \cite{Platzer2014} \\
Total plant availability &$f_{avail,plant,PTC}$ & 96 & \% & \cite{Morin2012} \\ 
\hline
\end{tabular}
\caption[Finacial input parameter for PTC-simulation in SAM.]{Finacial input parameter for PTC-simulation in SAM.}\label{tbl: PTCFinance}
\end{table}
\subsection{Results of PTC power plant simulation}
This section comprised the results of the simulation of the PTC power plant and there belonging LCOE results. The input parameter of the simulation and calculation can be find in Section~\ref{PTC power plant design  and simulation}
\subsubsection{Load curve covering}
As it was shown in Table~\ref{tbl: PTC_OverallConfig} on Page~\pageref{tbl: PTC_OverallConfig} the PTC system based power plant was simulated in 35 different configurations, using a SM from 2.0 to 5.0 and 8 to 16~h of TES, in order to reach the target of 90~\% load curve covering. This section compares the results of the configuration with the lowest SM and TES hours (SM: 2.0 \& TES: 8~h) with the configuration using the highest SM and TES hours  (SM: 5.0 \& TES: 16~h) representative for all in between.  

The annual average load curve covering result of the selected simulated PTC configurations is shown in Figure~\ref{PTC_annual_profil}. It can be seen, that the PTC power plant using a SM of 5.0 and 16~h of TES covers mostly any time of the year the prescribed load curve. But it is also obviously that power supply declines during the night from 3~am on and is also decreasing in the morning hours at 8~am. In comparison to that covers the PTC power plant using a SM of 2.0 and 8~h of TES significantly less of the year the prescribed load curve. This simulated configuration of the PTC system starts declining the covering directly in the afternoon and standstill during the night from 3 to 6~am during the whole year.

\begin{figure}[htbp]  
\centering
\includegraphics[width=0.8\linewidth]{FIG/PTC_annual_profil}
\caption[Annual average load profile of selected PTC power plant configurations.]{Annual average load profile of selected PTC power plant configurations.}\label{PTC_annual_profil}
\end{figure}
The annual average load curve covering coming from the covering of the prescribed load at any time of the year through the net output of the power plant. The net output variate at any time step and depends on one side from the local whether date and on the other side from the set parameters. The annual sun paths diagram of Upington can be seen on Page~\pageref{SunPathUpington} in Figure~\ref{SunPathUpington}. 

Figure~\ref{PTC_winter_load} shows the load curve behavior of the selected PTC power plant configurations during the time of winter solstice, so the time with the shortest time of sunlight and the lowest angle of the incoming sunlight in Upington, SA. In the portrayed time line the DNI coming fro the whether file has a maximum of a about 830~W/m\textsuperscript{2} in peak times and there are also days with lower or almost no direct irradiance. 

The net electricity production of the PTC power plant with low configurations don't reach the prescribed load at any time in this portrayed time line. Therefore it is obviously that the solar field don't collect enough thermal energy during the day for filling up the storage to supply the steam turbines also by night. The power that is produced by the solar field goes at any time directly to the power generation and without direct sun irradiance the production is coming to standstill. The parasitic consumer has a peak load of about 8~MW during these days in these configuration.

In comparison to that is reaches the net electricity production of the PTC power plant with high configurations almost everyday the prescribed load for some hours. These configuration can also produce enough thermal energy by the solar field to fill up the storage to produce energy till after midnight. But also in that configuration is coming to standstill every night and rises first with the incoming direct irradiance again. In this configuration there is a peak load of about 12~MW coming from the parasitic consumer. 

The overproduction of the net output depends on the gross output control and the variable parasitic consumers. The turbine output was preset for all PTC power plant simulation using the turbine output fraction of the TES dispatch control matrix which illustrated in Figure~\ref{PTC_turbineoutput} on Page~\pageref{PTC_turbineoutput}. So it was not possible to regulate the net output under variable external influences at any time. 

\begin{figure}[htbp]  
\centering
\includegraphics[width=1\linewidth]{FIG/PTC_winter_load}
\caption[PTC load profile during the time of winter solstice (15. June - 25. June).]{PTC load profile during the time of winter solstice (15. June - 25. June).}\label{PTC_winter_load}
\end{figure}
It is obviously that the performance of the PCT power plant isn't great at all during the winter solstice. Neither under high then low performance configurations. This leads mainly from the declining optical performance of the solar collector field under impact of lower sunlight angle during the winter months. This performance loss can mainly reduce to the cosine efficiency of the solar field. Figure~\ref{PTC_field_eff} gives an impression of the strong influence of the cosine efficiency on the total optical field efficiency of the simulated PTC power plants. It can be seen that the cosine effect reduces the total field efficiency by almost 50~\% during the midday in June. 

\begin{figure}[!htbp]
        \centering                
        \begin{subfigure}[b]{0.5\textwidth}
                \centering
                \includegraphics[width=1\textwidth]{FIG/PTC_field_eff_winter}
                \caption{Avarage influence of the cosine efficiency on the avarage total optical field efficiency in June.}\label{PTC_field_eff_winter}
        \end{subfigure}%
        ~
        \begin{subfigure}[b]{0.5\textwidth}
                \centering
                \includegraphics[width=1\textwidth]{FIG/PTC_field_eff_summer}
                \caption{Avarage influence of the cosine efficiency on the avarage total optical field efficiency in December.}\label{PTC_field_eff_summer}
        \end{subfigure}
        \caption[Avarage influence of the cosine efficiency on the avarage total optical field efficiency of a PTC power plant for different months of the year.]{Avarage influence of the cosine efficiency on the avarage total optical field efficiency of a PTC power plant for different months of the year.}\label{PTC_field_eff}
\end{figure}
The loss in optical field performance by the cosine effect arise from the low angle of sunlight, therefor is the efficiency loss in the summer months through the cosine effect comparability low and has just an low effect in the morning and evening hours. 

The load behavior of the two exemplary PTC power plants during the summer solstice is shown in Figure~\ref{PTC_summer_load}. It is obviously that the higher values of the DNI which reaches 1~150~W/m\textsuperscript{2} in peak and the better field efficiency leads to a considerably higher covering of the prescribed load in both configurations. 

The lower PTC configuration which is using a SM of 2.0 and 8~h of TES can cover almost the full daytime load and a part of the night time load eith the net electricity production. But under this configuration the PTC power plant can't cover the prescribed load during the night at the best irradiance time of the year, therefrom is also the mentioned gap in production in the annual average load curve coming which was mentioned at the beginning of these section. The net electricity production over-scaled prescribed load at the beginning of the days which leads from the mentioned gross power output control. The parasitic consumers reaching a peak demand of 15~MW in this configuration. 

Figure~\ref{PTC_summer_load} shows also the load behavior of the PTC power plant with the highest simulated configuration, which doesn't stand still in this portrayed time span. The graph shows that the net out put is not constant at all. This is coming from the gross power output control and from the massive rises from the parasitic consumers, especially the solar field HTF pump. These pump needs to move a high volume of HTF through the HCEs. At a SM of 5.0 the solar field produce 5 times that much heat then the steam turbine actually needs at the design point. This over-scaling is necessary for the energy production in winter times during the night. Therefore the fractions of focused SCA's getting reduced when the generated energy filled up the storage completely and no more energy in demanded by the steam turbine. This is what  happens to the total parasitic consumption in the chart. The fractions of focused SCA's getting reduced so the power of solar field HTF pump is getting reduced as well. The peak of the parasitic consumers reaching a demand of about 33~MW which makes about 27.5~\% of the gross turbine capacity.

\begin{figure}[htbp]  
\centering
\includegraphics[width=1\linewidth]{FIG/PTC_summer_load}
\caption[PTC load profile during the time of summer solstice (16. December - 26. December).]{PTC load profile during the time of summer solstice (16. December - 26. December).}\label{PTC_summer_load}
\end{figure}
The parasitic demand is composed from different electrical loads of the PTC power plant. Dominantly is the mentioned solar field HTF pump but also the condenser operation of the power cycle. Additionally are the parasitic consumption of the TES \& cycle HTF pump, the field collector drives and the fixed loads. Figure~\ref{PTC_parasitics} shows the share of the produced gross energy of the selected configurations for the simulated year. About 10~\% of the annual gross energy production is going to the parasitic consumers. Therefrom goes about one-third respectively to the condenser operation and the solar field HTF pump. The reaming parasitic load is shared by the TES \& cycle HTF pump and the fixed loads. The share of the field collector drives to the parasitic loads is below 1~\% in any case.

\begin{figure}[!thbp]
        \centering   
        \begin{subfigure}[b]{0.65\textwidth}
                \centering
                \includegraphics[width=1\textwidth]{FIG/PTC_parasitics_low}
                \caption{SM: 2.0 TES: 8 h}\label{PTC_parasitics_low}
        \end{subfigure}
\par\medskip % Linebreak              
        \begin{subfigure}[b]{0.65\textwidth}
                \centering
                \includegraphics[width=1\textwidth]{FIG/PTC_parasitics_high}
                \caption{SM: 5.0 TES: 16 h}\label{PTC_parasitics_high}
        \end{subfigure}
        \caption[Share of annual gross energy output of selected PTC power plant configurations.]{Share of annual gross energy output of selected PTC power plant configurations.}\label{PTC_parasitics}
\end{figure}
Figure~\ref{PTC_parasitics_low} names the total net output for covering the prescribed load at about 380.51~GWh for the PTC power plant with a SM of 2.0 and 8~h of TES. The total sum of the annual prescribed load is 711.75~GWh. Therefrom results a covering of 53.5~\% of the PTC power plant with the lowest configurations. The annual net output result of the highest simulated PTC configuration can be found in Figure~\ref{PTC_parasitics_high} and is about 646.70~GWh which results in a load covering from about 90.9~\% over the year of the prescribed load. 

The remaining load covering results of all simulated PTC power plants can be found in Figure~\ref{PTC_LCCF}. The chart shows that at a SM of 2 all simulated TES variations reaching a load curve covering of 53.4~\%. It can be noteced that the growing rate of the covering is declining with the rising SM. The growing rate from the SM 2.0 to 2.5 is between 10.9~\% and 14.9~\% while it is from the SM 4.5 to 5.0 just between 1~\% and 2~\%. Also there is no noticeable gain between using 12, 14 or 16~h of TES. The difference is at highest 0.7~\%. 

\begin{figure}[htbp]  
\centering
\includegraphics[width=1\linewidth]{FIG/PTC_LCCF}
\caption[Load curve covering result of simulated PTC systems.]{Load curve covering result of simulated PTC systems.}\label{PTC_LCCF}
\end{figure}
The target of 90~\% load covering is just reached by a SM of 5.0 and more then 12~h of TES and it can be noted that it seams to be really elaborate for reaching that high value of annual covering using a PTC plant. 70~\% load covering was reached at all simulated PTC configurations. While 80~\% load covering was reached by all configurations besides the 8~h of TES.
\subsubsection{Levelized costs of electricity}
For the calculated LCOE the finacial input parameter in Table~\ref{tbl: PTCFinance} and a simplified method which is documented in Appendix~\ref{ChapterLCOE} on Page \pageref{ChapterLCOE} was used. These results of the LCOE claculation for the simulated PTC configurations can be seen in Figure~\ref{PTC_LCOE}. 

The lowest LCOE result of the simulated PTC power plants is 16.17~\$cent/kWh at a SM of 2.5 and 8~h of TES. The 10~h TES has marginal higher result using a SM of 3.0 (16.20~\$cent/kWh) and 2.5 (16.24~\$cent/kWh). The highest LCOE result of 21.66~\$cent/kWh was reached at a SM of 2.0 and 16~h of TES. 

Depending from almost identical load curve covering is the behavior of LCOE results of the simulated TES variations 12, 14 and 16~h almost identical. The difference in the height of the charts results from the higher investment cost of the larger storage configuration.

\begin{figure}[htbp]  
\centering
\includegraphics[width=1\linewidth]{FIG/PTC_LCOE}
\caption[LCOE calculation results for PTC simulation.]{LCOE calculation results for PTC simulation.}\label{PTC_LCOE}
\end{figure}
The calculated annual LCOE's of the simulated PTC power plants are composed by six cost parts, namely the investment costs of the collector field, TES, power block, land purchase and additional EPC, project management (PM) and risk as well as the annual operation and maintenance (O\&M) costs. The share of the annual LCOE for the selected PTC power plants can be seen in Figure~\ref{PTC_LCOE_BreakDown}. The share of the land purchase is almost irrelevant for the LCOE. 

When comprising the load curve covering results and the LCOE calculation results of the simulated PTC power plants, the lowest LCOE for reaching the target of 90~\% load curve covering is 19.22~\$cent/kWh at a SM of 5.0 and 16~h of TES. 

For reaching a covering of 80~\% of the load curve the PTC configuration with a SM of 3.5 and 12~h of TES has the lowest LCOE with 17.03~\$cent/kWh. The lowest LCOE for reaching 70~\% of the prescribed load curve is 16.20~\$cent/kWh at a SM of 3.0 and 10~h of TES.
\begin{figure}[!htbp]
        \centering                
        \begin{subfigure}[b]{0.5\textwidth}
                \centering
                \includegraphics[width=1\textwidth]{FIG/PTC_LCOE_lowinvest_BreakDown}
                \caption{LCOE break-down for SM~2.0 and 8~h~TES.}\label{PTC_LCOE_lowinvest_BreakDown}
        \end{subfigure}%
        ~
        \begin{subfigure}[b]{0.5\textwidth}
                \centering
                \includegraphics[width=1\textwidth]{FIG/PTC_LCOE_highinvest_BreakDown}
                \caption{LCOE break-down for SM~5.0 and 16~h~TES.}\label{PTC_LCOE_highinvest_BreakDown}
        \end{subfigure}
        \caption[Break-down of selected PTC-LCOE calculation results.]{Break-down of selected PTC-LCOE calculation results.}\label{PTC_LCOE_BreakDown}
\end{figure}

\pagebreak
\section{PV power plant with adapted EES}
\subsection{Design and simulation} \label{section PV system}
For this simulation a PV power plant was extended with an electrical energy storage (EES). This extended PV power plant was simulated in SAM's "Photovoltaic (detailed)" model with enabled battery storage option. Also for this simulation was the EPW weather file for Upington from Section~\ref{Location and weather data} used. For the PV simulation SAM uses the following input data:
\begin{itemize}
\item Latitude ($\,^{\circ}$)
\item Longitude ($\,^{\circ}$)
\item Elevation above sea level (m)
\item GHI, DNI and DHI (W/m\textsuperscript{2})
\item Dry bulb temperature ($\,^{\circ}\mathrm{C}$)
\item Wind velocity (m/s)
\end{itemize}
This section describes in detail the residual input data of the PV power plant with adapted battery storage. As before in the sections of simulated CSP technologies first the essential components of the system are defined. The financial parameters and the LCOE for this simulation was also here calculated separately and is documented in Appendix~\ref{ChapterLCOE} on Page \pageref{ChapterLCOE}.



As mentioned at the beginning of these Chapter this must be seen as theoretical simulation. Actually there are not that large electrical energy storage systems in form of battery storage available.



For this simulation the load was conected to the PV and Battery system as shown on Figure~\ref{PV_model_config}. The load shape was defined as before specified before in Chapter~\ref{Overall simulated configuration} and simulates the demand of the power grid. As it is shown there is also the actual power grid connected as additional component, but for this simulation just the power flows between PV, battery and load is in focus. As it is shown in the Figure, SAM actually just support the battery connection at the AC-bus via a power conversion system and is not able to simulate direct DC-connected batteries besides the PV-array before the PV-inverter. However in order to produce comparable results SAM was also used for the simulation of the PV power plant. As mentioned before this system contains out of the PV system with solar modules, solar and inverter and are described in the following section as well as the electrical energy storage.
\begin{figure}[htbp]  
\centering
\includegraphics[width=0.55\linewidth]{FIG/PV_model_config}
\caption[Model of the configurated PV plus batterie scheme.]{Model of the configurated PV plus batterie scheme \cite{Diorio2015}.}\label{PV_model_config}
\end{figure}
\subsubsection{Simulated configurations}
The configurations of the PV system with adapted electrical energy storage has also the target to reach 90~\% of the scheduled production curve from Section~\ref{Overall simulated configuration}. To reach this target it is necessary to over scale the energy production of the PV system to produce enough power to covers the given load and load the storage during the day. This is quite similar to the solar multiple of the CSP system, but can not be put on a same level. Therefore this over scaling will be called "PV multiple" (PVM) and means the multiple of the PV inverter output. For example the system with a maximum inverter output of 100~MW\textsubscript{el} is a PVM of 1 than has the PV system with a PVM of 2 a maximum inverter output of 200~MW\textsubscript{el}. 
\begin{table}[!b]  
  \centering
	\begin{tabular}{ p{4.0cm}  C{1.0cm} C{0.3cm} C{0.3cm} C{0.3cm} C{0.3cm} C{0.3cm}  | C{0.3cm} C{0.3cm} C{0.3cm} C{0.3cm} C{0.3cm} } 
	\hline	
\textbf{Item} & \textbf{Unit} & \multicolumn{10}{c}{\textbf{Value}} \\ \hline \hline
Maximum load supply & MW\textsubscript{el} & \multicolumn{10}{c}{100} \\ \hline
PV multiple & - & \multicolumn{5}{c}{1.0} & \multicolumn{5}{c}{1.8} \\
EES capacity & h & \multicolumn{5}{c}{-} & 4 & 5 & 6 & 7 & 8 \\ \hline 
PV multiple & - & \multicolumn{5}{c}{2.0} & \multicolumn{5}{c}{2.2} \\
EES capacity& h &  4 & 5 & 6 & 7 & 8 & 4 & 5 & 6 & 7 & 8 \\ \hline 
PV multiple & - & \multicolumn{5}{c}{2.4} & \multicolumn{5}{c}{2.6} \\
EES capacity & h & 4 & 5 & 6 & 7 & 8 & 4 & 5 & 6 & 7 & 8 \\ \hline 
\end{tabular}
\caption[Simulated configurations of the PV system with adapted EES.]{Simulated configurations of the PV system with adapted EES.}\label{tbl: PV_OverallConfig}
\end{table}
The simulated system load is maximum 100~MW\textsubscript{el}, so this is also the maximum supply of the system. The PV system was simulated one time without storage and a maximum inverter output of 100~MW\textsubscript{el} which is a PVM of 1. After that, the system was simulated in steps 0.2 of the PVM from 1.8 to 2.6. The adapted energy storage capacity was simulated from 4 to 8 hours in hourly steps. This storage capacity range is also defined for large-scale off-grid application in \cite{IEA2014c}. Table~\ref{tbl: PV_OverallConfig} summarizes the simulated configurations.
\subsubsection{PV system}
The simulated PV system is orientated at actual PV systems in SA. Therefore the currently under construction situated Mulilo Sonnedix Prieska PV Project in the Northern Cape will serve as a role model for the main components. In this 75~MW\textsubscript{ac}-project are poly crystalline 305~W\textsubscript{dc} modules of type BYD 305P6C-36 from BYD installed. The efficiency of the modules is about 15.72~\%. The full module specification is in Table~\ref{tbl: PVmodule}. The peak performance of the project is 86.23~MW\textsubscript{dc}. \cite{Morse2014}

\begin{table}[!h]  
  \centering
	\begin{tabular}{  p{5.0cm}  C{5.0cm}  C{1.4cm} } 
	\hline	
\textbf{Item} & \textbf{Value} & \textbf{Unit} \\ \hline \hline
Manufacturer  & BYD COMPANY LIMITED & - \\ 
Model & BYD 305P6C-36 & - \\ 
Cell type &  poly-crystalline silicon & - \\ \hline
Maximum power & 304.99 & W\textsubscript{dc} \\ 
Nominal efficiency & 15.72 & \% \\ 
Maximum power voltage & 36.2 & V\textsubscript{dc} \\ 
Maximum power current & 8.4 & A\textsubscript{dc}  \\
Open circuit voltage & 45.5 & V\textsubscript{dc}  \\ 
Short circuit current & 8.9 & A\textsubscript{dc}  \\
Temperature efficiency & -0.41 & \%/$\,^{\circ}\mathrm{C}$\\
Module area & 1.94 & m\textsuperscript{2} \\ 
Number of cells & 72 & -\\
\hline
\end{tabular}
\caption[Module specification of BYD 305P6C-36.]{Module specification of BYD 305P6C-36 under STC: 1000~W/m\textsuperscript{2}, cell temperature 25$\,^{\circ}\mathrm{C}$ \cite{NREL2015g}.}\label{tbl: PVmodule}
\end{table}


These for the simulation selected module type was already deposited in the module library in SAM. This large library is based on the database of Go Solar California for photovoltaic modules and inverters from the California Energy Commission \cite{NREL2015g}.

\begin{figure}[htbp]  
\centering
\includegraphics[width=0.70\linewidth]{FIG/PVModuleVA}
\caption[Current–voltage characteristic under STC of module BYD 305P6C-36.]{Current–voltage characteristic under STC of module BYD 305P6C-36 \cite{NREL2015g}.}\label{PVModuleVA}
\end{figure}


Figure~\ref{PVModuleVA} shows the module’s current–voltage characteristic with a short circuit voltage of 8.9~A and an open circuit voltage of 45.5~V.\newpage\noindent
The Mulilo Sonnedix Prieska PV Project install inverter from AEG. These specific AEG Protect PV.800 inverter is not available in the library of SAM, but a similar model with almost the same power rating was selected. Table~\ref{tbl: PVinverter} depicts the specifications of the for the simulation used Ingeteam inverter INGECON SUN 805TL U X420 Outdoor with 805~kW\textsubscript{ac} power at an efficiency of 98.33~\%. The efficiency curve of the inverter can be seen in Figure~\ref{InverterEfficiencyCurve}. 

\begin{figure}[htbp]  
\centering
\includegraphics[width=0.75\linewidth]{FIG/InverterEfficiencyCurve}
\caption[Efficientcy characteristic of Ingecon Sun 805TL U X420 Outdoor.]{Efficientcy characteristic of Ingecon Sun 805TL U X420 Outdoor \cite{IngeteamINC.2015,NREL2015g}.}\label{InverterEfficiencyCurve}
\end{figure}
\begin{table}[htbp]  
  \centering
	\begin{tabular}{ p{6.0cm}  C{7.0cm}  C{1.5cm} } 
	\hline	
\textbf{Item} & \textbf{Value} & \textbf{Unit} \\ \hline \hline
Manufacturer  & Ingeteam Power Technology & - \\ 
Model & Ingecon Sun 805TL U X420 Outdoor & - \\ 
Type &  central inverter & - \\ \hline
\textbf{Input (dc)} &  &  \\ 
Maximum power & 821.39 & kW\textsubscript{dc} \\ 
Voltage range & 611-820 & V\textsubscript{dc} \\ 
Maximum voltage & 1~000 & V\textsubscript{dc} \\ 
Maximum current & 1~350 & A\textsubscript{dc} \\
Nominal voltage & 715.91 & V\textsubscript{dc} \\ \hline
\textbf{Output (ac)} &  &  \\ 
Maximum power & 805 & kW\textsubscript{ac} \\ 
Nominal voltage & 420 & V\textsubscript{ac} \\
Maximum current & 1.35 & A\textsubscript{ac} \\
Frequency & 50-60 & Hz \\
cos$\phi$ & 1 & -\\ \hline
Maximum efficiency & 98.33 & \\% 
European efficiency & 98.29 & \\% 
Power consumption in operation &1.25 & kW\textsubscript{dc} \\ 
Power consumption at night & 0.12 & kW\textsubscript{ac} \\ 
\hline
\end{tabular}
\caption[Inverter specifications of Ingecon Sun 805TL U X420 Outdoor.]{Inverter specifications of Ingecon Sun 805TL U X420 Outdoor \cite{IngeteamINC.2015,NREL2015g}.}\label{tbl: PVinverter}
\end{table}
\newpage\noindent
As mentioned before the PV system was design under similar condition that the Mulilo Sonnedix Prieska PV Project. There a dc/ac ratio of 1.15 is used \cite{Morse2014}. This power ratio compares the photovoltaic array power to the inverter capacity \cite{Woodcock2013}. For the simulated PV system the same ratio was assumed. So at 100 MW\textsubscript{ac} inverter output, a total module capacity of 115 MW\textsubscript{dc} is needed. From the discussed parameter and the PVM SAM calculates the resulting number of modules and inverters as well as other parameters on its own. These PV system parameter are summarized in Table~\ref{tbl: PVsystemdesign}.

\begin{table}[!b]  
  \centering
	\begin{tabular}{ p{4.5cm} C{1.0cm} C{1.2cm} C{1.2cm} C{1.2cm} C{1.2cm} C{1.2cm} C{1.2cm} } 
	\hline	
\textbf{Item} & \textbf{Unit} & \multicolumn{6}{c}{\textbf{Value}} \\ \hline \hline
Module azimuth & $\,^{\circ}$ &\multicolumn{6}{c}{0 (north)}\\
Module tilt & $\,^{\circ}$ & \multicolumn{6}{c}{28.4}\\
Modules per string& - & \multicolumn{6}{c}{21}\\
String open circuit voltage& V\textsubscript{oc} & \multicolumn{6}{c}{955.3}\\
String max. power rated voltage& V\textsubscript{mp} & \multicolumn{6}{c}{759.8}\\
Maximum dc-voltage& V\textsubscript{dc} & \multicolumn{6}{c}{1~000}\\
Aspired dc/ac-ratio & - &\multicolumn{6}{c}{1.15}\\
\hline
\textbf{PV multiple} & - & \textbf{1.0} & \textbf{1.8} & \textbf{2.0} & \textbf{2.2} & \textbf{2.4} & \textbf{2.6}\\ \hline 
Total inverter capacity & MW\textsubscript{ac} & 99.8 & 180.3 &199.6 & 219.8 & 239.9 & 260.0 \\
Total module capacity & MW\textsubscript{dc} & 115.0 & 207.0 & 230.0 &253.0 & 276.0 & 299.0 \\
Strings in parallel & - & 17~954 & 32~318 & 35~909 & 39~500 & 43~091 & 46~682 \\
Number of modules & - & 377~034 & 678~678& 754~089 & 829~500 & 904~911 & 980~322 \\
Number of inverter  & - & 124 & 224 & 248 & 273 & 298 & 323 \\
Total module area & ha & 73.1 & 131.7 & 146.3 & 160.9 & 175.6 & 190.1 \\
Total land area & ha & 244 & 439 & 488 & 536 & 585 & 634 \\
\hline
\end{tabular}
\caption[PV system design parameter.]{PV system design parameter.}\label{tbl: PVsystemdesign}
\end{table}

For the orientation of the simulated PV-modules the latitude of Upington was used as tilt angle (28.4$\,^{\circ}$) and the azimuth is 0$\,^{\circ}$, so directly facing north. The self-shading model was selected, without external shading parameter. Therefore 2 modules along the side of row and 10 along the bottom of row was selected. Therefrom a shading loss of 0.545~\% on the solar radiation incident on the subarray. Further assumed losses are soiling, mismatch, diodes and connections and wiring losses. The assumed soiling losses reduce the solar radiation incident on the subarray of about 5~\%. The assumed mismatch loss of about 2~\% is related to slight differences in performance of individual modules in the array. Voltage drops across blocking diodes and electrical connections leads to a assumed diodes and connections loss of 0.5~\%. There are also assumed resistive losses of 2~\% in wiring on the dc-side and 1~\% wiring loss between the inverter and the grid connection point on the ac-side of the system. All PV system loss parameter coming from a IEEE photovoltaics specialists conference paper about performance parameters for grid-connected PV systems \cite{Marion2005}.
\subsubsection{Electrical energy storage (EES)}
The electrical energy storage is adapted though the system as it is shown in Figure~\ref{PV_model_config} on Page~\pageref{PV_model_config}. The EES basically consists out of two parts - the power conversions system (PCS) and the storage unit - as it is shown in Figure~\ref{TCC_EES} on Page~\pageref{TCC_EES}. For the simulation in SAM the PCS was simplified to the conversion efficiency. The AC to DC conversion efficiency was assumed with 99~\% as well as the DC to AC efficiency. Nevertheless for the LCOE calculation it is still relevant to define the performance of the PCS. As it is shown in Table~\ref{tbl: PVsystemdesign} the inverter capacity range is between 240 and 320~MW\textsubscript{ac} for the simulation cases with adapted storage. The maximum scheduled load from Section~\ref{Overall simulated configuration} is 100~MW. So depending from the PV inverter capacity the PCS capacity needs to variate as well to collect the  PV-overproduction. For the calculation of the LCOE it was assumed that the PCS needs to collects the inverter capacity minus the 100~MW daily base load.



As mentioned before the storage unit consists out of a Li-Ion battery, more precisely a using a Nickel Cobalt Aluminum ($LiNiCoAlO_2$  or NCA) cathode. This battery excels through a less expensive cathode material with improved safety characteristics and high specific energy \cite{NREL2015a}. The voltage characteristics of these battery for the simulation are shown in Figure~\ref{EES_VoltageDischarge}. The NCA battery is inter alia installed in the Tesla Model S and X \cite{Nykvist2015} which uses currently Panasonic cells and will also be installed in Teslas Power Wall system \cite{Shahan2015}. The performance and lifetime of the NCA batteries depending on how they are used. Tesla names the lifetime of 5~000 cycles without mentioning the assumed depth of discharge (DOD) \cite{Shahan2015}.

\begin{figure}[!htbp]  
\centering
\includegraphics[width=0.6\linewidth]{FIG/EES_VoltageDischarge}
\caption[Voltage proparties of NCA Li-Ion battery.]{Voltage proparties of NCA Li-Ion battery..}\label{EES_VoltageDischarge}
\end{figure}



The DOD is the amount of capacity in the battery that is usable by the system. For electric vehicles (EV) the DOD is set between 80-95~\% at the "top" end of the battery and 10-20~\% at the "bottom" end \cite{Warner2014}. The state of charge (SOC) is an expression of the present battery capacity as a percentage of maximum capacity and is the inverse of the DOD. It must be noted, that the nominal capacity of a battery is not the effective usable capacity of a battery and depends on the DOD. Also the DOD effects the cycle life of the batteries. The higher the DOD, the lower the cycle life. In other words, the lower the SOC, the longer the cycle life. \cite{MitElectricVehilceTeam2008}



SAM visualizes the decreasing capacity of the battery over the number of cycles with capacity fades as it is shown in Figure~\ref{CapacityFade} for the NCA. The figure shows the characteristic for 20~\% and 80~\% DOD. As mentioned above the graph shows the higher decreasing of the effective capacity at higher DOD. The performance data behind this graph is coming from \cite{Dahn2011} and correspond with other tests \cite{Read2009}.
\begin{figure}[bhtp]  
\centering
\includegraphics[width=0.75\linewidth]{FIG/CapacityFade}
\caption[Capacity fade of Nickel Cobalt Aluminum Lithium-Ion battery.]{Capacity fade of Nickel Cobalt Aluminum Lithium-Ion battery.}\label{CapacityFade}
\end{figure}

It must be noted, that the cycle lifetime is not only affected by the DOD but also by other conditions such as temperature and humidity \cite{MitElectricVehilceTeam2008}. SAM designs the cycle lifetime for the batteries at a constant room temperature of 20$\,^{\circ}\mathrm{C}$ therefore the storage room needs to be conditioned. Also the charging and discharging of the battery results losses in form of heat which needs to get led away. The cooling demand of the EES system is not included in the simulation, but must be actually significant. \cite{Diorio2015} 



The system was designed for a lifetime of 25 years. Therefore also the EES is designed for this period. For the simulation a amount of 365 cycles per year was assumed. So the NCA was designed for 9~125 cycles whiteout using a battery bank replacement. SAM also can analyses the cycle lifetime of the batteries. This analyses showed that the NCA battery reaches this cycle lifetime with a DOD of 50~\%. But under this conditions the effective capacity of the battery is nearly 0~\% after 25 years. As it is shown in Figure~\ref{CapacityFade} the effective capacity of the NCA battery decreases almost lineal with the cycle number. There are two opportunities to reduce the loss to the effective capacity increasing over the lifetime of the system. First is a battery bank replacement after a specified schedule or loss of effective capacity to a specified amount and second is to use a lower DOD which leads also to a lower effective usable capacity of the battery, so a higher nominal capacity of the battery would be necessary.



Further is to note that Li-Ion batteries has a expiry period \cite{Jossen2006}. Tesla gives a warranty of 10 years of the batteries \cite{Shahan2015} so it can be expected, that the expiry period is close to these warranty. So also for the simulation an battery bank replacement is indispensable.



The for a long lifetime an average SOC of 30-70~\% is recommended for Li-Ion batteries \cite{Jossen2006}. Consequently the SOC was set to a maximum of 30~\% and a minimum of 70~\% for the simulation. So a total DOD of 40~\% which signified 40~\% of the nominal battery capacity was assumed as usable. In order to have high storage availability over the total system lifetime it was assumed that the battery bank will replaced at a effective capacity loss of 20~\%. This is also advised for EES in electronic consumer devices \cite{Spotnitz2003}. Resulting from this assumptions the cycle lifetime of the cells in the battery bank is 2~500 cycles. Therefore the battery bank needs to be replaced after 6-7~years. So the battery bank needs to be replaced three times in the total lifetime of the system. 



Table~\ref{tbl: EESsystemdesign} summarizes results of the discussed parameter for the simulation. The adapted voltage level for this large-scale EES application from 820~V\textsubscript{dc}\ per string is coming from \cite{Leuthold2014}. The remaining values results from the above discussed parameter or are adapted from the SAM library \cite{Diorio2015}. 
\begin{table}[!htbp]  
  \centering
	\begin{tabular}{ p{5.0cm} C{1.0cm} C{1.2cm} C{1.2cm} C{1.2cm} C{1.2cm} C{1.2cm} } 
	\hline	
\textbf{Item} & \textbf{Unit} & \multicolumn{5}{c}{\textbf{Value}} \\ \hline \hline
Chemistry & - & \multicolumn{5}{c}{Lithium ion: nickel cobalt aluminum oxide} \\
Cell nominal voltage & V\textsubscript{dc} &\multicolumn{5}{c}{3.6}\\
Internal resistance & Ohm &\multicolumn{5}{c}{0.1}\\
Fully charged cell voltage & V\textsubscript{dc} &\multicolumn{5}{c}{4.2}\\
Exponential zone cell voltage & V\textsubscript{dc} &\multicolumn{5}{c}{4.1}\\
Nominal zone cell voltage & V\textsubscript{dc} &\multicolumn{5}{c}{3.6}\\
Nominal bank voltage & V\textsubscript{dc} &\multicolumn{5}{c}{820}\\
Cells in in series& - &\multicolumn{5}{c}{228}\\
Cell capacity & Ah &\multicolumn{5}{c}{55}\\
C-rate of discharge curve & - &\multicolumn{5}{c}{0.2}\\
Maximum C-rate charge & per/h &\multicolumn{5}{c}{1}\\
Maximum C-rate discharge & per/h &\multicolumn{5}{c}{1}\\
Total DOD& \% &\multicolumn{5}{c}{20}\\
\hline
\textbf{Storage capacity at 100~MW output} & \textbf{h} & \textbf{4} & \textbf{5} & \textbf{6} & \textbf{7} & \textbf{8} \\ \hline 
Effective capacity & MWh & 400 & 500 & 600 & 700 & 800 \\
Nominal capacity & MWh & 1~000 & 1~250 & 1~500 & 1~750 & 2~000\\
Cells & $\times 10^6$ & 5.1 & 6.3 & 7.6 & 8.8 & 10.1\\
\hline
\end{tabular}
\caption[EES system design parameter.]{EES system design parameter.}\label{tbl: EESsystemdesign}
\end{table}
\pagebreak
\subsubsection{Financial parameter}
The financial parameter for the LCOE calculation are composed in Table~\ref{tbl: PVFinance}. As mentioned before was the LCOE calculation for this simulation made separately in Microfoft Excel using a simplified method which is documented in Appendix~\ref{ChapterLCOE} on Page \pageref{ChapterLCOE}. As for the other systems is the PV power plant designed for a lifetime of 25~years. 



The specific large-scale PV system costs in South Africa  amount to 1.285~\$/W\textsubscript{DC,Peak} (14.5~ZAR/W\textsubscript{DC,Peak} using 11.286~USD/ZAR average in 2014 \cite{IRS2015}) which leads from an in 2014 actually built PV system \cite{Terblanche2015}. These specific costs are close to utility-scale PV systems in Europe with specific investments of 1~€/W\textsubscript{DC,Peak} \cite{FraunhoferISE2013}, corresponding to 1.33~\$/W (using the average exchange rate in 2014 of 1.33~\$/€ \cite{StatistaGmbH2015}).

The LCOE was calculated with different interest rates and O\&M costs for the PV power plant and the EES system. The real interest rate for the PV power plant was amused with 2.8~\% \cite{FraunhoferISE2013} and the annual O\&M costs are 1~\%  of the PV investment costs \cite{IEA2014a}. A degradation of 1~\% per year of the annual energy output was assumed for the ageing phenomena of the PV cells \cite{Tidball2010}.



The storage costs are calculates from three  parts. The costs for the PCS, the battery bank and the replacement costs of the battery bank during the lifetime of the system.

The costs of the PCS of 615~\$/kW (463~€/kW using the average exchange rate in 2014 of 1.33 \$/€ \cite{StatistaGmbH2015}) was taken over from a study toward comparative life cycle cost analysis of EES \cite{Zakeri2015}. This is the average value of  a wide cost range (250-600€/kW) of PCS for Li-Ion EES systems.

For the storage part of the EES a highly optimistic value of 300~\$/kWh was assumed from \cite{Nykvist2015}. The average storage part cost for Li-Ion EES in other studies is 914~\$/kWh \cite{Zakeri2015}.

For the replacement costs of the an other highly optimistic value of 150~\$/kWh was assumed. The cost for the replacement of the storage in \cite{Zakeri2015} are approximately the half of the storage costs. This costs was also assumed with a look to Figure~\ref{CostofLi-ion} on Page~\pageref{CostofLi-ion}.

The assumed EES battery bank replacements during the lifetime was described in this Chapter already.

The annual O\&M costs of 9.18~\$/kW was also a adaption from \cite{Zakeri2015} as well as the real interest rate of 8.0~\% 



As before for the other solar power plants also for the PV system was land purchase costs of \$3,000/ha \cite{Cassell2012} assumed.
\pagebreak
\begin{table}[!h]  
  \centering
	\begin{tabular}{  p{5.5cm} C{1.5cm} C{1.5cm}  C{1.5cm}  C{4.0cm} } 
	\hline	
\textbf{Item} & \textbf{Symbol}& \textbf{Value} & \textbf{Unit} & \textbf{Source}\\ \hline \hline
Lifetime &$n$ & 25 & years & \cite{FraunhoferISE2013} \\ \hline
Solar modules & $c_{sm}$&0.481 & \$/W\textsubscript{dc} & \cite{Terblanche2015}\\ 
Structural &$c_{st}$ &0.138 & \$/W\textsubscript{dc} & \cite{Terblanche2015} \\ 
Electrical parts &$c_{ep}$ &0.116 & \$/W\textsubscript{dc} & \cite{Terblanche2015} \\ 
Inverters&$c_{inv}$ &0.117 & \$/W\textsubscript{dc} & \cite{Terblanche2015} \\ 
Engineering and labour costs & $c_{elc}$ & 0.242 & \$/W\textsubscript{dc} & \cite{Terblanche2015} \\ 
Security and infrastructure & $c_{si}$& 0.091 & \$/W\textsubscript{dc} & \cite{Terblanche2015} \\ 
Transport and logistics & $c_{tl}$& 0.067 & \$/W\textsubscript{dc} & \cite{Terblanche2015}\\ 
Sub-Station & $c_{ss}$ & 0.071 & \$/W\textsubscript{dc} &\cite{Terblanche2015} \\ \hline
Total PV system costs & $c_{PV}$ & 1.285 &  \$/W\textsubscript{dc} &\cite{Terblanche2015} \\ 
Annual degradation factor &$f_{degrad}$ & 1 & \% & \cite{Tidball2010}\\ 
Annual PV O\&M costs factor &$f_{O\&M,PV}$ & 1 & \% & \cite{IEA2014a}\\
Interest rate of PV & $i_{PV}$& 2.8 & \% & \cite{FraunhoferISE2013} \\ \hline
EES PCS costs & & 615 & \$/kW & \cite{Zakeri2015} \\ 
EES battery bank costs & &300 & \$/kWh & \cite{Nykvist2015} \\ 
EES battery bank replacement costs & & 150 & \$/kWh & \cite{Zakeri2015} \\ 
Assumed EES battery bank replacements during lifetime & - & 3 & - & - \\ 
Annual EES O\&M costs & & 9.18 & \$/kW & \cite{Zakeri2015}\\
Interest rate of EES & $i_{EES}$& 8.0 & \% & \cite{Zakeri2015} \\ \hline
Land purchase &$c_{LP}$ & 3~000.00 & \$/ha & \cite{Cassell2012} \\ 
\hline
\end{tabular}
\caption[Finacial input parameter for PV-simulation in SAM.]{Finacial input parameter for PV-simulation in SAM.}\label{tbl: PVFinance}
\end{table}
\pagebreak
\subsection{Simulation results of PV power plant}
In this section are the results of the simulated PV system with and without EES and there belonging LCOE calculation results compromised. The results of the PV system without EES are described in the next section. In the following section after that are the results for the PV system with adapted EES described.  
\subsubsection{PV power plant without storage}

First year result 

LCOE incl degradation of 1\%

\begin{figure}[!htbp]
        \centering                
        \begin{subfigure}[b]{0.5\textwidth}
                \centering
                \includegraphics[width=1\textwidth]{FIG/PVwhithoutEESwinter}
                \caption{.}\label{PVwhithoutEESwinter}
        \end{subfigure}%
        ~
        \begin{subfigure}[b]{0.5\textwidth}
                \centering
                \includegraphics[width=1\textwidth]{FIG/PVwhithoutEESsummer}
                \caption{.}\label{PVwhithoutEESsummer}
        \end{subfigure}
        \caption[.]{.}\label{PVwhithoutEES}
\end{figure}


\begin{figure}[htbp]  
\centering
\includegraphics[width=0.8\linewidth]{FIG/PVwhithoutEESanual}
\caption[Annual average profile of the simulated PV power plant without EES.]{Annual average profile of the simulated PV power plant without EES.}\label{PVwhithoutEESanual}
\end{figure}

\pagebreak
\subsubsection{Load curve covering}
The PV system was also simulated with the adapted EES system as it was described in Section~\ref{section PV system}. In order to find a suitable size of the PV system and for the adapted EES, there was 25 different configurations simulated as they are described in Table~\ref{tbl: PV_OverallConfig}. The lowest defined configuration was simulated using a PVM of 2.0 and a 4~h of EES. The highest simulation configuration was defined with a PVM of 2.8 and a EES capacity for 8~h at 100~MW output. In the following the simulation results of these configurations are delineated  representative for the remaining 23 configurations in between. 

The simulation results are based on the first year energy output of the designed PV power plants and don't includes degradation of the PV system or the adapted EES system. 

The result of the simulation for the both selected configurations are present in the annual average load profile in Figure~\ref{PV_annual_profil}. The simulation result shows that the highest configuration can cover the load almost completely during the simulated first year and just has appreciable shortfalls of covering in the morning hours. Considering the the load curve behavior from 10~am to 4~pm it can be noticed, that the predicted load is covered over the whole year in that time span. Compared to that covers the lowest PV power plant configuration much less of the predicted load curve. The  annual average load profile shows that during the night the system coming to standstill from 12 pm to 5 am over the full year. Nevertheless, the PV power plant can cover the prescribed load completely during the midday at any time of the year.

\begin{figure}[htbp]  
\centering
\includegraphics[width=0.8\linewidth]{FIG/PV_annual_profil}
\caption[Annual average load profile of selected PV power plant configurations.]{Annual average load profile of selected PV power plant configurations.}\label{PV_annual_profil}
\end{figure}
The covering of the prescribed load during the day results mainly from the PV system, which also loads with there surplus power production the EES. During the night the EES taken over the covering of the load till it reaches the fixed bottom limit of the SOC. If the EES is fully loaded and the PV system is still producing more electricity than the prescribed load needs the surplus power is not going in to the load curve covering calculation as well as into the LCOE calculation as it was defined in Section~\ref{Overall simulated configuration}. 

\begin{figure}[!bhtp]  
\centering
\includegraphics[width=1\linewidth]{FIG/PV_winter_load}
\caption[PV system with adapted EES load profile during the time of winter solstice (15. June - 25. June).]{PV system with adapted EES load profile during the time of winter solstice (15. June - 25. June).}\label{PV_winter_load}
\end{figure}
The behavior of the power output of the selected simulated PV power plants and there power flow is shown in Figure~\ref{PV_winter_load} for the time with the lowest irradiance values. During the time of the winter solstice the value of GHI reaches just about 630~W/m\textsuperscript{2} in peak. The graph shows the behavior of the EES during the days and as it was described above the PV system first covers the prescribed load and then charges the EES with there surplus power. In the time span of the 22. to the 25. of June are the irradiance values of GHI during the days almost perfect. Nevertheless, both PV power plant configurations can't produce power without coming to standstill. The configuration with a PVM of 2.0 and 4~h of EES stops covering the prescribed load before this steps down to the night reduction. But the PV power plant configuration with a PVM of 2.8 and 8~h of EES covering the load till in the early morning hours. When comparing the value of the EES charging power it is obviously that there higher charging power peaks at a higher PVM configuration a the lower. 

During the winter times captures the EES the whole electricity power from the simulated PV systems, but this is not always the situation during the summer times. Figure~\ref{PV_summer_load} shows the load profile during the longest days of the year for the two selected configurations of the simulated PV power plants. The GHI rises up to 1~210~W/m\textsuperscript{2} in peak at the portrayed time span.

\begin{figure}[htbp]
\centering
\includegraphics[width=1\linewidth]{FIG/PV_summer_load}
\caption[PV system with adapted EES load profile during the time of summer solstice (16. December - 26. December).]{PV system with adapted EES load profile during the time of summer solstice (16. December - 26. December).}\label{PV_summer_load}
\end{figure}
The graph shows that the lowest PV power plant configuration can't cover the prescribed load during the night. The by the system provided power coming to stand still when the load got reduced for the night reduction at latest. Therefore the electricity production standstill during the night in the annual average load profile (Figure~\ref{PV_annual_profil}) as well. In comparison to that covers the highest PV power plant configuration almost the full prescribed load. Just during the following night times of days with lower irradiance (21. \& 22. December) comes the systems electricity output to standstill. Considering the system electricity output more precise, it offers high overproduction peaks which are the above mentioned situations where the storage got full charged and the PV system produces more power than the load requests.  

This surplus output depends on the ratio between the PV and EES system. The behavior of the ratio is shown in Figure~\ref{PV_energy_output}, which describes the share of the PV power plants energy output in the first year. The chart shows that the share of surplus energy rises with a higher PVM and decreases by larger EES capacity.

\begin{figure}[htbp]  
\centering
\includegraphics[width=1\linewidth]{FIG/PV_energy_output}
\caption[Share of energy output of all simulated PV power plants with adapted EES.]{Share of energy output of all simulated PV power plants with adapted EES.}\label{PV_energy_output}
\end{figure}
The Figure also gives an idea of the amount of the energy output of the simulated PV power plants for covering the prescribed load. The lowest simulated configuration produces an energy output of about 479.54~GWh in the first year which covers the prescribed load. The total sum of the annual prescribed load is 711.75~GWh. Consequently, the lowest configuration of the simulated PV power plants covers 67.4~\% of the prescribed load in the first year. The share of the surplus output is 1.2~\% of the total generated energy. The energy output of the highest simulated configuration has a share of a surplus output of about 4.9~\% and produces an load covering energy output of about 669.44~GWh in the first year. This output covers 94.1~\% of the prescribed load.

The from the energy output derived load curve covering results of the simulated PV power plant are shown in Figure~\ref{PV_LCCF}. At a PVM of 2.0 the size of the variations of EES don't have a big influence to the load curve covering. With the rising PVM the significance of the EES capacity gains as well. The forced target to covering 90~\% of the prescribed load is reached with the EES configuration 7 and 8~h from a PVM of 2.6 on. The configuration PVM 2.6 with 7~h of EES reaches actually just about 89.58~\% but reaching the 90~\% by round up. 70~\% load covering is reached by all EEs capacities and besides of the 4~h EES configuration also all simulated EES variants reaching the covering of 80~\% of the prescribed load. 

\begin{figure}[htbp]  
\centering
\includegraphics[width=1\linewidth]{FIG/PV_LCCF}
\caption[Load curve covering result of simulated PV systems with adapted EES.]{Load curve covering result of simulated PV systems with adapted EES.}\label{PV_LCCF}
\end{figure}
\subsubsection{Levelized costs of electricity}
The LCOE was calculated by using the finacial input parameter from Table~\ref{tbl: PVFinance} and a simplified method which is documented in Appendix~\ref{ChapterLCOE} on Page \pageref{ChapterLCOE}. It must be noted, that the calculation is based on the simulation result for the first year PV power plant results under consulting of a degradation factor for the PV system, but not for the EES. The results of the LCOE claculation for the simulated PV power plant configurations can be seen in Figure~\ref{PV_LCOE}. 

The lowest LCOE result of 25.18~\$cent/kWh is reached at a PVM of 2.2 and the lowest EES configuration and is marginal lower than  25.24~\$cent/kWh at a PVM of 2.4 with the same EES configuration. Generally must be said that there are no overlaps of LCOE lines in this chart and the spacing leads from the EES system costs.

\begin{figure}[htbp]  
\centering
\includegraphics[width=1\linewidth]{FIG/PV_LCOE}
\caption[LCOE calculation results for PV systems with adapted EES simulation.]{LCOE calculation results for PV systems with adapted EES simulation.}\label{PV_LCOE}
\end{figure}
The EES systems costs share makes the main part of the LCOE cost of the simulated PV power plants with an adapted EES. The break-down of the LCOE is shown for the lowest and highest configuration in Figure~\ref{SMPV_LCOE_BreakDown}. This chart clarify the big impact of the EES cost. At the lowest configuration 81~\% of the LCOE are traced to the EES and at the highest about 86~\%. The other results showed that the range of the EES is between 76\% and 89~\% of the calculated LCOE's.

\begin{figure}[!htbp]
        \centering                
        \begin{subfigure}[b]{0.5\textwidth}
                \centering
                \includegraphics[width=1\textwidth]{FIG/PV_LCOE_lowinvest_BreakDown}
                \caption{LCOE break-down for PVM~2.0 and 4~h~EES.}\label{PV_LCOE_lowinvest_BreakDown}
        \end{subfigure}%
        ~
        \begin{subfigure}[b]{0.5\textwidth}
                \centering
                \includegraphics[width=1\textwidth]{FIG/PV_LCOE_highinvest_BreakDown}
                \caption{LCOE break-down for PVM~2.8 and 8~h~EES.}\label{PV_LCOE_highinvest_BreakDown}
        \end{subfigure}
        \caption[Break-down of selected LCOE calculation  results of PV systems with adapted EES.]{Break-down of selected LCOE calculation results of PV systems with adapted EES.}\label{SMPV_LCOE_BreakDown}
\end{figure}
Comprising the results of the LCOE calculation and associated load curve covering results shows that for covering 90\% of the predicted load the configuration of PVM 2.6 and 7~h of EES reaching the lowest LCOE of 32.22~\$cent/kWh.

The lowest LCOE for reaching 80~\% of the prescribed load is 27.52~\$cent/kWh using 5~h of EES and also a PVM of 2.6. For reaching 70~\% the 2.2 and 4~h of EES reaches the lowest LCOE of 25.18~\$cent/kWh.
\pagebreak

\section{Interpretation and comparison of results}
PV-Storage --> Battery lifetime

Large scale storage costs (Deutschland)

Spinnennetzdiagramm 
PV unity scale without storage 
6-7.5 €cent/kWh (GHI 2000kWh/m2) in 2013 \cite{FraunhoferISE2013}\\\\


\chapter{Current economical situation and digression potential of main technologies to a long therm marked potential}
\section{CSP}
\cite{Smith2012}

\section{PV plant}

\subsubsection{Electrical storage}
lead-acid



Li-ion\cite{Nykvist2015}

\begin{figure}[htbp]  
\centering
\includegraphics[width=0.95\linewidth]{FIG/CostofLi-ion}
\caption[Cost of Li-ion battery packs in battery electric vehicles.]{Cost of Li-ion battery packs in battery electric vehicles \cite{Nykvist2015}.}\label{CostofLi-ion}
\end{figure}

\pagebreak

\chapter{Conclusion and outlook}
This thesis gives an overview of the power supply structure in South Africa and there need for flexible renewable energy sources. It is shown that SA has a high solar radiation which solar power plants in combination with storage applications can use to supply the bottleneck of there daily load curve in the evening hours. 


solar power plants in combination with storage applications are able to support the SA power supply where it is necessary.

For supporting the South African power supply over the full year and especially during the critical evening time the PV system in combination with a EES based on Li-ion technology is much more expansive than a CR or PTC power plant with TES technology. 

State of art in CSP TES molten salt

over the full day and esspasialy where it is essential 

PV state of art is mono and multi Si 

\subsubsection{Conclusion of main findings in this thesis}
\begin{itemize}
\item It was shown, that SA has a great potential for solar power due to high solar irradiation. The annual GHI of Upigton is more than \SI{2280}{\kilo\watt\hour\per\square\metre\per\year}, while  DNI is \SI{2621}{\kilo\watt\hour\per\square\metre\per\year}. Other regions in SA has even higher irradiation values. 
\item In the first half of 2015 SA had a total of \SI{1083}{\mega\watt\hour} unserved energy between 7:00 and 22:00 due to load shedding. This showed South Africas need for flexible power generation during daytime that espacially needs supply in the evening peak. 
\item Currently SA has a total capacity of \SI{1059.05}{\mega\watt} from PV system, which can supply a part of the South African system load during the day but come to a standstill when the evening peak begin, which then needs to be covert by OCGTs or pumped storage.
\item status of solar power generation in SA
\item generating a simplified daily generation schedule to supply at relevant times 
\item marked situation
\item financing 
\item PV, low LCOE, low load covering
\item 
\item CR best option for high load covering over the full year
\item PTC is also a good option but is reduced by cosine effect in Winter
\item LCOE could be higher for CSP if no load shape and turbine runs as long as it can


\end{itemize}


Spinnennetzdiagramm (cost reduction potential, Current LCOE, market maturity, contribution to grid stability, capacity factor, load curve covering factor)
Auf Christophs arbeit verweisen und auf hybridPV:
http://www.belectric.com/de/hybrid/

Sehr optemistischer LCOE bei PV ohne speicher.
Keine kühlung 
verlust durch alterung nur bedingt betrachtet



\subsubsection{Limitation}

\subsubsection{Outlook}

%
% Hier beginnen die Verzeichnisse.
%
\clearpage

%
%
% Hier beginnt der Anhang.
%
\clearpage
\appendix
\chapter{Appendix I}
\section{Part A}
\begin{table}[h] % Electricity in SA
\centering
\begin{tabular}{| l | r |}\hline
Production from: & Electricity [GWh]:\\\hline
- coal & 239~344 \\
- oil & 194 \\
- gas & 0 \\
- biofuels & 293 \\
- wast & 0 \\
- nuclear & 13~073 \\
- hydro (incl. PSP) & 4~860 \\
- geothermal & 0 \\
- solar PV & 50 \\
- solar thermal & 0 \\
- wind & 103 \\
- tide & 0 \\
- other sources & 0 \\\hline
Total production: & 257~919 \\\hline
Imports & 10 006 \\
Exports & -15~035 \\\hline
Domestic supply: & 252~890 \\\hline
Statistical differences & -2 769 \\
Energy industry own use & 30~678 \\
Losses & 22~351 \\\hline
Final consumption: & 197~092\\\hline
Industry & 117~272 \\
Transport & 3~826 \\
Residential & 38~779\\
Commercial and public services & 28~183 \\
Agriculture / forestry  & 5~709 \\
Other non-specified & 3~323 \\\hline
\end{tabular}
\caption[Electricity flow in South Africa 2012.]{Electricity flow in South Africa 2012\cite{Agency2015}.}\label{tab1}
\end{table}
\pagebreak
\begin{figure}[h]  
\centering
\includegraphics[height=0.95\textheight]{FIG/CSPOverview1}
\caption[CSP Technologies – Comparison I]{CSP Technologies – Comparison I \cite{Fichtner2010}.}\label{CSPOverview1}
\end{figure}
\begin{figure}[h]  
\centering
\includegraphics[height=0.95\textheight]{FIG/CSPOverview2}
\caption[CSP Technologies – Comparison I]{CSP Technologies – Comparison II \cite{Fichtner2010}.}\label{CSPOverview2}
\end{figure}
\pagebreak
%
\newpage

\chapter{Appendix B: Methodology for calculating the LCOE} \label{ChapterLCOE}
The LCOE is a financial value to compare different power producing technologies with different financial parameters and generation structures over a hole economical lifetime of power plants. To calculate the LCOE a simplified method from \cite{Morin2012}, originally proposed by \cite{Roy1997} was considered:
\begin{equation}
LCOE=\frac{C_{invest}\times(f_{annuity}+f_{ins.ann.})+C_{O\&M}}{E_{el,net,ann.}\times f_{avail,plant}} \label{LCOEold}
\end{equation}
This equation is commonly used for CSP plants and don't contain a degradation factor as it is necessary for the lifetime consideration of PV power plants. So the equation was extended with the average degradation factor of full observation period $f_{Full,degrad}$:

For the calculation of the LCOE of all power plants the extended equation was used:
\begin{equation}
LCOE=\frac{C_{invest}\times(f_{annuity}+f_{ins.ann.})+C_{O\&M}}{E_{el,net}\times f_{avail,plant} \times f_{Full,degrad}}\label{LCOE}
\end{equation}
with:
\begin{equation}
f_{annuity} = \frac{(1+i)^n \times i}{(1+i)^n-1} \label{annuity}
\end{equation}
\begin{equation}
f_{Full,degrad} = \frac{\sum\limits_{t=0}^{n-1} \frac{1}{(1+f_{degrad})^{t}}}{n} \label{GL_Degradationfactor}
\end{equation} 
\begin{itemize}
\item[ ] 
\begin{itemize}
\item[ ] 
\begin{itemize}
\item[$LCOE$]levelized cost of electricity in \$/kWh
\item[$C_{invest}$]investment costs in \$
\item[$C_{O\&M}$]annual operation and maintenance costs in \$
\item[$E_{el,net}$]produced electricity in the first year in kWh
\item[$f_{annuity}$]annuity factor in \%
\item[$f_{ins.ann.}$]annual insurance costs in \%
\item[$f_{avail,plant}$]total plant availability in \%
\item[$f_{Full,degrad}$]average degradation factor of full observation period in \%
\item[$f_{degrad}$]annual degradation factor in \%
\item[$n$]useful life and amortization period
\item[$i$]interest rate in \%
\item[$t$]year of lifetime (1, 2, ...n)
\end{itemize}
\end{itemize}
\end{itemize}
\section{CR power plant}
The investment costs for the CR power plant are calculated with the parameters from Table~\ref{tbl: CRFinance} acording to:
\begin{equation}
C_{invest,CR} = (C_{HF}+C_{LP,CR}+C_{PB,CR}+C_{T+R}+C_{TES,CR})\times(1+f_{EPC,CR}) \label{GL_CRInvest}
\end{equation} 
\begin{itemize}
\item[ ] 
\begin{itemize}
\item[ ] 
\begin{itemize}
\item[$C_{invest,CR}$]investment costs of CR in \$
\item[$C_{HF}$]heliostat field costs in \$
\item[$C_{LP,CR}$]land purchase costs in \$
\item[$C_{PB,CR}$]power block costs in \$
\item[$C_{T+R}$]tower and receiver costs in \$
\item[$C_{TES,CR}$]thermal energy storage costs in \$
\item[$f_{EPC,CR}$]surcharge for EPC, project management and risk in \%
\end{itemize}
\end{itemize}
\end{itemize}
The individual investment costs segments of the CR system can be derived from:
\begin{equation}
C_{HF} = c_{HF} \times A_{reflective,CR}
\end{equation} 
\begin{equation}
C_{LP,CR} = c_{LP} \times A_{land,CR}
\end{equation} 
\begin{equation}
C_{PB,CR} = c_{PB,PTC} \times P_{gross,CR}
\end{equation} 
\begin{equation}
C_{T+R} = c_{T+R} \times P_{receiver,th}
\end{equation} 
\begin{equation}
C_{TES,CR} = c_{TES,CR} \times E_{storage,th,CR}
\end{equation} 
\begin{itemize}
\item[ ] 
\begin{itemize}
\item[ ] 
\begin{itemize}
\item[$c_{HF}$]specific heliostat field costs in \$/m\textsuperscript{2}
\item[$A_{reflective}$]heliostat field reflective area in m\textsuperscript{2}
\item[$c_{LP}$]specific land purchase costs in \$/ha
\item[$A_{land,CR}$]total land area of CR power plant in ha
\item[$c_{PB,CR}$]specific power block and balance of CR plant costs in \$/kW\textsubscript{e}
\item[$P_{gross,CR}$]turbine gross capacity of CR power plant in kW\textsubscript{e}
\item[$c_{T+R}$]specific tower and receiver costs in \$/kW\textsubscript{th}
\item[$P_{receiver,th}$]receiver thermal power in kW\textsubscript{th}
\item[$c_{TES,CR}$]specific thermal energy storage costs for CR power plants in \$/kWh\textsubscript{th}
\item[$E_{storage,th,CR}$]thermal energy storage capacity of CR power plant in kWh\textsubscript{th}
\end{itemize}
\end{itemize}
\end{itemize}
Also the annual operational and maintenance costs are calculated with parameters from Table~\ref{tbl: CRFinance} and the investment costs acording to:
\begin{equation}
C_{O\&M,CR} = C_{invest,CR} \times f_{O\&M,CR}
\end{equation} 
\begin{itemize}
\item[ ] 
\begin{itemize}
\item[ ] 
\begin{itemize}
\item[$C_{O\&M,CR}$]annual O\&M costs of CR power plant in \$
\item[$C_{invest,CR}$]investment costs of CR power plant in \$m
\item[$f_{O\&M,CR}$]annual O\&M costs factor of CR power plant in \%
\end{itemize}
\end{itemize}
\end{itemize}
The CR system assume an annual plant availability $f_{avail,plant,CR}$ of 96~\% which is already included in $E_{el,net,ann.}$ as a result from the simulation in SAM. For the CR is no annual degradation provided. The factor $f_{Full,degrad}$ can be neglected.
\section{PTC power plant}
The investment costs for the PTC power plant are calculated with the parameters from Table~\ref{tbl: PTCFinance} acording to:
\begin{equation}
C_{invest,PTC} = (C_{CF}+C_{LP,PTC}+C_{PB,PTC}+C_{TES,PTC})\times(1+f_{EPC,PTC}) \label{GL_CRInvest}
\end{equation} 
\begin{itemize}
\item[ ] 
\begin{itemize}
\item[ ] 
\begin{itemize}
\item[$C_{invest,PTC}$]investment costs of PTC in \$
\item[$C_{CF}$]collector field costs in \$
\item[$C_{LP,PTC}$]land purchase costs in \$
\item[$C_{PB,PTC}$]power block costs in \$
\item[$C_{TES,PTC}$]thermal energy storage costs in \$
\item[$f_{EPC,PTC}$]surcharge for EPC, project management and risk in \%
\end{itemize}
\end{itemize}
\end{itemize}
The individual investment costs segments of the PTC system can be derived from:
\begin{equation}
C_{CF} = c_{CF} \times A_{reflective,PTC}
\end{equation} 
\begin{equation}
C_{LP,PTC} = c_{LP} \times A_{land,PTC}
\end{equation} 
\begin{equation}
C_{PB,PTC} = c_{PB,PTC} \times P_{gross,PTC}
\end{equation} 
\begin{equation}
C_{TES,PTC} = c_{TES,PTC} \times E_{storage,th,PTC}
\end{equation} 
\begin{itemize}
\item[ ] 
\begin{itemize}
\item[ ] 
\begin{itemize}
\item[$c_{CF}$]specific parabolic trough collector field costs in \$/m\textsuperscript{2}
\item[$A_{reflective}$]parabolic trough collector field reflective area in m\textsuperscript{2}
\item[$c_{LP}$]specific land purchase costs in \$/ha
\item[$A_{land,PTC}$]total land area of PTC power plant in ha
\item[$c_{PB,PTC}$]specific power block and balance of PTC plant costs in \$/kW\textsubscript{e}
\item[$P_{gross,PTC}$]turbine gross capacity of PTC power plant in kW\textsubscript{e}
\item[$c_{TES,PTC}$]specific thermal energy storage costs for PTC power plants in \$/kWh\textsubscript{th}
\item[$E_{storage,th,PTC}$]thermal energy storage capacity of PTC power plant in kWh\textsubscript{th}
\end{itemize}
\end{itemize}
\end{itemize}
Also the annual operational and maintenance costs are calculated with parameters from Table~\ref{tbl: PTCFinance} and the investment costs acording to:
\begin{equation}
C_{O\&M,PTC} = C_{invest,PTC} \times f_{O\&M,PTC}
\end{equation} 
\begin{itemize}
\item[ ] 
\begin{itemize}
\item[ ] 
\begin{itemize}
\item[$C_{O\&M,PTC}$]annual O\&M costs of PTC power plant in \$
\item[$C_{invest,PTC}$]investment costs of PTC power plant in \$
\item[$f_{O\&M,PTC}$]annual O\&M costs factor of PTC power plant in \%
\end{itemize}
\end{itemize}
\end{itemize}
The PTC system assume an annual plant availability $f_{avail,plant,PTC}$ of 96~\% which is already included in $E_{el,net,ann.}$ as a result from the simulation in SAM. For the CR is no annual degradation provided. The factor $f_{Full,degrad}$ can be neglected.
\section{PV power plant}
The investment costs for the PV system with storage are calculated with the parameters from Table~\ref{tbl: PVFinance} according to:
\begin{equation}
C_{invest,PV} = C_{PV-field}+C_{EES}+C_{LP,PV}
\end{equation} 
\begin{itemize}
\item[ ] 
\begin{itemize}
\item[ ] 
\begin{itemize}
\item[$C_{invest,PV}$]investment costs of PV power plant in \$
\item[$C_{PV-field}$]PV plant costs without storage in \$
\item[$C_{EES}$]electrical energy storage costs in \$
\item[$C_{LP,PV}$]land purchase costs in \$
\end{itemize}
\end{itemize}
\end{itemize}
The individual investment cost segments of the PV power plant with electrical energy storage can be derived from:
\begin{equation}
C_{PV-field} = P_{peak} \times (c_{sm}+c_{st}+c_{ep}+c_{inv}+c_{elc}+c_{si}+c_{tl}+c_{ss})
\end{equation} 
\begin{equation}
C_{EES} = c_{EES} \times E_{storage,el}
\end{equation}
\begin{equation}
C_{LP,PV} = c_{LP}\times A_{land,PV}
\end{equation} 
\begin{itemize}
\item[ ] 
\begin{itemize}
\item[ ] 
\begin{itemize}
\item[$P_{peak}$]total PV module peak power in W\textsubscript{p}
\item[$c_{sm}$]solar module costs in \$/W\textsubscript{p}
\item[$c_{st}$]structural costs in \$/W\textsubscript{p}
\item[$c_{ep}$]electrical parts costs in \$/W\textsubscript{p}
\item[$c_{inv}$]inverter costs in \$/W\textsubscript{p}
\item[$c_{elc}$]engineering and labour costs in \$/W\textsubscript{p}
\item[$c_{si}$]security and infrastructure costs in \$/W\textsubscript{p}
\item[$c_{tl}$]transport and logistics costs in \$/W\textsubscript{p}
\item[$c_{ss}$]sub-station costs in \$/W\textsubscript{p}
\item[$c_{EES}$]specific electrical energy storage costs in \$/kWh\textsubscript{el}
\item[$E_{storage,el}$]electrical energy storage capacity in kWh\textsubscript{el}
\item[$c_{LP}$]specific land purchase costs in \$/ha
\item[$A_{land,PV}$]total land area of PV power plant in ha
\end{itemize}
\end{itemize}
\end{itemize}
Also the annual operational and maintenance costs are calculated with parameters from Table~\ref{tbl: PVFinance} and the investment costs acording to:
\begin{equation}
C_{O\&M,PV} = C_{invest,PV} \times f_{O\&M,PV}
\end{equation} 
\begin{itemize}
\item[ ] 
\begin{itemize}
\item[ ] 
\begin{itemize}
\item[$C_{O\&M,PV}$]annual O\&M costs of PV power plant in \$
\item[$C_{invest,PV}$]investment costs of PV power plant in \$
\item[$f_{O\&M,PV}$]annual O\&M costs factor of PV power plant in \%
\end{itemize}
\end{itemize}
\end{itemize}
The annual operational and maintenance costs of the EES is already in the $c_{EES}$ included. The annual plant availability $f_{avail,plant}$ is assumed with 100\%. The $f_{annuity}$ for the PV plant is constituted by $f_{annuity,PV}$ and $f_{annuity,EES}$  calculated with:
\begin{equation}
f_{annuity}=\frac{f_{annuity,PV}\times(C_{PV-field}+C_{LP,PV})+ f_{annuity,EES}\times C_{EES}}{C_{invest,PV}}
\end{equation}

\pagebreak
\chapter{Appendix C: Annual avarage load profiles} \label{all_load_profile}
\section{CR power plants}

\section{PTC power plants}

\section{PV power plants}
\pagebreak
\chapter{Simulation results} \label{Simulation results}
%

\begin{sidewaystable} 
  \centering
\begin{tabularx}{\columnwidth}{ p{4.0cm}  C{2.0cm} *5{>{\centering\arraybackslash}X} | *5{>{\centering\arraybackslash}X} } 
	\hline	
\textbf{Item} & \textbf{Unit} & \multicolumn{10}{c}{\textbf{Value}} \\ \hline \hline
Net turbine capacity & \si{\mega\wattel} & \multicolumn{10}{c}{100} \\
Gross turbine capacity & \si{\mega\wattel} & \multicolumn{10}{c}{111} \\ \hline
Solar multiple & - & \multicolumn{5}{c|}{2.0} & \multicolumn{5}{c}{2.5} \\
TES capacity & h &  8 & 10 & 12 & 14 & 16 &  8 & 10 & 12 & 14 & 16 \\ \hline
gross output & \si{\giga\watt\hour} & \num{544.9} & \num{562.2} & \num{563.8} & \num{564.7} & \num{565.2} & 606.9 & 648.7 & 655.9  & 659.8 & 662.6 \\
parasitic & \si{\giga\watt\hour} & \num{45.1} & 46.1 & 46.3 & 46.3 & 46.4 & 52.5 & 55.3 & 55.8 & 56.1 & 56.3 \\
surplus net output & \si{\giga\watt\hour} & 16.2 & 16.7 & 16.7 & 16.8 & 16.8 & 14.7 & 15.5 & 15.8 & 15.9 & 16.1 \\
net output & \si{\giga\watt\hour} & 483.6 & 499.4 & 500.9 & 501.6 & 502.0 & 539.7 & 578.0 & 584.3 & 587.7 & 590.2\\
LCC & \si{\percent} & \num{67.9} & \num{70.2} & \num{87.0} & \num{70.5} & \num{70.5} & 75.8 & 81.2 & 82.1 & 82.6 & 82.9 \\
LCOE & \si{\usd/\mega\watt\hour} & 142.4 & 142.3 & 146.3 & 150.5 & 154.8 & 149.1 & 143.0 & 145.3 & 148.2 & 151.3\\ 
\hline 
Solar multiple & - & \multicolumn{5}{c|}{3.0} & \multicolumn{5}{c}{3.5} \\
TES capacity & h &   8 & 10 & 12 & 14 & 16 &  8 & 10 & 12 & 14 & 16 \\ \hline
gross output & \si{\giga\watt\hour} & 632.0 & 693.4 & 711.8 & 716.7 & 720.4 & 639.3 & 711.4 & 744.6 & 750.7 & 755.4 \\
parasitic & \si{\giga\watt\hour} & 58.0 & 62.3 & 63.5 & 63.9 & 64.3 & 62.6 & 68.2 & 70.4 & 70.9 & 71.3 \\
surplus net output & \si{\giga\watt\hour} & 12.5 & 13.2 & 13.8 & 14.0 & 14.2 & 11.0 & 11.4 & 11.9 & 12.2 & 12.4 \\
net output & \si{\giga\watt\hour} & 561.5 & 617.8 & 634.4 & 638.7 & 641.9 & 565.7 & 631.9 & 662.3 & 667.6 & 671.7\\
LCC & \si{\percent} & 78.9 & 86.8 & 89.1 & 89.7 & 90.2 & 79.5 & 88.8 & 93.0 & 93.8 & 94.4\\
LCOE & \si{\usd/\mega\watt\hour} & 164.3 & 152.8 & 152.3 & 154.8 & 157.4 & 185.2 & 169.3 & 164.9 & 166.8 & 169.1 \\ 
\hline \hline
\end{tabularx}
\caption[Results of CR systems.]{Results of CR systems.}\label{tbl: CR_results}
\end{sidewaystable} 



\begin{sidewaystable} 
  \centering
\begin{tabularx}{\columnwidth}{ p{4.0cm}  C{2.0cm} *5{>{\centering\arraybackslash}X} | *5{>{\centering\arraybackslash}X} } 
	\hline	
\textbf{Item} & \textbf{Unit} & \multicolumn{10}{c}{\textbf{Value}} \\ \hline \hline
Net turbine capacity & \si{\mega\wattel} & \multicolumn{10}{c}{100} \\
Gross turbine capacity & \si{\mega\wattel} & \multicolumn{10}{c}{120} \\ \hline
Solar multiple & - & \multicolumn{5}{c|}{2.0} & \multicolumn{5}{c}{2.5} \\
TES capacity & h &  8 & 10 & 12 & 14 & 16 &  8 & 10 & 12 & 14 & 16 \\ \hline
gross output & \si{\giga\watt\hour} & 433.5 & 433.3 & 433.3 & 433.0 & 432.6 & 519.1 & 545.9 & 549.0 & 550.5 & 551.1 \\
parasitic & \si{\giga\watt\hour} & 44.3 & 44.3 & 44.4 & 44.4 & 44.4 & 52.6 & 55.0 & 55.3 & 55.5 & 55.6 \\
surplus net output & \si{\giga\watt\hour} & 8.7 & 8.6 & 8.5 & 8.4 & 8.4 & 9.0 & 9.4 & 9.4 & 9.3 & 9.3\\
net output & \si{\giga\watt\hour} & 380.5 & 380.3 & 380.4 & 380.1 & 379.8 & 457.5 & 481.5 & 484.3 & 485.6 & 486.1 \\
LCC & \si{\percent} & 53.5 & 53.4 & 53.4 & 53.4 & 53.4 & 64.3 & 67.6 & 68.0 & 68.2 & 68.3\\
LCOE & \si{\usd/\mega\watt\hour} & 157.0 & 168.4 & 179.8 & 191.4 & 202.9 & 145.2 & 147.0 & 155.1 & 163.6 & 172.3 \\ 
\hline 
Solar multiple & - & \multicolumn{5}{c|}{3.0} & \multicolumn{5}{c}{3.5} \\
TES capacity & h & 8 & 10 & 12 & 14 & 16 &  8 & 10 & 12 & 14 & 16 \\ \hline
gross output & \si{\giga\watt\hour} & 558.0 & 600.1 & 612.2 & 615.2 & 617.5 & 587.5 & 635.4 & 652.6 & 657.1 & 660.5 \\
parasitic & \si{\giga\watt\hour} & 56.5 & 59.9 & 61.0 & 61.4 & 61.7 & 59.3 & 63.0 & 64.4 & 64.9 & 65.2 \\
surplus net output & \si{\giga\watt\hour} & 8.0 & 8.5 & 8.5 & 8.3 & 8.2 & 7.8 & 8.2 & 8.1 & 8.0 & 7.9 \\
net output & \si{\giga\watt\hour} & 493.5 & 531.7 & 542.7 & 545.5 & 547.5 & 520.4 & 564.2 & 580.1 & 584.2 & 587.4 \\
LCC & \si{\percent} & 69.3 & 74.7 & 76.3 & 76.6 & 76.9 & 73.1 & 79.3 & 81.5 & 82.1 & 82.5\\
LCOE & \si{\usd/\mega\watt\hour} & 147.3 & 144.9 & 150.0 & 157.2 & 164.5 & 152.6 & 148.4 & 151.8 & 158.2 & 164.7 \\ 
\hline \hline
\end{tabularx}
\caption[Results of PTC systems.]{Results of PTC systems.}\label{tbl: PTC_results}
\end{sidewaystable} 

\begin{sidewaystable} 
  \centering
\begin{tabularx}{\columnwidth}{ p{4.0cm}  C{2.0cm} *5{>{\centering\arraybackslash}X} | *5{>{\centering\arraybackslash}X} } 
	\hline	
\textbf{Item} & \textbf{Unit} & \multicolumn{10}{c}{\textbf{Value}} \\ \hline \hline
Net turbine capacity & \si{\mega\wattel} & \multicolumn{10}{c}{100} \\
Gross turbine capacity & \si{\mega\wattel} & \multicolumn{10}{c}{120} \\ \hline
Solar multiple & - & \multicolumn{5}{c|}{4.0} & \multicolumn{5}{c}{4.5} \\
TES capacity & h & 8 & 10 & 12 & 14 & 16 &  8 & 10 & 12 & 14 & 16 \\ \hline
gross output & \si{\giga\watt\hour} & 607.1 & 660.1 & 682.1 & 686.2 & 688.8 & 620.3 & 677.5 & 703.9 & 707.3 & 709.3 \\
parasitic & \si{\giga\watt\hour} & 61.0 & 65.2 & 66.9 & 67.4 & 67.7 & 62.2 & 67.2 & 69.2 & 69.6 & 69.9 \\
surplus net output & \si{\giga\watt\hour} & 8.1 & 8.0 & 7.9 & 7.7 & 7.6 & 8.3 & 7.9 & 7.6 & 7.4 & 7.3 \\
net output & \si{\giga\watt\hour} & 538.0 & 586.9 & 607.3 & 611.1 & 613.5 & 549.9 & 602.4 & 627.1 & 630.4 & 632.2 \\
LCC & \si{\percent} & 75.6 & 82.5 & 85.3 & 85.9 & 86.2 & 77.3 & 84.6 & 88.1 & 88.6 & 88.8 \\
LCOE & \si{\usd/\mega\watt\hour} & 159.3 & 153.4 & 155.4 & 161.5 & 168.0 & 168.0 & 160.6 & 161.2 & 167.2 & 173.6\\ 
\hline 
Solar multiple & - & \multicolumn{5}{c|}{5.0} &  \\
TES capacity & h & 8 & 10 & 12 & 14 & 16 &  \\ \hline
gross output & \si{\giga\watt\hour} & 629.4 & 691.1 & 719.5 &  722.9 & 725.3 &\\
parasitic & \si{\giga\watt\hour} & 63.3 & 68.4 & 70.4 & 70.8 & 71.0 & \\
surplus net output & \si{\giga\watt\hour} & 8.5 & 8.2 & 7.9 & 7.6 & 7.6 &\\
net output & \si{\giga\watt\hour} & 557.5 & 614.5 & 641.3 & 644.5 & 646.7 & \\
LCC & \si{\percent} & 78.3 & 86.3 & 90.1 & 90.6 & 90.9 & \\
LCOE & \si{\usd/\mega\watt\hour} & 177.0 & 167.7 & 167.4 & 173.3 & 179.5 &\\ 
\hline \hline
\end{tabularx}
\caption[Residual results of PTC systems.]{Residual results of PTC systems.}\label{tbl: PTC_results2}
\end{sidewaystable} 



\begin{sidewaystable} 
  \centering
\begin{tabularx}{\columnwidth}{ p{4.0cm}  C{2.0cm} *5{>{\centering\arraybackslash}X} | *5{>{\centering\arraybackslash}X} } 
	\hline	
\textbf{Item} & \textbf{Unit} & \multicolumn{10}{c}{\textbf{Value}} \\ \hline \hline
Maximum load supply & \si{\mega\wattel} & \multicolumn{10}{c}{100}\\ \hline
PVM & - & \multicolumn{5}{c|}{1.0} & \multicolumn{5}{c}{1.8} \\
EES capacity & h & \multicolumn{5}{c|}{-} & 4 & 5 & 6 & 7 & 8 \\ \hline
surplus net output & \si{\giga\watt\hour} & \multicolumn{5}{c|}{-}  & 6.0 & 3.3 & 3.3 & 3.3 & 3.2 \\
net output & \si{\giga\watt\hour} & \multicolumn{5}{c|}{233.7} & 479.5 & 484.8 & 485.9 & 486.5 & 487.0\\
LCC & \si{\percent} & \multicolumn{5}{c|}{32.8} & 67.4 & 68.1 & 68.3 & 68.4 & 68.4 \\
LCOE & \si{\usd/\mega\watt\hour} & \multicolumn{5}{c|}{77.6} & 344.5 & 336.3 & 338.0 & 343.7 & 350.5\\ 
\hline 
PVM & - & \multicolumn{5}{c|}{2.0} & \multicolumn{5}{c}{2.2} \\
EES capacity & h & 4 & 5 & 6 & 7 & 8 & 4 & 5 & 6 & 7 & 8 \\ \hline
surplus net output & \si{\giga\watt\hour} & 27.1 & 7.6 & 4.0 & 3.9 & 3.9 & 61.5 & 28.9 & 9.4 & 4.8 & 4.8 \\
net output & \si{\giga\watt\hour} & 507.4 & 532.8 & 538.7 & 540.0 & 540.6 & 521.1 & 561.7 & 586.4 & 593.1 & 594.1 \\
LCC & \si{\percent} & 71.3 & 74.9 & 75.7 & 75.9 & 75.9 & 73.2 & 78.9 & 82.4 & 83.3 & 83.5 \\
LCOE & \si{\usd/\mega\watt\hour} & 402.5 & 376.5 & 366.9 & 366.9 & 371.9 & 463.2 & 427.9 & 402.5 & 392.5 & 391.6\\ 
 \hline 
PVM & - & \multicolumn{5}{c|}{2.4} & \multicolumn{5}{c}{2.6} \\
EES capacity & h & 4 & 5 & 6 & 7 & 8 & 4 & 5 & 6 & 7 & 8 \\ \hline
surplus net output & \si{\giga\watt\hour} &  101.2 & 62.7 & 31.4 & 13.6 & 11.3 & 14.2 & 10.3 & 6.5 & 3.9 & 3.4 \\
net output & \si{\giga\watt\hour} & 528.4 & 576.6 & 615.3 & 637.6 & 641.0 & 533.9 & 583.6 & 630.7 & 663.2 & 669.4 \\
LCC & \si{\percent} & 74.2 & 81.0 & 86.4 & 89.6 & 90.1 & 75.0 & 82.0 & 88.6 & 93.2 & 94.1 \\
LCOE & \si{\usd/\mega\watt\hour} & 524.2 & 482.4 & 448.4 & 425.7 & 417.6 & 585.1 & 537.2 & 498.1 & 470.1 & 458.4\\ 
\hline \hline
\end{tabularx}
\caption[Results of PV systems.]{Results of PV systems.}\label{tbl: PTC_results2}
\end{sidewaystable} 



\pagebreak
\backmatter

\ifthenelse{\equal{\FHTWCitationType}{HARVARD}}{}{\bibliographystyle{gerabbrv}}
\bibliography{library}
\clearpage

\include{figures}
\include{tables}
\include{source_code}
\phantomsection
\addcontentsline{toc}{chapter}{\listacroname}
\chapter*{\listacroname}
\begin{acronym}[XXXXX]
  	\acro{AC}[AC]{alternative current}
  	\acro{BOS}[BOS]{balance of system}
  	\acro{CAES}[CAES]{compressed air energy storage}
  	\acro{CF}[CF]{capacity factor}
  	\acro{CO2}[CO\textsubscript{2}]{carbon dioxide}
    \acro{CPV}[CPV]{concentrating photovoltaic}
    \acro{CR}[CR]{central receiver}
    \acro{CSP}[CSP]{concentrating solar power}
    \acro{CST}[CST]{concentrating solar thermal} 
    \acro{DC}[DC]{direct current}
    \acro{DHI}[DHI]{diffuse horizontal irradiance}
    \acro{DNI}[DNI]{direct normal irradiance}
	\acro{DOD}[DOD]{depth of discharge}
    \acro{DSG}[DSG]{direct steam generation}
    \acro{EV}[EV]{electric vehicles}
    \acro{GHI}[GHI]{global horizontal irradiance}
    \acro{GNI}[GNI]{global normal irradiance}
    \acro{HCE}[HCE]{heat collecting element}
    \acro{HTF}[HTF]{heat-transfer fluid} 
    \acro{IEA}[IEA]{International Energy Agency}
    \acro{IPP}[IPP]{independent power producers}
    \acro{IRP}[IRP]{Integrated Resource Plan}
    \acro{LCC}[LCC]{load curve covering}
    \acro{LCOE}[LCOE]{levelised cost of electricity}
    \acro{LFR}[LFR]{linear Fresnel reflector}
    \acro{NCA}[NCA]{Nickel Cobalt Aluminum}
    \acro{NERSA}[NERSA]{National Energy Regulator of South Africa}
    \acro{NREL}[NREL]{US National Renewable Energy Laboratory}
    \acro{O&M}[O\&M]{operation and maintenance}
    \acro{OCGTs}[OCGTs]{open cycle gas turbines}
    \acro{PCS}[PCS]{power conversions system}
    \acro{PHS}[PHS]{pumped hydroelectric storage}
    \acro{PSP}[PSP]{pumped storage plants}
    \acro{PTC}[PTC]{parabolic trough collector}
    \acro{PV}[PV]{photovoltaic}
    \acro{REIPPPP}[REIPPPP]{Renewable Energy Independent Power Producer Procurement Program}
    \acro{SAM}[SAM]{System Advisor Model}
    \acro{SCA}[SCA]{solar collector assembly}
    \acro{SCE}[SCE]{solar collector element}
    \acro{SM}[SM]{solar multiple}
    \acro{SOC}[SOC]{state of charge}
    \acro{STE}[STE]{solar thermal electricity}
    \acro{TES}[TES]{thermal energy storage}
    \acro{UCLF}[UCLF]{unplanned capability loss factor}
\end{acronym}



\end{document}
