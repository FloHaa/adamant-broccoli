\chapter{The South African energy sector}
The Republic of South Africa (SA) is one of the most developed country in Sub-Saharan Africa and acording to the Human Development Index (HDI) it is growing constand since the 1980's, nevertheless it counts as medium developed country \cite{UNDP2014}. Also the population and energy demand is constantly growing \cite{TheWorldBank2015,Agency2015}.

SA has also one of the strongest econemys in Africa, therfore it is accounting for about 30~\% of the primary energy consumption of the entire continent Africa in 2014 \cite{BP2015b}. 
\section{Primary energy consumption}
\begin{figure}[!b]
        \centering                
        \begin{subfigure}[b]{0.45\textwidth}
                \centering
                \includegraphics[width=1\textwidth]{FIG/PrimWorld}
                \caption{World share of primary energy consuption.}\label{PrimWorld}
        \end{subfigure}
        ~
        \begin{subfigure}[b]{0.45\textwidth}
                \centering
                \includegraphics[width=1\textwidth]{FIG/PrimSA}
                \caption{SA share of primary energy consuption.}\label{PrimSA}
        \end{subfigure}
\caption[Comparision of Worldwide and South African primary energy consumption by fuel in 2014.]{Comparision of Worldwide and South African primary energy consumption by fuel in 2014 \cite{BP2015b}.}\label{PEKreis}
\end{figure}
The primary energy consumption of SA was in 2014 about 1~473.52~TWh \cite{BP2015b}. This consumption is mainly based on fossil energy resources. More than 96~\% of the primary energy consumption was in 2014 based fossil fuels and further 2.8~\% on nuclear. Thereby is the share on primary energy consumption predominant coming from coal. Figure \ref{PEKreis} shows the primary energy mix of SA in comparision with the worldwide primary energy mix. \cite{BP2015b}

It can be seen that coal is with about 71~\% the main primary energy source. Also crude oil (23~\%) is a very important energy source for SA. Therefor is the primary energy consumption from renewable energies in SA just about 0.7~\%. Comparing to this, the global share of  renewable primary energy consumption was about 9.3~\% in 2015. But it must be said that the share on renewable energy growth by 482.2 \% from 2013 to 2014. \cite{BP2015b}

Figure \ref{PrimEnergyDevelopment} shows the growing South African primary energy consumption. Between 1965 and 2014 the annual primary energy consumption in SA has risen from 351.96~TWh up to 1~473.52~TWh. Consequently a avarage anual growing rate in primary energy consumption in SA of 8.5~\% in the past half century. \cite{BP2015c}

\begin{figure}[htbp]  
\centering
\includegraphics[width=1\linewidth]{FIG/PrimEnergyDevelopment}
\caption[Evolution of primary energy consuption of SA.]{Evolution of primary energy consuption of SA \cite{BP2015c}.}\label{PrimEnergyDevelopment}
\end{figure}
The spread in consumption of primary energy is defined by three major consumption groups, namly the industry sector with about 34.9~\%, the transport sector which consumes about 28.6~\% and other sectors with about 36.5~\%, which includes agriculture, commerce and public services, residential and non-specified consumers \cite{DepartmentofEnergy2012}. So it can be said that the sectors industy and transport are the main energy consumer in SA. 
\pagebreak
\section{Electricity supply and demand}
Coal-fired power stations restrained the electricity generation in SA. 92.8~\% was generated thought coal (239~344~GWh) in 2012, while 5.1~\% of the annual supply was generated by nuclear power (13~075~GWh) and 1.3~\% was generated from hydropower applications (4~860~GWh). Less significant electricity sources for SA are 0.08~\% Oil (194~GWh), 0.11~\% bio-fuels (293~GWh), 0.02~\% PV (50~GWh) and 0.04~\% wind (103~GWh). \cite{Agency2015}



The final electricity consumption in SA was about 197~092~GWh in 2012. The final consumption consist also three main consumers. The largest consumer group with about 59.5~\% are industrial consumers, followed by residential consumers (19.7~\%) and commercial and public services (14.3~\%). The complete electricity flow is shown in Annexure I, Part A, Table \ref{tab1}, on page \pageref{tab1}. \cite{Agency2015}

\begin{figure}[!h] % Sommer/Winter Verbrauchskurve
\centering
\includegraphics[width=0.9\linewidth]{FIG/SummerWinterDemand}
\caption[Summer and winter load profiles.]{Summer and winter load profiles \cite{Eskom2014}.}\label{DEMAND}
\end{figure}


The demand in South Africa has different load profiles during winter and summer as shown in Figure~\ref{DEMAND}. The peak demand is usually in the evening hours and particularly high during winter time. According to Eskom is the peak demand at about 32~GW, while the usually demand during daytime is between 26~GW and 29~GW. \cite{Eskom2014}

\subsection{Rising energy consumption and security of supply}

\begin{figure}[!h] % Monthly Reserve
\centering
\includegraphics[width=0.9\linewidth]{FIG/AveragemonthlySA}
\caption[Average monthly \% operating reserves.]{Average monthly \% operating reserves\cite{Eskom2014}.}\label{Abb1}
\end{figure}


\begin{figure}[!h] % Demand growth
\centering
\includegraphics[width=0.9\linewidth]{FIG/SA_Electricity_demand_growth}
\caption[Electricity demand growth by sector in South Africa in the New Policies Scenario.]{Electricity demand growth by sector in South Africa in the New Policies Scenario \cite{IEA2014f}.}\label{Abb1}
\end{figure}

\subsection{Structure of power distribution}
Kilometerlängen, Verluste im Netz \cite{Eskom2014a}

On mainland sub-Saharan Africa, SA has with around 85~\% the highest electrification rate. About 11~\% of households don't have access to electricity and a further 4~\% rely on illegal access (non-paying) or obtain access informally (from one household to another but paying). \cite{IEA2014f}

\begin{figure}[htbp] % Netzstrucktur
\centering
\includegraphics[width=0.9\linewidth]{FIG/transmissionprojekts}
\caption[Eskom’s transmission projects as at 31 March 2014.]{Eskom’s transmission projects as at 31 March 2014 \cite{Eskom2014}.}\label{Abb1}
\end{figure}

http://integratedreport.eskom.co.za/supplementary/app-transmission.php

\section{Renewable energy potential in South Africa}

\subsection{Energy outlook for South Africa}
Development of a Renewable Energy Power Supply Outlook 2015 for the Republic of South Africa
Achieved by Sebastian Giglmayr, BSc
\cite{Giglmayr2013}

\subsection{Government Incentives}

\section{Chapter summary}
\pagebreak
